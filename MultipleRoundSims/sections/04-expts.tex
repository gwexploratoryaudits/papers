%Definitions
In this section, we motivate and describe the experiments. We consider a two candidate plurality contest, and assume that ballots are sampled with replacement, as is common in the literature. Note that \BRAVO and \Minerva can be extended for multiple-candidate, multiple-winner plurality contests by performing pairwise audits between the winners and the losers\cite{RLA, arxiv_athena}. Therefore, a two candidate plurality contest can be thought of as the most general case, and our simulations provide insight for multiple-candidate and multiple-winner contests too.

We first present relevant definitions.

\begin{definition}
An audit $\mathcal{A}$ takes a sample of ballots $X$ as input and gives one of the 
following decisions
\begin{enumerate}
\item
$Correct$: the audit is complete
\item
$Uncertain$: continue the audit
\end{enumerate}
\end{definition}

All of the audits discussed in this paper are modeled as binary hypothesis tests. Under the alternate hypothesis, $H_a$, the announced outcome is correct. That is, the true underlying ballot distribution is given by the announced ballot tallies. Under the null hypothesis, $H_0$, the true outcome is a tie (or the announced winner lost by one vote, and the number of ballots is large enough that the probability of drawing a ballot for the winner is that of drawing one for the loser).

The maximum risk of an audit is the probability that an audit stops, given that the underlying election is a tie. (Vora show that this is the maximum risk \cite{Bayesian-RLA}.) 
\begin{definition}[Risk]
The maximum risk $R$ of audit $\mathcal{A}$ with sample $X\in \{0,1\}^*$ drawn from 
the true underlying distribution of ballots is
$$R(\mathcal{A})=\Pr[\mathcal{A}(X)=Correct \mid H_0]$$
\end{definition}

This leads us to the following definition of an $\alpha$-RLA.
\begin{definition}[Risk Limiting Audit ($\alpha$-RLA)]
An audit $\mathcal{A}$ is a Risk Limiting Audit with 
risk limit $\alpha$ iff 
$$R(\mathcal{A}) \le \alpha.$$
\end{definition}

We present measures of stopping probability in the $j^{th}$ round of the audit, given that the underlying election is as announced.
\begin{definition}[Stopping Probability]
The stopping probability $S_j$ of an audit $\mathcal{A}$ in round $j$ is 
$$S_j(\mathcal{A})=\Pr[\mathcal{A}(X)=Correct ~in~round~j~\land \mathcal{A}(X) \neq Correct ~previously \mid H_a]$$
\end{definition}
Experimentally, using our simulations, $S_j$ would be estimated by the fraction of audits that stop in round $j$. 

Note that $\sum _j S_j(\mathcal{A})=1$. We can also consider the cumulative stopping probability: 
\begin{definition}[Cumulative Stopping Probability]
The cumulative stopping probability $C_j$ of an audit $\mathcal{A}$ in round $j$ is 
$$C_j(\mathcal{A})= \sum_{i=1}^j S_j$$
\end{definition}
Experimentally, using our simulations, $C_j$ would be estimated by the fraction of audits that stop in or before round $j$. 

Finally, we are also interested in the probability that an audit will stop in round $j$ given that it did not stop earlier: 
\begin{definition}[Conditional Stopping Probability]
The conditional stopping probability  of an audit $\mathcal{A}$ in round $j$ is 
$$\chi_j (\mathcal{A})=\Pr[\mathcal{A}(X)=Correct ~in~round~j~\mid H_a \land \mathcal{A}(X) \neq Correct ~previously]$$
\end{definition}
Experimentally, using our simulations, $\chi_j$ would be estimated by the ratio of the audits that stop in round $j$ to those that ``entered'' round $j$, i.e. those that did not stop before round $j$. 

We simulated audits for a risk limit of $0.1$ using margins from the 2020 US Presidential election, limiting ourselves to pairwise margins for the two main candidates of $0.05$ or larger. Round sizes increase roughly proportional to the inverse
square of the margin, so 
smaller margins are computationally much more expensive to simulate.
For each of these states, we simulated 
$10,000=10^4$ audits assuming the underlying election was as announced ($H_a$),  
and an additional $10,000=10^4$ audits assuming the underlying election was a tie ($H_0$). 

We ran simulations for: (a) $0.9$ probability of stopping in the first round, enabling election officials to be done in the first round with very high probability if the election is as announced and (b) $.25$ probability of stopping in the first round which is more favorable to \BRAVO. We ran our simulations for up to five rounds. 

For rounds after the first one, we chose the round schedule as follows. For both versions of \BRAVO, we chose a single round schedule: each round size has the same conditional stopping probability as the first one. As the proof of the risk-limiting property of \Minerva assumes that its round schedule is determined before any ballots are drawn, we could not use this approach for \Minerva round sizes. Instead, we chose to compare two fixed round schedules for \Minerva: one where the additional number of ballots drawn in a round is the same as in the previous round (multiplying factor of $1.0$) and the second where the multiplying factor is $1.5$. We consider the case of drawing samples of the same size because it may reflect a practical way to continue an
audit; if election officials have selected some first round size within
reasonable logistical bounds, drawing the same number of 
ballots in subsequent rounds may be practical.
We also consider round sizes with samples increasing by a multiple
of $1.5$ because this version is integrated into {\em Arlo}, and the multiplying factor was chosen as it roughly ensures a $0.9$ conditional stopping probability in the second round for a first round stopping probability of $0.9$. 

We used the R2B2 library \cite{r2b2_anon}, which provides a framework for the exploration of round-by-round
and ballot-by-ballot RLAs. It has implementations of several ballot polling risk-limiting audits as well as a simulator, 
all written in Python. For each of these audits, the software can compute the stopping condition for a given sample and estimates
of the next round size to achieve a desired stopping probability. 
For a given audit and random seed, the simulator draws random samples, with replacement, using a pseudorandom number generator, given the number of votes for each candidate, and the number of invalid votes, in the underlying election (these need not be chosen to be as announced). 

When the number of candidates is more than two, the audit is carried out pairwise for each candidate pair, and votes for all other candidates are considered invalid votes. 

After drawing a simulated sample of ballots, the simulator evaluates the given audit's stopping condition for this sample.
If the audit stops, the simulation stops, and if the audit continues, the simulation draws another round. 
The abstract simulator class does not prescribe any one method for choosing round sizes. 
We implement several classes to support various round size choices: 
round sizes from an estimate to achieve a desired probability of stopping, 
predetermined round sizes, and pseudorandomly-generated round sizes. 
