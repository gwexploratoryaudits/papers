
\begin{definition}[$(\alpha,p_a, p_0,k_{j-1},n_{j-1},n_j)$-\Providence]
    \label{def:minervatwo}
    For cumulative round size $n_i$ for round $i$ and a cumulative $k_i$ ballots for the reported winner found in round $i$, the \R \Providence stopping rule for the $j^{th}$ round is:
$$
\mathcal{A}(X_{j})=  \left\{ \begin{array}{ll} \text{Correct} ~~~~ \omega_{j}(k_{j}, k_{j-1}, p_a, p_0, n_j, n_{j-1}) \geq \frac{1}{\alpha}\\
        % & \\
        % incorrect& ~~~ \sigma_n < \frac{\beta}{1-\alpha} \\
        % & \\
        Undetermined ~~else \\
    \end{array}
    \right .
$$
where $\omega _{1}\triangleq \tau_{1}$ and for $j\ge 2$, we define $\omega _{j}$ as follows:
\begin{equation}
    \begin{aligned}
    \omega_{j}(k_{j}, k_{j-1}, p_a, p_0, n_{j}, n_{j-1})\triangleq
    \sigma(k_{j-1},p_a,p_0,n_{j-1})\cdot \tau_1(k_{j}-k_{j-1},p_a,p_0,n_j-n_{j-1})
    \end{aligned}
\end{equation}
\end{definition}
% old definition of omega_j for j\ge 2

Notice that for $j\ge 2$, unlike $\tau_j$, computing $\omega_j$ requires no convolution.

Now we prove that \Providence is risk-limiting.

\begin{lemma}
    \label{lemma:sigma_increasing}
For $0<p_0< p_a< 1$ and $n>0$, the ratio $\sigma(k,p_a,p_0,n)$ is strictly increasing as a function of $k$ for $0\le k\le n$.
\end{lemma}
\begin{proof}
From $0<p_0< p_a< 1$, we get
$$
\frac{p_a}{p_0}>1
$$
and
$$
1-p_0>1-p_a\implies \frac{1-p_0}{1-p_a}>1,
$$
and thus
$$
\frac{p_a(1-p_0)}{p_0(1-p_a)}>1.
$$
Now simply observe that
$$
\frac{p_a(1-p_0)}{p_0(1-p_a)}\cdot\sigma(k, p_a, p_0, n)
=\frac{p_a(1-p_0)}{p_0(1-p_a)}\cdot\frac{p_a^{k} (1-p_a)^{n-k}}{p_a^{k} (1-p_a)^{n-k}}
=\frac{p_a^{k+1} (1-p_a)^{n-(k+1)}}{p_a^{k+1} (1-p_a)^{n-(k+1)}}
=\sigma(k+1, p_a, p_0, n).
$$
\end{proof}


\begin{lemma}
    \label{lemma:frac_sums_increasing}
    Given a monotone increasing sequence: $\frac{a_1}{b_1}, \frac{a_2}{b_2}, \ldots, \frac{a_n}{b_n}$, for $a_i, b_i > 0$, the sequence:
    $$z_i = \frac{\sum_{j=i}^n a_j}{\sum_{j=i}^n b_j}$$
    is also monotone increasing.
\end{lemma}

\begin{proof}
    Note that $z_i$ is a weighted average of the values of $\frac{a_j}{b_j}$ for $j \geq i$:
    $$z_i = \sum_{j=i}^n y_j \frac{a_j}{b_j}$$  ~for~ $$y_j = \frac{b_j}{\sum _{j=i}^n b_j} > 0.$$
    Further, $\sum _{j=i}^n y_j = 1$ and hence  $y_j \leq 1$ ~and~ $y_j = 1 \Leftrightarrow i=j=n$.
    Observe that, because $\frac{a_i}{b_i}$ is monotone increasing,
    $z_i \geq \frac{a_i}{b_i}$
    with equality if and only if $i=n$.
    Suppose $i < n$. Then
    $$z_{i+1} \geq \frac{a_{i+1}}{b_{i+1}} > \frac{a_i}{b_i}, $$
    and
    $$z_i = y_i \frac{a_i}{b_i} + (1-y_i) z_{i+1} < z_{i+1}.$$
    Thus $z_i$ is also monotone increasing.
\end{proof}

\begin{lemma}
    \label{lemma:tau1_increasing}
For $0<p_0< p_a< 1$ and $n>0$, the ratio $\tau_1(k,p_a,p_0,n)$ is strictly increasing as a function $k$ for $0\le k\le n$.
\end{lemma}
\begin{proof}
    Apply Lemmas \ref{lemma:sigma_increasing}-\ref{lemma:frac_sums_increasing}.
\end{proof}
\begin{lemma}
    \label{lemma:imin-exists}
    Given a strictly monotone increasing sequence: $x_1, x_2, \ldots x_n $ and some constant $A$,
    %there exists an $i_{min}$ such that
    $$A \le x_i \Leftrightarrow \exists i_{min} \le i ~\text{s.t.}~   x_{i_{min} -1} < A \le x_{i_{min}} \le x_{i},$$
    unless $A\le x_1$, in which case $i_{min} =1 $.
\end{lemma}
\begin{proof}
    Evident.
\end{proof}

\begin{lemma}
    \label{lemma:minerva2_kmin_exists}
    For $\mathcal{A}=(\alpha,p_a, p_0,k_{j-1},n_{j-1},n_j)$-\Providence, there exists\\ a $k_{min,j}(\Providence, p_a, p_0, k_{j-1}, n_{j-1}, n_j)$ such that $$\mathcal{A}(X_j)=\text{Correct}\iff k_j\ge k_{min,j}(\Providence,  \bm{n_j}, p_a, p_0).$$
\end{lemma}
\begin{proof}
    From Definition~\ref{def:minervatwo}, $$\mathcal{A}(X_j)=\text{Correct}\iff \omega_j(k_{j}, k_{j-1}, p_a, p_0, n_j, n_{j-1}) \ge \frac{1}{\alpha}.$$
    Now to apply Lemma~\ref{lemma:imin-exists}, it suffices to show that
    $\omega_j$ is monotone increasing with respect to $k_j$.
    For $j=1$, we have $\omega_1=\tau_1$, so $\omega_1$ is strictly increasing by Lemma \ref{lemma:tau1_increasing}. For $j\ge 2$,
    $$\omega_j(k_j,k_{j-1},p_a,p_0,n_j,n_{j-1},\alpha)=\sigma(k_{j-1},p_a,p_0,n_{j-1})\cdot \tau_1(k_{j}-k_{j-1},p_a,p_0,n_j-n_{j-1}).$$
    As a function of $k_j$, $\sigma$ is constant, and thus $\omega$ is strictly increasing by Lemma \ref{lemma:tau1_increasing}. Therefore by Lemma \ref{lemma:imin-exists}, we have the desired property.
\end{proof}


%Here we introduce the notion of the history of an audit.
%In round $j+1$, all previous round sizes and cumulative
%tallies of winner ballots are together relevant information to the new round.
%We call this information the history of the audit through round $j$,
%and we denote it as follows:
%$$\bm{k}_j\triangleq(k_1,k_2,\ldots,k_j,n_1,n_2,\ldots,n_j)$$
\begin{lemma}
\label{lemma:any_ratio_is_sigma_simple}
For $j\ge 1$,
$$\frac{Pr[\bm{K_j}=\bm{k_j} \mid \bm{n_j}, H_a]}{Pr[\bm{K_j}=\bm{k_j} \mid \bm{n_j}, H_0]} = \sigma(k_j, p_a, p_0, n_j).$$
%$$\frac{Pr[K_j = k_{j} \mid n_{j-1}^*, k_{j-1}^*, H_a]}{Pr[K_j = k_{j} \mid n_{j-1}^*, k_{j-1}^*, H_0]} = \sigma(k_j, p_a, p_0, n_j).$$
\end{lemma}
\begin{proof}
We induct on the number of rounds.
For $j=1$, we have
$$\frac{Pr[\bm{K_1}=\bm{k_1} \mid \bm{n_1},H_a]}{Pr[\bm{K_1}=\bm{k_1} \mid  \bm{n_1},H_0]} =\frac{Pr[K_1 = k_{1} \mid n_1,H_a]}{Pr[K_1 = k_1 \mid n_1,H_0]} = \frac{\text{Bin}(k_1,n_1,p_a)}{\text{Bin}(k_1,n_1,p_0)}=\sigma(k_1, p_a, p_0, n_1).$$
%
% Suppose the lemma is true for $j=m$.
% That is,
% $$\frac{Pr[\bm{K_m}=\bm{k_m} \mid H_a]}{Pr[\bm{K_m}=\bm{k_m} \mid  H_0]} = \sigma(k_m, p_a, p_0, n_m).$$
%
Suppose the lemma is true for round $j=m$ with history $\bm{k_m}$.
%We will show it is true for $j=m+1$.
Observe that
 $$\frac{Pr[\bm{K_{m+1}}=\bm{k_{m+1}} \mid \bm{n_{m+1}},H_a]}{Pr[\bm{K_{m+1}}=\bm{k_{m+1}} \mid \bm{n_{m+1}}, H_0]} = \frac{ Pr[\bm{K_{m}}=\bm{k_{m}}\mid \bm{n_{m+1}},H_a] \cdot Pr[K_{m+1}'=k_{m+1}'|\bm{k_m},\bm{n_{m+1}},H_a]}{ Pr[\bm{K_{m}}= \bm{k_{m}} \mid  \bm{n_{m+1}},H_0]  \cdot  Pr[K_{m+1}'=k_{m+1}'|\bm{k_m},\bm{n_{m+1}},H_0]  }$$
 $$=\sigma(k_m, p_a, p_0, n_m) \cdot \frac{Pr[K_{m+1}'=k_{m+1}'|\bm{k_m}, \bm{n_{m+1}}, H_a]}{Pr[K_{m+1}'=k_{m+1}'|\bm{k_m},\bm{n_{m+1}},H_0]}$$
 by the induction hypothesis.
Then this is simply equal to
 $$\sigma(k_m, p_a, p_0, n_m)\cdot\frac{\text{Bin}(k_{m+1}',n_{m+1}',p_a)}{\text{Bin}(k_{m+1}',n_{m+1}',p_0)}
 $$$$
 =\frac{p_a^{k_m} (1-p_a)^{n_m-k_m}}{p_0^{k_m} (1-p_0)^{n_m-k_m}} \cdot
 \frac{p_a^{k_{m+1}'} (1-p_a)^{n_{m+1}'-k_{m+1}'}}{p_0^{k_{m+1}'} (1-p_0)^{n_{m+1}'-k_{m+1}'}}
 $$
 $$
 =\sigma(k_{m+1}, p_a, p_0, n_{m+1})
 $$
\end{proof}

\begin{theorem}
\label{thm:minerva2_is_rla_new}
An $(\alpha,p_a, p_0,k_{j-1},n_{j-1},n_j)$-\Providence audit is an
$\alpha$-RLA.
\end{theorem}
\begin{proof}
Let $\mathcal{A}=(\alpha,p_a, p_0,k_{j-1},n_{j-1},n_j)$-\Providence.
Let $\bm{n_j}$ be the cumulative roundsizes used in this
audit, with corresponding cumulative tallies of
ballots for the reported winner $\bm{k_j}$.
For round $j=1$, by Definitions \ref{def:minervatwo}
and \ref{def:minerva_ratio}, we see that
the $\mathcal{A}=\text{Correct}$ (the audit stops) only when
$$
\tau_1(k_{1},p_a,p_0,n_1)
=\frac{Pr[K_{1} \geq k_{1} \mid H_a, n_1]}{Pr[K_{1} \geq k_{1} \mid H_0, n_1]}
\ge \frac{1}{\alpha}.
$$
By Lemma \ref{lemma:minerva2_kmin_exists}, we see that this
is equivalent to the following:
$$
\frac{Pr[K_{1} \geq k_{min,1} \mid H_a, n_1]}{Pr[K_{1} \geq k_{min, 1} \mid H_0, n_1]}
\ge \frac{1}{\alpha}.
$$

For any round $j\ge 2$, by Definition \ref{def:minervatwo}
and Lemma \ref{lemma:minerva2_kmin_exists},
$\mathcal{A}=\text{Correct}$ (the audit stops) only when
$$
\omega_{j}(k_{j}, k_{j-1}, p_a, p_0, n_{j}, n_{j-1}, \alpha )\triangleq
\sigma(k_{j-1},p_a,p_0,n_{j-1})\cdot \tau_1(k_{j}-k_{j-1},p_a,p_0,n_j-n_{j-1})
\ge \frac{1}{\alpha}.
$$
%Let $d_j=k_j-k_{j-1}$ be shorthand for the new draw.
%could do this for simpler notation^
By Lemma \ref{lemma:any_ratio_is_sigma_simple}
and Definition \ref{def:minerva_ratio}, this is equivalent to
$$
\frac{\Pr[\bm{k_{j-1}} \mid H_a]\cdot Pr[K_{j} \ge k_{j} \mid \bm{k_{j-1}}, H_a, n{j}]}{\Pr[\bm{k_{j-1}} \mid H_0]\cdot Pr[K_{j} \ge k_{j} \mid \bm{k_{j-1}}, H_0, n{j}]}\ge \frac{1}{\alpha}.
$$
By Lemma~\ref{lemma:minerva2_kmin_exists} and Definition~\ref{def:minervatwo},
we see that there exists a $k_{min, j}\le k_j$ for which
$$
\frac{\Pr[\bm{k_{j-1}} \mid H_a]\cdot Pr[K_{j} \ge k_{min, j} \mid \bm{k_{j-1}}, H_a, n{j}]}{\Pr[\bm{k_{j-1}} \mid H_0]\cdot Pr[K_{j} \ge k_{min, j} \mid \bm{k_{j-1}}, H_0, n{j}]} \ge
\frac{\Pr[\bm{k_{j-1}} \mid H_a]\cdot Pr[K_{j} \ge k_{j} \mid \bm{k_{j-1}}, H_a, n{j}]}{\Pr[\bm{k_{j-1}} \mid H_0]\cdot Pr[K_{j} \ge k_{j} \mid \bm{k_{j-1}}, H_0, n{j}]}\ge \frac{1}{\alpha}.
$$

%Equivalently,
%$$
%\frac{Pr[K_{j} \ge k_{min, j}\wedge K_{j-1} = k_{j-1} \mid n_{j}^*, k_{j-2}^*, H_a]}{Pr[K_{j}\ge k_{min, j} \wedge K_{j-1} = k_{j-1} \mid n_{j}^*, k_{j-2}^*, H_0]}\ge \frac{1}{\alpha}.
%$$
%Taking the sum over all possible rounds and corresponding preceding values
%of $k$, we get
Taking the sum over all possible audit histories, we get
$$
\frac{\sum_{\bm{k_j}}\Pr[\bm{k_{j-1}} \mid H_a]\cdot Pr[K_{j} \ge k_{min, j} \mid \bm{k_{j-1}}, H_a, n{j}]}{\sum_{\bm{k_j}}\Pr[\bm{k_{j-1}} \mid H_0]\cdot Pr[K_{j} \ge k_{min, j} \mid \bm{k_{j-1}}, H_0, n{j}]}\ge \frac{1}{\alpha}.
$$
Finally, because the total probability of stopping the audit under
the alternative hypothesis is less than 1, we get
$$
\frac{Pr[\mathcal{A}=\text{Correct} \mid H_a]}{Pr[\mathcal{A}=\text{Correct} \mid H_0]}\ge \frac{1}{\alpha}
$$
$$
Pr[\mathcal{A}=\text{Correct} \mid H_0]
\le
Pr[\mathcal{A}=\text{Correct} \mid H_a] \cdot \alpha
\le
\alpha.
$$
\end{proof}

\subsection{Resistance against an adversary choosing round sizes}
