\begin{abstract}
    Risk-Limiting Audits (RLAs) guarantee with known probability 
    that if the outcome of an 
    election is incorrectly announced, it will be detected, 
    and a full hand recount will be performed. 
    In Ballot-polling RLAs, samples of ballots are drawn and tallied.
    \fpo{Then the sample tally is (statistically) compared with announced tally.}
    
    In this paper we present an audit simulation framework
    that can be used to support theoretical results of audits.
    We present such experimental results for multiple round 
    ballot-polling RLAs.
    \BRAVO [citation] has long been the standard ballot-polling RLA,
    while \Minerva was recently introduced with the claim
    that fewer ballots on average are necessary to conclude 
    an audit.
    [Minerva paper] presented experimental
    results for one round audits, and here
    we present results
    from simulations of multiple round audits using \Minerva 
    as well as \BRAVO 
    (both Selection-Ordered 
    \BRAVO and End-of-Round \BRAVO).
    The simulation results agree within reasonable error with
    the mathematical properties claimed in [Minerva paper].
    On average, \BRAVO audits are unnecessarily conservative 
    while \Minerva audits stop with fewer ballots. We also
    present details on software implementing \Minerva and
    the simulations themselves.

\keywords{Risk-limiting audit  \and Ballot polling audit}
\end{abstract}
