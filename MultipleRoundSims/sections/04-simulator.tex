%Simulator
\subsection{Simulations to Support Theoretical Audit Properties}

The outcomes of RLAs depend on random chance; some random samples
support the alternative hypothesis more than expected, resulting in 
quick low-risk conclusions, while other samples require subsequent
rounds in order to confirm the announced results.
We can simulate random samples for various underlying ballot 
distributions by computing pseudorandom samples. 
By applying an audit's stopping condition to thousands of such
simulated samples, the average behavior of the simulated
audits will tend towards the true behavior of the audit.
In this way, we can examine whether theoretical claims about an audit are
actually correct.

\subsection{Software for Simulations}
Our open source audit software library r2b2 [link] has implementations of several ballot polling risk-limiting audits as well as a simulator, 
all written in Python.
For each of these audits, the software can evaluate the stopping condition for a given sample and can give estimates
of the minimum round size to achieve a desired stopping probability. 
For a given audit and random seed, the simulator draws random samples using the pseudorandom number generator, [need to check].
Ballots can be sampled from any distribution of the users choosing. 
It is often useful to consider the distribution of ballots corresponding to a tie and the 
distribution of ballots corresponding with the announced results; these are the distributions represented
by the null and alternative hypotheses.
After drawing a sample, the simulator then evaluates the given audit's stopping condition for this simulated sample.
If the audit stops, the simulation stops, and if the audit continues, the simulation draws another round. 
The abstract simulator class does not prescribe any one method for choosing round sizes. 
We implement several classes to support various round size choices: 
round sizes from an estimate to achieve a desired probability of stopping, 
predetermined round sizes, and random round sizes. 

