%Intro
It is well-known that electronic voting systems are vulnerable to software errors and manipulation which may be undetected. Errors and/or manipulation may not always change an election outcome, but we would want to know when they do. {\em Software independent} voting systems \cite{SI-Wack,rivest2008notion} are ones where an undetected change in the software cannot lead to an undetected change in the election outcome. {\em Evidence-based elections} \cite{evidence-based} use software independent systems to produce trustworthy evidence of outcome correctness; incorrect outcomes are detected with high probability when the evidence is examined. One approach to evidence-based elections is to use voter-verified paper ballots, store them securely, and perform public audits---a compliance audit to determine whether the ballots were stored securely; and a rigorous tabulation audit, known as a risk-limiting audit (RLA) \cite{RLA}, to determine whether the outcome is correctly computed from the stored ballots. Many US states have had pilot RLA programs. Additionally, some states allow RLAs to be used towards audit requirements, and some states require RLAs before elections can be certified. 

We propose the \Providence audit, a new approach to the ballot polling RLA, and propose a new model for the work load of an election. We show that \Providence is superior to the popular ballot polling RLA \Bravo for real elections, and describe the use of our open source implementation by the Rhode Island Board of Elections for an audit of their 2021 elections. Our implementation of \Providence is likely to be useful later this year; ballot polling audits are expected to be used as pre-certification RLAs in at least one statewide contest in both Georgia and Pennsylvania for the 2022 general elections in the US.   

\subsection{Background}
Ballot comparison RLAs require the fewest ballots of all known RLA approaches. On the other hand, they require the use of special election technology and are not always feasible. Ballot-polling RLAs require a larger number of ballots, but are more feasible because they do not require any additional functionality of the voting system. What is needed is a complete ballot manifest (a list of ballot storage containers and the number of ballots in each) which enables the creation of a well defined list of the ballots and their locations (the fifth ballot in box number 20, for example). 

A ballot polling audit begins when a {\em round} \cite{usenix_minerva} of multiple randomly-chosen ballots is drawn. A risk measure is then computed to determine whether (a) the audit ends in success (the election outcome is declared correct) or (b) another round should be drawn. Election officials would typically decide to perform a full manual hand count if the audit does not stop in spite of drawing a large number of ballots, typically over multiple rounds. Ballot polling audits have been used in a number of US state pilots (California, Georgia, Indiana, Michigan, Ohio, Pennsylvania, Virginia and elsewhere) and in real statewide audits (Georgia, Virgina) \cite{vv_audits} as well as in audits of smaller jurisdictions, such as Montgomery County, Ohio \cite{usenix_minerva}. 

A {\em round-by-round (R2)} audit is one where the decision of whether to draw more ballots or not is taken after drawing a round of ballots; typically hundreds or thousands or tens of thousands of ballots in statewide elections. A {\em ballot-by-ballot (B2)} audit is the special case of round size one---when the decision is made after each ballot is drawn. The popular \BRAVO audit requires the smallest expected number of ballots when the announced tally of the election is correct, and stopping decisions are taken a ballot at a time (that is, when it is used as a B2 audit). Election officials typically draw ballots in large round sizes, and \BRAVO needs to be modified for use in this manner. For use as an R2 audit, the \BRAVO stopping condition can be applied once at the end of each round (End-of Round (EoR)), or retroactively after each ballot drawn if ballot order is retained (Selection-Ordered (SO)). SO \BRAVO is closer to the original B2 \BRAVO, and requires fewer ballots on average than EoR \BRAVO. But it requires the additional effort of tracking the order of ballots. 

Zag\'{o}rski {\em et al.} propose ballot polling RLA \Minerva \cite{usenix_minerva}, which does not need ballot order and relies only on sample and round tallies. They prove that it requires fewer ballots than EoR \BRAVO when both audits have the same pre-determined round schedule and the true tally is as announced. 
%\footnote{Their proof assumes that the number of relevant ballots drawn in each round is know beforehand. In MINERVA, the number of ballots drawn in each round is determined before any ballots are drawn. Because invalid ballots and ballots that are inconsequential for the contest being audited would be drawn in addition to relevant ballots, the assumption used by the proof is not true in general. (We are grateful to Philip Stark for drawing our attention to this.) However, any variation in number of relevant ballots drawn for a fixed round size would be random and not chosen by an adversary; the proof showing the risk-limiting property of MINERVA could hence be extended.}
They also present first-round simulations demonstrating that \Minerva draws fewer ballots than SO \BRAVO in the first round for large first round sizes when the true tally is as announced. 
Broadrick {\em et al.} provide further simulations that show \Minerva requires fewer ballots over multiple rounds and for lower stopping probability \cite{simulations}, though the improvement from using \Minerva over either version of \BRAVO decreases with round size. 

A major limitation of \Minerva is that one needs to determine the round schedule before the audit begins, because \Minerva has not been shown to be risk-limiting if an adversary can choose subsequent round sizes after knowing the sample drawn. \BRAVO, on the other hand, is not limited in this manner. This allows \BRAVO audits the flexibility of choosing smaller subsequent round sizes if the sample drawn so far is a ``good'' sample. An open question is whether a ballot polling RLA exists with the efficiency of \Minerva and this flexibility of \BRAVO.

A major limitation of our understanding of the ballot polling problem as a community is that we use the number of ballots drawn or values proportional to this number \cite{mclaughlin_thesis,bernhard-diss,RI-report} as measures of the workload of an audit. If this were a correct measure of the workload of an audit, we would want to use B2 audits (round size is one) and make decisions about stopping the audit after drawing each ballot, because this leads to the smallest expected number of ballots. Election officials, on the other hand, greatly prefer drawing many ballots at once. This preference is likely because each round has an overhead workload as well, including setting up the round and communicating among the various localities involved in conducting the audit (for example, audits of statewide contests involve the drawing of ballots at county offices where the ballots are stored). Further, there is an overhead to finding a storage box and unsealing it. For large round sizes, multiple ballots may be drawn at once from a box, and the number of boxes retrieved is smaller than the number of ballots (storage boxes commonly contain many hundreds of ballots each). For smaller round sizes, the number of times a box is retrieved would be roughly identical to the number of ballots drawn, as it is unlikely that a single box will hold multiple ballots from the sample. Finally, in the current environment of misinformation, election officials would want to ensure that the probability of a misleading audit sample (falsely indicating that the loser won) is very small, which implies that round sizes should be large. Thus the workload of an audit is not simply a linear (or affine) function of the number of ballots drawn. Relatedly, an optimal round schedule is not completely determined by the expected number of ballots drawn. It depends on other variables as well. The consideration of all these variables is necessary while planning an audit. 

\subsection{Our Contributions}
We present \Providence, and provide the following:
\begin{enumerate}
\item Proof that \Providence is an RLA and resistant to an adversary who can choose subsequent round sizes with knowledge of previous samples.
\item Simulations of \Providence, \Minerva, SO \BRAVO, and EoR \BRAVO which show that \Providence uses number of ballots similar to those of \Minerva, both fewer than either version of \BRAVO.
\item Results and analysis from the use of \Providence in a pilot audit in Rhode Island.
\item Open source implementation of \Providence. 
\item A model of workload that includes the overhead effort of each round and the overhead effort of retrieving a storage unit of ballots; simulations that illustrate the use of this model to compare the different types of ballot polling audits and to plan an audit with minimal workload.
\item An analysis of round size as a function of the maximum acceptable probability of a misleading audit sample.
%including the novel metric Probability of Misleading(name?)
\end{enumerate}

Our results demonstrate the superiority of \Providence over the other audits. Our work may be used by election officials to plan ballot polling audits, including in Georgia and Pennsylvania in 2022. 

\subsection{Organization} 
Section \ref{sec:related} describes related work. Section \ref{sec:prov} describes the \Providence audit, section \ref{sec:sims} the simulations comparing the number of ballots drawn using various ballot polling audits and section \ref{sec:pilot} the use of \Providence in an audit carried out by the Board of Elections of Rhode Island. Section \ref{sec:workload} presents our workload model and describes its use for a ballot polling audit using details of the 2020 US Presidential election in the state of Virginia. Our conclusions, the availability of an audit implementation and acknowledgements may be found in sections \ref{sec:conc}, \ref{sec:avail} and \ref{sec:ack} respectively. 

