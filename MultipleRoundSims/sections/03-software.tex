%Software
In this section we describe the software implementing ballot polling audits, termed the R2B2 library, and the simulator software used for this research. All the software is released as open source under the MIT License.
\subsection{R2B2 Library}

The R2B2 Python library \cite{r2b2_anon} provides a framework for the exploration of round-by-round
and ballot-by-ballot RLAs. The goal in designing R2B2 is two fold:
\begin{enumerate}
    \item Provide an elegant Python library which can be easily imported and used
    in any other code base.
    \item Provide an interactive set of tools which can be utilized `out-of-the-box'
    for experimenting with and learning about risk-limiting audits.
\end{enumerate}

\subsubsection{Design}

The high-level design of R2B2 is an object-oriented view of election audits.
The three main object classes, \code{Election}, \code{Contest}, and \code{Audit},
serve to group data into logically independent structures.

The \code{Election} contains the information that comprises an entire election,
most importantly, the total number of ballots cast in the election and the list
of \code{Contest}s from the election. At the moment \code{Election} does not offer
functionality beyond grouping \code{Contest}s.

The \code{Contest} contains the information related to a single
contest such as the ballots cast in that contest, the candidates, the type of contest,
and the reported tally. Providing a structure to hold this information independent of
any particular audit is especially useful for exploratory work.

The \code{Audit} contains information related to the audit parameters for a single contest, 
such as the risk limit, sampling method, and \code{Contest} to audit. It is important
to note the \code{Audit} is an Abstract Base Class upon which specific RLAs are built.
It only contains the parameters and attributes common to the RLAs of this paper and provides a set
of methods that can be called by any audit implementation. The functionality of
\code{Audit} can be divided into two basic groups: \textit{interactive}
and \textit{bulk}.\\
\\
The interactive implementation allows users to execute an audit step-by-step as it
might progress during a live election audit through the following:

\begin{itemize}
    \item The \code{run()} method begins an interactive audit executing where users
    are prompted for round sizes and the counts of winner ballots found in the sample
    and in return are given information about the current state of the audit and whether
    the stopping condition(s) have been met.
    \item Two distributions representing the null and alternative hypotheses are maintained
    and allow for computation of the audits per-round risk and stopping probability
    schedules.
    \item Before each round, the audit will recommend possible next round sizes given
    different criteria, such as a set of desired stopping probabilities.
\end{itemize}
The bulk implementations allows users to generate a larger set of data from an audit
such as:

\begin{itemize}
    \item A set of stopping conditions given a set of round sizes.
    \item A set of risk levels given a set of round size and winner ballots pairs.
    \item A list of all stopping conditions from the minimum to the maximum round size.
\end{itemize}

\subsubsection{Usage}

R2B2 makes understanding and exploring election audits simple for the user with no
Python knowledge while simultaneously providing a comprehensive set of tools for
the experienced Python developer.

Using R2B2 is as simple as using any other Python library: simply import the library
and all of the functionality is at your finger tips. Not only does this allow users
to write their own Python scripts for exploring RLAs, it also allows R2B2 to be plugged
in to any other Python library. See the following Jupyter Notebooks for information on
the usage of R2B2:

\begin{itemize}
    \item \href{https://github.com/gwexploratoryaudits/r2b2/blob/notebooks/notebook/R2B2%20Basics.ipynb}{Basic Usage}
    \item \href{https://github.com/gwexploratoryaudits/r2b2/blob/notebooks/notebook/Generating%20Graphs.ipynb}{Generating Graphs}
\end{itemize}

R2B2 also provides a significant amount of functionality `out-of-the-box' for educational
or exploratory use. For those who wish learn about RLAs without having to write any
code themselves, R2B2 provides a command line tool for both interactive auditing and
generating audit results and statistics for larger data sets.

\subsection{Simulation Software}
As described above R2B2 has implementations of several ballot polling risk-limiting audits as well as a simulator, 
all written in Python.
For each of these audits, the software can compute the stopping condition for a given sample and estimates
of the next round size to achieve a desired stopping probability. 
For a given audit and random seed, the simulator draws random samples, with replacement, using a pseudorandom number generator, [need to check].
given the number of votes for each candidate, and the number of invalid votes, in the underlying election (these need not be chosen to be those announced). 

When the number of candidates is more than two, the audit is carried out pairwise for each candidate pair, and votes for all other candidates are considered invalid votes. 

After drawing a simulated sample of ballots, the simulator evaluates the given audit's stopping condition for this sample.
If the audit stops, the simulation stops, and if the audit continues, the simulation draws another round. 
The abstract simulator class does not prescribe any one method for choosing round sizes. 
We implement several classes to support various round size choices: 
round sizes from an estimate to achieve a desired probability of stopping, 
predetermined round sizes, and pseudorandomly-generated round sizes. 

\subsection{Testing}

The R2B2 software is used to compute stopping conditions and next round estimates. It is intended for use by us and other researchers, and designed for this purpose. We have also independently implemented all the functionality in matlab \cite{brla_explore_anon} (the two codebases are written by different individuals) and have extensively checked the results of the two codebases. Additionally, for use in regular election audits by election officials, we have written an add-on \cite{athena_anon} to the {\em Arlo} risk-limiting audit software, the results of which have also been extensively checked against the other two codebases. 


