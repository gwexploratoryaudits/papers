%Definitions
Now we present relevant definitions.
We consider a two candidate plurality contest.
To begin we define an audit.
\begin{definition}
An audit $\mathcal{A}$ takes a sample $X$ as input and gives one of the 
following decisions
\begin{enumerate}
\item
$Correct$: the audit is complete
\item
$Uncertain$: continue the audit
\end{enumerate}
\end{definition}
All of the audits discussed in this paper are modeled as binary hypothesis tests.
Under the alternate hypothesis, $H_a$, the announced outcome is correct. 
That is, the true underlying ballot distribution is given by the announced ballot tallies.
Under the null hypothesis, $H_0$, the true outcome is a tie 
(or a the announced winner losing by one vote if there is an odd number of total ballots).
Now we define key attributes of an audit that we will consider while analyzing our simulations.
The risk of an audit is the probability that an audit stops when a tie has occurred.
\begin{definition}[Risk]
The risk $R$ of an audit $\mathcal{A}$ is
$$R(\mathcal{A})=\Pr[\mathcal{A}(X)=Correct \mid H_0]$$
\end{definition}
This leads us to the following simple definition of an $\alpha$-RLA.
\begin{definition}[Risk Limiting Audit ($\alpha$-RLA)]
An audit $\mathcal{A}$ is a Risk Limiting Audit with 
risk limit $\alpha$ iff 
$$R(\mathcal{A}) \le \alpha.$$
\end{definition}

It is useful to discuss the probability of an audit stopping in
some round, if the outcome is correctly announced.
\begin{definition}[Stopping Probability]
The stopping probability $S$ of an audit $\mathcal{A}$ in round $j$ is 
$$S_j(\mathcal{A})=\Pr[\mathcal{A}(X_j)=Correct \mid H_a]$$
\end{definition}
The notion of stopping probability can be useful for selecting round sizes.

