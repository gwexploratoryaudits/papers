% This is a draft of the coming paper "Multiple Round Ballot Polling Risk-Limiting Audit Simulations".
% 2021.
%
\documentclass[runningheads]{llncs}
%
\usepackage{float}
\usepackage{xspace}
\usepackage{graphicx}
\graphicspath{{./imgs/}}
\usepackage{comment}
\usepackage{listings,hyperref}

% \usepackage{graphicx}  --> Used for displaying a sample figure. If possible, figure files should
% be included in EPS format.
%
% If you use the hyperref package, please uncomment the following line
% to display URLs in blue roman font according to Springer's eBook style:
% \renewcommand\UrlFont{\color{blue}\rmfamily}


% nice looking audit titles
\newcommand{\Minerva}{\textsc{Minerva}\xspace}
\newcommand{\B}{{{B2}}\xspace}
\newcommand{\R}{{{R2}}\xspace}
\newcommand{\BRAVO}{\textsc{Bravo}\xspace}

\usepackage{color}
\definecolor{green}{rgb}{0.31, 0.78, 0.47}
\newcommand{\fpo}[1]{\textcolor{green}{#1}}

\def\code#1{\texttt{#1}}

\begin{document}
\lstset{language=Python}
%
\title{Simulations of Ballot Polling Risk-Limiting Audits}
%add \thanks{} within the title brackets if want to refer to supporting organization ^^
%
%\titlerunning{Abbreviated paper title}
% If the paper title is too long for the running head, you can set
% an abbreviated paper title here
%
\begin{comment}
\author{Oliver Broadrick\inst{1} \and
Sarah Morin\inst{1} \and
Grant McClearn\inst{2} \and Neal McBurnett \and Poorvi L. Vora\inst{1} \and
Filip Zagorski\inst{3} }
%
\authorrunning{Broadrick, Morin et al.}
% First names are abbreviated in the running head.
% If there are more than two authors, 'et al.' is used.
%
\institute{Department of Computer Science, The George Washington University 
\email{obroadrick@gwmail.gwu.edu}\\
\and
Department of Computer Science, Stanford University 
\email{grantmcc@stanford.edu}\\
\and Wroclaw University of Science and Technology\\
\email{filip.zagorski@gmail.com}}
%
\end{comment}
\maketitle              % typeset the header of the contribution
%
\begin{abstract}
    In this paper we present simulation results comparing the risk, stopping probability and number of ballots required over multiple rounds of ballot-polling risk limiting audits (RLAs) \Minerva, Selection-Ordered \BRAVO and End-of-Round \BRAVO. We also
    present details on the R2B2 open source software library and the related simulation software.    
    
    \BRAVO is the most commonly used ballot-polling RLA. \Minerva was recently proposed, and requires fewer first-round ballots, on average, than both Selection-Ordered \BRAVO and End-of-Round \BRAVO when the first-round stopping probability is large. 
    
    An open question, however, is how \Minerva compares to Selection-Ordered \BRAVO and End-of-Round \BRAVO over multiple rounds.  In this paper, we present results from simulations of multiple round audits with first-round stopping probabilities of $0.9$, a common choice among election officials. Because the size of rounds in \Minerva needs to be predetermined (and independent of the audit draws), we use two pre-determined round sequences for \Minerva: (a) all rounds are of the same size (which might be a practical choice for election officials if they had planned to have resources for the first round size) and (b) each round is $1.5$ times the previous one (which is the preset value for \Minerva as integrated into election audit software Arlo). 
    
    We show that the simulation results are consistent with predictions of the R2B2 open-source library for ballot polling audits. We also observe that both \BRAVO audits are more conservative  than needed,
    while \Minerva audits stop with fewer ballots. 

\keywords{Risk-limiting audit  \and Ballot polling audit}
\end{abstract}
%
%
%
\section{Introduction}
\label{sec:intro}
%Intro
The literature contains numerous descriptions of vulnerabilities in deployed voting systems, and it is not possible to be certain that any system, however well-designed, will perform as expected in all instances. For this reason, 
{\em evidence-based elections} \cite{evidence-based} aim to produce trustworthy and compelling evidence of the correctness of election outcomes, enabling the detection of problems with high probability. One way to implement an evidence-based election is to use a well-curated voter-verified paper trail, compliance audits, and a rigorous tabulation audit of the election outcome, known as a risk-limiting audit (RLA) \cite{RLA}. An RLA is an audit which guarantees that the probability of concluding that an election outcome is correct, given that it is not, is below a pre-determined value known as the risk limit of the audit, independent of the true, unknown vote distribution of the underlying election. Over a dozen states have seriously explored the use of RLAs---some have pilot programs, some allow RLAs to satisfy a general audit requirement and some have RLAs in statute.     

This paper provides insight into the main approach to ballot polling RLAs, the \BRAVO audit \cite{bravo}, and the newer \Minerva \cite{usenix_minerva} ballot polling RLA, through the presentation of simulation results. While some properties of the two audits may be theoretically derived, for other properties theoretical results are not available. This paper examines the number of ballots drawn over multiple rounds of both audits, the related probabilities of stopping if the election is as announced, and the maximum risk of the audit. 

\subsection{Background}
Of the multiple types of RLAs, this paper focuses on ballot-polling RLAs, which require a large number of ballots but do not rely on any special features of the election technology. In the general ballot-polling RLA, a number of ballots are drawn and tallied in what is termed a {\em round} of ballots \cite{usenix_minerva}. A statistical measure is then computed to determine whether there is sufficient evidence to declare the election outcome correct within the pre-determined risk limit. Because the decision is made after drawing a round of ballots, the audit is termed a {\em round-by-round (R2)} audit. The special case when round size is one---that is, stopping decisions are made after each ballot draw---is a {\em ballot-by-ballot (B2)} audit. 

The \BRAVO audit is designed for use as a B2 audit: it requires the smallest expected number of ballots when the true tally of the underlying election is as announced, and stopping decisions are made after each ballot draw. In practice, election officials draw many ballots at once, and the \BRAVO stopping rule needs to be modified for use in an R2 audit that is not B2. There are two obvious approaches. The B2 stopping condition can be applied once at the end of each round: End-of Round (EoR) \BRAVO.  Alternatively, the order of ballots in the sample can be tracked by election officials and the B2 \BRAVO stopping condition can be applied retroactively after each ballot drawn: Selection-Ordered (SO) \BRAVO. SO \BRAVO requires fewer ballots on average than EoR \BRAVO but requires the work of tracking the order of ballots rather than just their tally.

\Minerva was designed for R2 audits and applies its stopping rule once for each round. Thus it does not require the tracking of ballots that SO \BRAVO does. Zag{\'o}rski {\em et al.} \cite{usenix_minerva} prove that \Minerva is a risk-limiting audit and requires fewer ballots to be sampled than EoR \BRAVO when an audit is performed in rounds and the round schedule is pre-determined (before any ballots are drawn). They also present first-round simulations which show that \Minerva draws fewer ballots than SO \BRAVO in the first round for large first rounds when the (true) underlying election is as announced. 

There are no results, either theoretical or based on simulations, regarding the number of ballots drawn over multiple rounds in a \Minerva audit with a pre-determined schedule. Because \BRAVO does not need to work on a pre-determined round schedule, it can optimize the size of the next round based on the sample drawn so far. Thus an open question is whether the constraint of a predetermined round schedule limits the efficacy of \Minerva in future rounds, and there is no literature comparing the number of ballots drawn by \Minerva and SO \BRAVO over multiple rounds. Note that the Average Sample Number (ASN) computations for \BRAVO \cite{bravo} apply only for B2 audits and are especially misleading as estimates of the number of ballots drawn over multiple rounds when first round sizes are large. 

Both \BRAVO and \Minerva have been integrated into election audit software {\em Arlo} \cite{arlo}, and, as such, available for use in real election audits. Both have been used in real election audits. For this reason, it is very important to understand their properties over multiple rounds. 

\subsection{Our Results}

\subsection{Organization} Section \ref{sec:related} describes related work and \ref{sec:software} our open source software. The experiments we performed are described in section \ref{sec:expts} and sections \ref{sec:results1} and \ref{sec:results2} present our results. Section \ref{sec:conc} has our conclusions. 



\section{Related work}
\label{sec:related}
%Related Work
Computations of first-round size for a 0.9 stopping probability when the election is as announced have been computed for a wide range of margins and shown to be smaller than those for both EoR and SO \BRAVO. First round simulations of \Minerva \cite{arxiv_athena} demonstrate that its first-round properties---regarding the probabilities of stopping when the underlying election is tied and when it is as announced---are as predicted for first round sizes with stopping probability 0.9. 

\section{Definitions}
\label{sec:defs}
\input{sections/03-defs}

\section{Software}
\label{sec:software}
%Software
In this section we describe the software implementing ballot polling audits and the simulator, termed the R2B2 library. All the software is released as open source under the MIT License.
\subsection{R2B2 Library}

The R2B2 Python library provides a framework for the exploration of round-by-round
and ballot-by-ballot Risk-Limiting Audits (RLAs). The goal in designing R2B2 is two fold:
\begin{enumerate}
    \item Provide an elegant Python library which can be easily imported and used
    in any other code base.
    \item Provide an interactive set of tools which can be utilized `out-of-the-box'
    for experimenting with and learning about risk-limiting audits.
\end{enumerate}
The following describes the design, usage and possible next steps for the R2B2 library.

\subsection{Design}

The high-level design of R2B2 is an object-oriented view of election audits.
The three main object classes, \code{Election}, \code{Contest}, and \code{Audit},
-serve to group data into logically independent structures.

\subsubsection{Election}
The \code{Election} contains the information that comprises an entire election,
most importantly, the total number of ballots cast in the election and the list
of \code{Contest}s from the election. At the moment \code{Election} does not offer
functionality beyond grouping \code{Contest}s; potential functionality discussed
in 1.3 Future Work.

\subsubsection{Contest}
The \code{Contest} contains the information related to a single
contest such as the ballots cast in that contest, the candidates, the type of contest,
and the reported tally. Providing a structure to hold this information independent of
any particular audit is especially useful for exploratory work.

\subsubsection{Audit}


The \code{Audit} contains information related to a single contests audit parameters
such as the risk limit, sampling method, and \code{Contest} to audit. It is important
to note the \code{Audit} is an Abstract Base Class upon which specific RLAs are built.
It only contains the parameters and attributes common to all RLAs and provides a set
of methods that can be called by any audit implementation. The functionality of
\code{Audit} can be divided into two basic groups: \textit{interactive}
and \textit{bulk}.\\
\\
The interactive implementation allows users to execute an audit step-by-step as it
might progress during a live election audit through the following:

\begin{itemize}
    \item The \code{run()} method begins an interactive audit executing where users
    are prompted for round sizes and the counts of winner ballots found in the sample
    and in return are given information about the current state of the audit and whether
    the stopping condition(s) have been met.
    \item Two distributions representing the null and alternative hypotheses are maintained
    and allow for computation of the audits per-round risk and stopping probability
    schedules.
    \item Before each round, the audit will recommend possible next round sizes given
    different criteria, such as a set of desired stopping probabilities.
\end{itemize}
The bulk implementations allows users to generate a larger set of data from an audit
such as:

\begin{itemize}
    \item A set of stopping conditions given a set of round sizes.
    \item A set of risk levels given a set of round size and winner ballots pairs.
    \item A list of all stopping conditions from the minimum to the maximum round size.
\end{itemize}

\subsection{Usage}

R2B2 makes understanding and exploring election audits simple for the user with no
Python knowledge while simultaneously providing a comprehensive set of tools for
the experienced Python developer.

\subsubsection{Python Library}

Using R2B2 is as simple as using any other Python library: simply import the library
and all of the functionality is at your finger tips. Not only does this allow users
to write their own Python scripts for exploring RLAs, it also allows R2B2 to be plugged
in to any other Python library. See the following Jupyter Notebooks for information on
the usage of R2B2:

\begin{itemize}
    \item \href{https://github.com/gwexploratoryaudits/r2b2/blob/notebooks/notebook/R2B2%20Basics.ipynb}{Basic Usage}
    \item \href{https://github.com/gwexploratoryaudits/r2b2/blob/notebooks/notebook/Generating%20Graphs.ipynb}{Generating Graphs}
\end{itemize}


\subsubsection{Stand-Alone Tool}

R2B2 provides a significant amount of functionality `out-of-the-box' for educational
or exploratory use. For those who wish learn about RLAs without having to write any
code themselves, R2B2 provides a command line tool for both interactive auditing and
generating audit results and statistics for larger data sets.

\subsection{Simulations to Support Theoretical Audit Properties}

The outcomes of RLAs depend on random chance; some random samples
support the alternative hypothesis more than expected, resulting in 
quick low-risk conclusions, while other samples require subsequent
rounds in order to confirm the announced results.
We can simulate random samples for various underlying ballot 
distributions by computing pseudorandom samples. 
By applying an audit's stopping condition to thousands of such
simulated samples, the average behavior of the simulated
audits will tend towards the true behavior of the audit.
In this way, we can examine whether theoretical claims about an audit are
actually correct.

\subsection{Software for Simulations}
Our open source audit software library r2b2 [link] has implementations of several ballot polling risk-limiting audits as well as a simulator, 
all written in Python.
For each of these audits, the software can evaluate the stopping condition for a given sample and can give estimates
of the minimum round size to achieve a desired stopping probability. 
For a given audit and random seed, the simulator draws random samples using the pseudorandom number generator, [need to check].
Ballots can be sampled from any distribution of the users choosing. 
It is often useful to consider the distribution of ballots corresponding to a tie and the 
distribution of ballots corresponding with the announced results; these are the distributions represented
by the null and alternative hypotheses.
After drawing a sample, the simulator then evaluates the given audit's stopping condition for this simulated sample.
If the audit stops, the simulation stops, and if the audit continues, the simulation draws another round. 
The abstract simulator class does not prescribe any one method for choosing round sizes. 
We implement several classes to support various round size choices: 
round sizes from an estimate to achieve a desired probability of stopping, 
predetermined round sizes, and random round sizes. 



\section{Simulation Results}
\label{sec:results}
%Results
For this paper, we simulated audits for 
the 2020 Presidential election
in all US states whose margin was at least $5\%$.
Round sizes increase roughly proportional to the inverse
square of the margin, so 
smaller margins are computationally much more expensive to simulate.
For each of these states, we simulated 
$10,000=10^4$ audits with the announced
underyling ballot distribution
and an additional $10,000=10^4$ audits with a tie
as the underlying ballot distribution.
These are reasonable choices for inital simulation experiments
because most audits frame the stopping decision as a binary
hypothesis test where the null hypothesis assumes an underyling tie
and the alternative hypothesis assumes the announced distribution.
A standard first round size in ballot-polling audits
has been one which achieves a $90\%$ probability
of stopping, and we chose round sizes to reflect this standard
in our simulations.
We ran our simulations for up to five rounds.
With a $90\%$ stopping probability, 
an audit only has a $a$ probablity of not stopping
by the end of the fifth round, assuming the outcome was correctly
announced.

\subsection{End-of-Round \BRAVO}
For the EoR \BRAVO simulations, our software estimated and used for each round
the minimum round size that would achieve a $90\%$ probability of stopping.
For the simulations with the announced outcome as the underlying
ballot distribution, for each round $j$ we computed the proportion of audits 
that stopped in round $j$ among those that had not stopped before round $j$.

\begin{figure}[H]
\includegraphics[width=\textwidth]{eor_bravo_90perc_10^4_corrected/sprob_first_three.png}\caption{
For the first three rounds of the EoR \BRAVO simulations, this plot shows the proportion of audits that stopped in the $j$th round
to all audits which had not yet stopped before the $j$th round.}
\label{fig:eor_bravo_sprob}
\end{figure}

In Figure \ref{fig:eor_bravo_sprob}, we display proportions for only the first three rounds
since very few audits, $(.1)^{j-1}\cdot(10^4)$ on average, 
make it to the $jth$ round.
As  a result, the proportions are based on an exponentially smaller dataset in each round.
The proportions shown in Figure \ref{fig:eor_bravo_sprob} give an estimate
of the true probability of an EOR \BRAVO audit stopping in the $j$th round,
given that it has not already stopped in a previous round. 
In \ref{fig:eor_bravo_sprob}, we see that, especially in earlier rounds for which 
the values are more representative of true audit behavior, 
our predictions are accurate.
In particular, the average across all margins is just above $.9=90\%$ for
all three rounds.

The risk of this audit, across all $5$ rounds, is an important metric since it determines whether an audit is risk-limiting.
Therefore, we now consider the proportion of audits that stopped with an underlying tie.
This proportion, for a risk-limiting audit, should approach a value less than the risk limit, $0.1$, as more audits are performed.

\begin{figure}[H]
\includegraphics[width=\textwidth]{eor_bravo_90perc_10^4_corrected/total_risk.png}
\caption{For each state margin, this plot shows
the proportion of EOR \BRAVO audits with an underlying
tie that stopped.}
\label{fig:eor_bravo_risk}
\end{figure}

Figure~\ref{fig:eor_bravo_risk} makes it clear that the risk of EOR \BRAVO is roughly
an order of magnitude less than the risk limit. 
That is, on average, roughly $10$ times more audits could have stopped 
with the audit still meeting the risk limit.
These simulations support the claim made in [\Minerva paper]
that EOR \BRAVO is unnecessarily conservative and thus requires
more ballots on average.

\subsection{Selection-Ordered \BRAVO}
% next_sample_size code is slow... tie simulations running still!
For the SO \BRAVO simulations, our software estimated and used for each round
the minimum round size that would achieve a $90\%$ probability of stopping.
For the simulations with the announced outcome as the underlying
ballot distribution, for each round $j$ we computed the proportion of audits 
that stopped in round $j$ among those that had not stopped before round $j$.

\begin{figure}[H]
\includegraphics[width=\textwidth]{so_bravo_90perc_10^4/sprob_first_three.png}\caption{
For the first three rounds of the SO \BRAVO simulations, this plot shows the proportion of audits that stopped in the $j$th round
to all audits which had not yet stopped before the $j$th round.}
\label{fig:so_bravo_sprob}
\end{figure}

In Figure \ref{fig:so_bravo_sprob}, we again display proportions for only the first three rounds.
The proportions shown in Figure \ref{fig:so_bravo_sprob}, like those in \ref{fig:eor_bravo_sprob}, give an estimate
of the true probability of an SO \BRAVO audit stopping in the $j$th round,
given that it has not already stopped in a previous round. 
In \ref{fig:so_bravo_sprob}, we see that our round size predictions are relatively accurate,
all three rounds being near $90\%$.

Again, we will next consider the proportion of audits that stopped with an underlying tie over all five rounds.
This proportion, for a risk-limiting audit, should approach a value less than the risk limit, 
$0.1$, as more audits are performed.

\begin{figure}[H]
\includegraphics[width=\textwidth]{so_bravo_90perc_10^4/total_risk.png}
\caption{For each state margin, this plot shows
the proportion of SO \BRAVO audits with an underlying
tie that stopped in any of the five rounds.}
\label{fig:so_bravo_risk}
\end{figure}

In Figure~\ref{fig:so_bravo_risk} we show only the results for the $13$
states whose simulations with an underlying tie have finished running.
To estimate the minimum next round size that achieves a desired stopping probability,
the SO \BRAVO software generates the probability distribution ballot by ballot since
the stopping condition needs to be evaluated for each individual ballot drawn.
This means, when an underyling tie causes audits to quickly move on to larger rounds, 
the simulations are computationally expensive. 
Regardless, the claim that these simulations support, that SO \BRAVO is a Risk-Limiting Audit,
is a proven and well known result.
Notice in Figure~\ref{fig:so_bravo_risk},
that the risk of SO \BRAVO is much
nearer the risk limit than that of EoR \BRAVO. 

%TODO SO and EoR macros

\subsection{\Minerva Simulations}
For \Minerva, it has not been shown that round sizes can be chosen
during the audit. That is, an adversary with knowledge of the history
of the audit may be able to choose round sizes which cause the 
risk of the audit to exceed the risk limit.
For this reason, we have to choose the round sizes of a \Minerva 
audit a priori.
For this paper, we consider two choices of round sizes.
For both, we estimate and then use the minimum first round size 
which achieves
a $90\%$ probability of stopping.
Then, for subsequent rounds, we either (i) 
draw the same number of ballots in each round or (ii)
multiply the previous round size by a factor of $1.5$ and 
sample this many new ballots.
We consider the case of drawing samples of the same size
because it may reflect a practical way to continue an
audit; if election officials have selected some first round size within
reasonable logistical bounds, drawing the same number of 
ballots in subsequent rounds may be practical.
We also consider round sizes with samples increasing by a multiple
of $1.5$ because this multiple gives a very rough approximation of 
round sizes with a $90\%$ probability of stopping.

\subsubsection{Round Sizes with Multiple of $1.0$}

As with the preceding simulations, we ran $10,000=10^4$ trials
per state for both the underlying tie and underlying reported
outcome.

\begin{figure}[H]
\includegraphics[width=\textwidth]{minerva_multiround_1x_10^4/total_risk.png}
\caption{This plot shows, for each state's margin, the proportion of audits that stopped of
all $10^4$ audits with an underlying tie in the simulations with a round size multiple of $1.0$.}
\label{fig:minerva1_risk}
\end{figure}

Notice in Figure \ref{fig:minerva1_risk} that all simulations for a tie had a proportion of audits that stopped less than $.1$, the risk limit, supporting
the claim that \Minerva is a risk limiting audit. 
Unlike EOR \BRAVO, the experimental risks here are much closer to the risk limit,
showing that \Minerva stops on average with a less conservative risk; \Minerva is sharper.

\begin{figure}[H]
\includegraphics[width=\textwidth]{minerva_multiround_1x_10^4/sprobs_first_three.png}
\caption{This plot shows, for each state's margin, the proportion of audits that stopped of
all $10^4$ audits with the announced results as the underlying distribution
in the \Minerva simulations with a round size multiple of $1.0$.}
\label{fig:minerva1_sprob}
\end{figure}

Figure~\ref{fig:minerva1_sprob} shows that the round size estimate for the first round achieved the desired
stopping probability of $90\%$. For subsequent rounds, the multiplier of $1.5$ did not consistently achieve $90\%$ stopping
probability, but it was within  roughly $10\%$. 

\subsubsection{Round Sizes with Multiple of $1.5$}

%For the Minerva simulations with a round size multiple of $1.5$,
%we increased the number of simulations to $10^6$ per state for 
%both an underlying tie and underlying announced outcome. 

%\includegraphics[width=\textwidth]{minerva_multiround_1p5x_10^6/total_risk.png}

\begin{figure}[H]
\includegraphics[width=\textwidth]{minerva_multiround_1p5x_10^4/total_risk.png}
\caption{This plot shows, for each state's margin, the proportion of audits that stopped of
all $10^4$ audits with an underlying tie in the simulations with a round size multiple of $1.5$.}
\label{fig:minerva1p5_risk}
\end{figure}

Figure~\ref{fig:minerva1p5_risk} shows, for each state's margin, the proportion of audits that stopped of
all $10^4$ audits with an underlying tie in the simulations with a round size multiple of $1.5$.

\begin{figure}[H]
\includegraphics[width=\textwidth]{minerva_multiround_1p5x_10^4/sprobs_first_three.png}
\caption{This plot shows, for each state's margin, the proportion of audits that stopped in each round
with the announced results as the underlying distribution
in the \Minerva simulations with a round size multiple of $1.5$.}
\label{fig:minerva1p5_sprob}
\end{figure}

Figure~\ref{fig:minerva1p5_sprob} shows that the round size estimate for the first round achieved the desired
stopping probability of $90\%$. For subsequent rounds, the multiplier of $1.5$ did not consistently achieve $90\%$ stopping
probability, but it was within  roughly $10\%$. 
% we could suggest that predicting accurate multiples is possible (same curve rather than line)





\section{Conclusions and Future Work}
\label{sec:conc}
\input{sections/06-conc}


%
% ---- Bibliography ----
%
% BibTeX users should specify bibliography style 'splncs04'.
% References will then be sorted and formatted in the correct style.
%
\bibliographystyle{splncs04}
\bibliography{audits}
%
\end{document}

\begin{thebibliography}{8}
\bibitem{ref_article1}
Author, F.: Article title. Journal \textbf{2}(5), 99--110 (2016)

\bibitem{ref_lncs1}
Author, F., Author, S.: Title of a proceedings paper. In: Editor,
F., Editor, S. (eds.) CONFERENCE 2016, LNCS, vol. 9999, pp. 1--13.
Springer, Heidelberg (2016). \doi{10.10007/1234567890}

\bibitem{ref_book1}
Author, F., Author, S., Author, T.: Book title. 2nd edn. Publisher,
Location (1999)

\bibitem{ref_proc1}
Author, A.-B.: Contribution title. In: 9th International Proceedings
on Proceedings, pp. 1--2. Publisher, Location (2010)

\bibitem{ref_url1}
LNCS Homepage, \url{http://www.springer.com/lncs}. Last accessed 4
Oct 2017
\end{thebibliography}

