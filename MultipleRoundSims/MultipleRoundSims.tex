\documentclass{article}
\usepackage[utf8]{inputenc}

\title{Multiple Round Ballot Polling Risk-Limiting Audit Simulations}
\author{ }
\date{ }

\begin{document}

\maketitle

\begin{abstract}
    Risk-Limiting Audits (RLAs) guarantee with known probability 
    that if the outcome of an 
    election is incorrectly announced, it will be detected, 
    and a full hand recount will be performed. 
    In Ballot-polling RLAs, samples of ballots are drawn and tallied.
    In this paper we present an audit simulation framework
    that can be used to confirm theoretical results of audits.
    We present such experimental results for multiple round 
    ballot-polling RLAs.
    BRAVO [citation] has long been the standard ballot-polling RLA,
    while Minerva was recently introduced with the claim
    that fewer ballots on average are necessary to conclude 
    an audit.
    [Minerva paper] presented experimental
    results for one round audits, and here
    we present results
    from simulations of multiple round audits using Minerva 
    as well as BRAVO 
    (both Selection-Ordered 
    BRAVO and End-of-Round BRAVO).
    The simulation results agree within reasonable error with
    the mathematical properties claimed in [Minerva paper].
    On average, BRAVO audits are unnecessarily conservative 
    while Minerva audits stop with fewer ballots. We also
    present details on software implementing Minerva and
    the simulations themselves.
\end{abstract}

\section{Introduction: B2 and R2 Ballot Polling RLAs}
    In ballot-polling RLAs, samples of ballots are drawn and tallied
    in rounds
    after each of which a statistical measure determines whether to
    continue. 
 
\section{Simulations to Confirm Theoretical Results}
The outcomes of RLAs depend on random chance; some random samples
support the alternative hypothesis more than expected, resulting in 
quick low-risk conclusions, while other unlucky samples require subsequent
rounds to confirm the announced result.
We can simulate random samples for various true underlying ballot 
distributions by computing pseudorandom samples. 
By applying an audit's stopping condition to thousands of such
simulated samples, the stopping behavior will tend towards the true
frequencies for the audit. 
In this way, we can examine whether theoretical claims for an audit are
actually correct.
For this paper, we simulated audits for all US states with at least a $5\%$ 
margin in the 2020 Presidential election. 
For each of these states, we simulated 
$10,000=10^4$ audits with the announced
underyling ballot distribution
and an additional $10,000=10^4$ audits with a tie
as the underlying ballot distribution.
\subsection{End-of-Round BRAVO}
\subsection{Selection-Ordered BRAVO}
\subsection{Minerva}

\end{document}

