%Older intro file, which is mostly a repeat of the Voting paper intro 
The literature contains numerous descriptions of vulnerabilities in deployed voting systems, and it is not possible to be certain that any system, however well-designed, will perform as expected in all instances. For this reason, 
{\em evidence-based elections} \cite{evidence-based} aim to produce trustworthy and compelling evidence of the correctness of election outcomes, enabling the detection of problems with high probability. One way to implement an evidence-based election is to use a well-curated voter-verified paper trail, compliance audits, and a rigorous tabulation audit of the election outcome, known as a risk-limiting audit (RLA) \cite{RLA}. An RLA is an audit which guarantees that the probability of concluding that an election outcome is correct, given that it is not, is below a pre-determined value known as the risk limit of the audit, independent of the true, unknown vote distribution of the underlying election. Over a dozen states in the US have seriously explored the use of RLAs---some have pilot programs, some allow RLAs to satisfy a general audit requirement and some have RLAs in statute.     

This paper concerns ballot-polling RLAs, which require a large number of ballots relative to comparison RLAs but do not rely on any special features of the election technology. Since comparison RLAs are not always feasible, ballot-polling audits remain an important resource and have been used in a number of US state pilots (California, Georgia, Indiana, Michigan, Ohio, Pennsylvania, Virginia and elsewhere). In the general ballot-polling RLA, a number of ballots are drawn and tallied in what is termed a {\em round} of ballots \cite{usenix_minerva}. A statistical measure is then computed to determine whether there is sufficient evidence to declare the election outcome correct within the pre-determined risk limit. If so, the audit stops; else another round is drawn. Election officials would typically decide to do a full manual hand count if the audit does not stop in spite of drawing a large number of ballots. Because the decision to stop or draw more ballots is made after drawing a round of ballots, the audit is termed a {\em round-by-round (R2)} audit. The special case when round size is one---that is, stopping decisions are made after each ballot draw---is a {\em ballot-by-ballot (B2)} audit.

The \BRAVO audit is designed for use as a B2 audit: it requires the smallest expected number of ballots when the true tally of the underlying election is as announced and stopping decisions are made after each ballot draw. In practice, election officials draw many ballots at once, and the \BRAVO stopping rule needs to be modified for use in an R2 audit that is not B2. There are two obvious approaches. The B2 stopping condition can be applied once at the end of each round: End-of Round (EoR) \BRAVO.  Alternatively, the order of ballots in the sample can be tracked by election officials and the B2 \BRAVO stopping condition can be applied retroactively after each ballot drawn: Selection-Ordered (SO) \BRAVO. SO \BRAVO requires fewer ballots on average than EoR \BRAVO but requires the work of tracking the order of ballots rather than just their tally. 

\Minerva was designed for R2 audits and applies its stopping rule once for each round. Thus it does not require the tracking of ballots that SO \BRAVO does. Zag{\'o}rski {\em et al.} \cite{usenix_minerva} prove that \Minerva is a risk-limiting audit and requires fewer ballots to be sampled than EoR \BRAVO when an audit is performed in rounds, the two audits have the same pre-determined (before any ballots are drawn) round schedule and the underlying election is as announced.
%\footnote{Their proof assumes that the number of relevant ballots drawn in each round is know beforehand. In MINERVA, the number of ballots drawn in each round is determined before any ballots are drawn. Because invalid ballots and ballots that are inconsequential for the contest being audited would be drawn in addition to relevant ballots, the assumption used by the proof is not true in general. (We are grateful to Philip Stark for drawing our attention to this.) However, any variation in number of relevant ballots drawn for a fixed round size would be random and not chosen by an adversary; the proof showing the risk-limiting property of MINERVA could hence be extended.}
They also present first-round simulations which show that \Minerva draws fewer ballots than SO \BRAVO in the first round for first round sizes with a large probability of stopping when the (true) underlying election is as announced. 
Broadrick {\em et al.} provide further simulations that show \Minerva requires fewer ballots over multiple rounds and for lower stopping probability.

While more efficient than \BRAVO, \Minerva requires that the round schedule is fixed in advance of the audit. This may lead to significant unnecessary work. For example, suppose the fixed round schedule for \Minerva is $10,000$ ballots per round (a reasonable number in a state-wide contest with a narrow margin), and the first round sample finds a number of ballots for the winner just short of that necessary for the audit to stop. It then may be sufficient to draw a very small second round and still stop with high probability in the second round, but the additional $10,000$ ballots must be drawn. In contrast, subsequent \BRAVO round sizes can be chosen based on preceding samples.

An open question is whether a ballot polling RLA exists with the efficiency of \Minerva and this flexibility of \BRAVO.

\subsection{Our Contributions}
We present \Providence, and provide the following:
\begin{enumerate}
\item Proof that \Providence is an RLA and resistant to an adversary who can choose subsequent round sizes with knowledge of previous samples
\item Simulations of \Providence, \Minerva, SO \BRAVO, and EoR \BRAVO which show that \Providence has similar efficiency to \Minerva, both greater than either implementation of \BRAVO
\item Results and analysis from the use of \Providence in a pilot audit in Rhode Island
\item Open source implementation of \Providence 
%including the novel metric Probability of Misleading(name?)
\end{enumerate}

%\subsection{Organization} 
%Section \ref{sec:related} describes related work. The experiments we performed are described in section \ref{sec:expts}, and sections \ref{sec:results1} and \ref{sec:results2} present our results. Section \ref{sec:conc} has our conclusions. 

