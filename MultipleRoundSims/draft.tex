% This is a draft of the coming paper "Multiple Round Ballot Polling Risk-Limiting Audit Simulations".
% 2021.
%
\documentclass[runningheads]{llncs}
%
\usepackage{float}
\usepackage{xspace}
\usepackage{graphicx}
\graphicspath{{./imgs/}}
\usepackage{comment}
\usepackage{listings,hyperref}

% \usepackage{graphicx}  --> Used for displaying a sample figure. If possible, figure files should
% be included in EPS format.
%
% If you use the hyperref package, please uncomment the following line
% to display URLs in blue roman font according to Springer's eBook style:
% \renewcommand\UrlFont{\color{blue}\rmfamily}


% nice looking audit titles
\newcommand{\Minerva}{\textsc{Minerva}\xspace}
\newcommand{\B}{{{B2}}\xspace}
\newcommand{\R}{{{R2}}\xspace}
\newcommand{\BRAVO}{\textsc{Bravo}\xspace}

\usepackage{color}
\definecolor{green}{rgb}{0.31, 0.78, 0.47}
\newcommand{\fpo}[1]{\textcolor{green}{#1}}

\def\code#1{\texttt{#1}}

\begin{document}
\lstset{language=Python}
%
\title{Simulations of Ballot Polling Risk-Limiting Audits}
%add \thanks{} within the title brackets if want to refer to supporting organization ^^
%
%\titlerunning{Abbreviated paper title}
% If the paper title is too long for the running head, you can set
% an abbreviated paper title here
%
\begin{comment}
\author{Oliver Broadrick\inst{1} \and
Sarah Morin\inst{1} \and
Grant McClearn\inst{2} \and Neal McBurnett \and Poorvi L. Vora\inst{1} \and
Filip Zag{\'o}rski\inst{3} }
%
\authorrunning{Broadrick, Morin et al.}
% First names are abbreviated in the running head.
% If there are more than two authors, 'et al.' is used.
%
\institute{Department of Computer Science, The George Washington University 
\email{obroadrick@gwmail.gwu.edu}\\
\and
Department of Computer Science, Stanford University 
\email{grantmcc@stanford.edu}\\
\and Wroclaw University of Science and Technology\\
\email{filip.zagorski@gmail.com}}
%
\end{comment}
\maketitle              % typeset the header of the contribution
%
\begin{abstract}
    In this paper we present simulation results comparing the risk, stopping probability and number of ballots required over multiple rounds of ballot-polling risk limiting audits (RLAs) \Minerva, Selection-Ordered (SO) \BRAVO and End-of-Round (EoR) \BRAVO. We also
    present details on the R2B2 open source software library and the related simulation software.    
    
    \BRAVO is the most commonly used ballot-polling RLA and requires the smallest expected number of ballots when ballots are drawn one at a time and the (true) underlying election is as announced. In real audits, election officials draw multiple ballots at a time, and \BRAVO rules may be implemented as SO \BRAVO or EoR \BRAVO, with neither being optimal for large first round sizes. \Minerva was recently proposed for use when the first-round stopping probability is large. It requires fewer first-round ballots, on average, than both SO \BRAVO and EoR \BRAVO for first round stopping probabilities of $0.9$ when the (true) underlying election is as announced. 
    
    An open question, however, is how \Minerva compares to SO \BRAVO and EoR \BRAVO over multiple rounds, and for lower stopping probabilities.  In this paper, we present results from simulations of multiple round audits. We examine both: first-round stopping probabilities of $.9$, a common choice among election officials, and $.25$, which would be more favorable to \BRAVO. Because the size of rounds in \Minerva needs to be predetermined (and independent of the audit draws), we use two pre-determined round sequences for \Minerva: (a) all rounds are of the same size (which might be a practical choice for election officials if they had planned to have resources for the first round size) and (b) each round is $1.5$ times the previous one (which is the preset value for \Minerva as integrated into election audit software Arlo). 
    
We show that the simulation results are consistent with predictions of the R2B2 open-source library for ballot polling audits. We also observe that both \BRAVO audits are more conservative than needed, while \Minerva audits stop with fewer ballots. 

\keywords{risk-limiting audit (RLA)  \and ballot polling audit \and evidence-based elections \and statistical election audit}
\end{abstract}
%
%
%
\section{Introduction}
\label{sec:intro}
%Intro
The literature contains numerous descriptions of vulnerabilities in deployed voting systems, and it is not possible to be certain that any system, however well-designed, will perform as expected in all instances. For this reason, 
{\em evidence-based elections} \cite{evidence-based} aim to produce trustworthy and compelling evidence of the correctness of election outcomes, enabling the detection of problems with high probability. One way to implement an evidence-based election is to use a well-curated voter-verified paper trail, compliance audits, and a rigorous tabulation audit of the election outcome, known as a risk-limiting audit (RLA) \cite{RLA}. An RLA is an audit which guarantees that the probability of concluding that an election outcome is correct, given that it is not, is below a pre-determined value known as the risk limit of the audit, independent of the true, unknown vote distribution of the underlying election. Over a dozen states have seriously explored the use of RLAs---some have pilot programs, some allow RLAs to satisfy a general audit requirement and some have RLAs in statute.     

This paper provides insight into the main approach to ballot polling RLAs, the \BRAVO audit \cite{bravo}, and the newer \Minerva \cite{usenix_minerva} ballot polling RLA, through the presentation of simulation results. While some properties of the two audits may be theoretically derived, for other properties theoretical results are not available. This paper examines the number of ballots drawn over multiple rounds of both audits, the related probabilities of stopping if the election is as announced, and the maximum risk of the audit. 

\subsection{Background}
Of the multiple types of RLAs, this paper focuses on ballot-polling RLAs, which require a large number of ballots but do not rely on any special features of the election technology. In the general ballot-polling RLA, a number of ballots are drawn and tallied in what is termed a {\em round} of ballots \cite{usenix_minerva}. A statistical measure is then computed to determine whether there is sufficient evidence to declare the election outcome correct within the pre-determined risk limit. Because the decision is made after drawing a round of ballots, the audit is termed a {\em round-by-round (R2)} audit. The special case when round size is one---that is, stopping decisions are made after each ballot draw---is a {\em ballot-by-ballot (B2)} audit. 

The \BRAVO audit is designed for use as a B2 audit: it requires the smallest expected number of ballots when the true tally of the underlying election is as announced, and stopping decisions are made after each ballot draw. In practice, election officials draw many ballots at once, and the \BRAVO stopping rule needs to be modified for use in an R2 audit that is not B2. There are two obvious approaches. The B2 stopping condition can be applied once at the end of each round: End-of Round (EoR) \BRAVO.  Alternatively, the order of ballots in the sample can be tracked by election officials and the B2 \BRAVO stopping condition can be applied retroactively after each ballot drawn: Selection-Ordered (SO) \BRAVO. SO \BRAVO requires fewer ballots on average than EoR \BRAVO but requires the work of tracking the order of ballots rather than just their tally.

\Minerva was designed for R2 audits and applies its stopping rule once for each round. Thus it does not require the tracking of ballots that SO \BRAVO does. Zag{\'o}rski {\em et al.} \cite{usenix_minerva} prove that \Minerva is a risk-limiting audit and requires fewer ballots to be sampled than EoR \BRAVO when an audit is performed in rounds and the round schedule is pre-determined (before any ballots are drawn). They also present first-round simulations which show that \Minerva draws fewer ballots than SO \BRAVO in the first round for large first rounds when the (true) underlying election is as announced. 

There are no results, either theoretical or based on simulations, regarding the number of ballots drawn over multiple rounds in a \Minerva audit with a pre-determined schedule. Because \BRAVO does not need to work on a pre-determined round schedule, it can optimize the size of the next round based on the sample drawn so far. Thus an open question is whether the constraint of a predetermined round schedule limits the efficacy of \Minerva in future rounds, and there is no literature comparing the number of ballots drawn by \Minerva and SO \BRAVO over multiple rounds. Note that the Average Sample Number (ASN) computations for \BRAVO \cite{bravo} apply only for B2 audits and are especially misleading as estimates of the number of ballots drawn over multiple rounds when first round sizes are large. 

Both \BRAVO and \Minerva have been integrated into election audit software {\em Arlo} \cite{arlo}, and, as such, available for use in real election audits. Both have been used in real election audits. For this reason, it is very important to understand their properties over multiple rounds. 

\subsection{Our Results}

\subsection{Organization} Section \ref{sec:related} describes related work and \ref{sec:software} our open source software. The experiments we performed are described in section \ref{sec:expts} and sections \ref{sec:results1} and \ref{sec:results2} present our results. Section \ref{sec:conc} has our conclusions. 



\section{Related work}
\label{sec:related}
%Related Work
Computations of first-round size for a 0.9 stopping probability when the election is as announced have been computed for a wide range of margins and shown to be smaller than those for both EoR and SO \BRAVO. First round simulations of \Minerva \cite{arxiv_athena} demonstrate that its first-round properties---regarding the probabilities of stopping when the underlying election is tied and when it is as announced---are as predicted for first round sizes with stopping probability 0.9. 

\section{Software}
\label{sec:software}
%Software
In this section we describe the software implementing ballot polling audits, termed the R2B2 library, and the simulator software used for this research. All the software is released as open source under the MIT License.
\subsection{R2B2 Library}

The R2B2 Python library \cite{r2b2_anon} provides a framework for the exploration of round-by-round
and ballot-by-ballot RLAs. The goal in designing R2B2 is two fold:
\begin{enumerate}
    \item Provide an elegant Python library which can be easily imported and used
    in any other code base.
    \item Provide an interactive set of tools which can be utilized `out-of-the-box'
    for experimenting with and learning about risk-limiting audits.
\end{enumerate}

\subsubsection{Design}

The high-level design of R2B2 is an object-oriented view of election audits.
The three main object classes, \code{Election}, \code{Contest}, and \code{Audit},
serve to group data into logically independent structures.

The \code{Election} contains the information that comprises an entire election,
most importantly, the total number of ballots cast in the election and the list
of \code{Contest}s from the election. At the moment \code{Election} does not offer
functionality beyond grouping \code{Contest}s.

The \code{Contest} contains the information related to a single
contest such as the ballots cast in that contest, the candidates, the type of contest,
and the reported tally. Providing a structure to hold this information independent of
any particular audit is especially useful for exploratory work.

The \code{Audit} contains information related to the audit parameters for a single contest, 
such as the risk limit, sampling method, and \code{Contest} to audit. It is important
to note the \code{Audit} is an Abstract Base Class upon which specific RLAs are built.
It only contains the parameters and attributes common to the RLAs of this paper and provides a set
of methods that can be called by any audit implementation. The functionality of
\code{Audit} can be divided into two basic groups: \textit{interactive}
and \textit{bulk}.

The interactive implementation allows users to execute an audit step-by-step as it
might progress during a live election audit through the following:

\begin{itemize}
    \item The \code{run()} method begins an interactive audit executing where users
    are prompted for round sizes and the counts of winner ballots found in the sample
    and in return are given information about the current state of the audit and whether
    the stopping condition(s) have been met.
    \item Two distributions representing the null and alternative hypotheses are maintained
    and allow for computation of the audits per-round risk and stopping probability
    schedules.
    \item Before each round, the audit will recommend possible next round sizes given
    different criteria, such as a set of desired stopping probabilities.
\end{itemize}
The bulk implementations allows users to generate a larger set of data from an audit
such as:

\begin{itemize}
    \item A set of stopping conditions given a set of round sizes.
    \item A set of risk levels given a set of round size and winner ballots pairs.
    \item A list of all stopping conditions from the minimum to the maximum round size.
\end{itemize}

\subsubsection{Usage}

R2B2 makes understanding and exploring election audits simple for the user with no
Python knowledge while simultaneously providing a comprehensive set of tools for
the experienced Python developer.

Using R2B2 is as simple as using any other Python library: simply import the library
and all of the functionality is at your finger tips. Not only does this allow users
to write their own Python scripts for exploring RLAs, it also allows R2B2 to be plugged
in to any other Python library. See the following Jupyter Notebooks for information on
the usage of R2B2: Basic Usage~\cite{jupyterBasic}, Generating Graphs~\cite{jupyterGraphs}.

% \begin{itemize} 
%     \item {Basic Usage} \cite{} % \href{https://github.com/gwexploratoryaudits/r2b2/blob/notebooks/notebook/R2B2%20Basics.ipynb}{Basic Usage}
%     \item \href{https://github.com/gwexploratoryaudits/r2b2/blob/notebooks/notebook/Generating%20Graphs.ipynb}{Generating Graphs}
% \end{itemize}

R2B2 also provides a significant amount of functionality `out-of-the-box' for educational
or exploratory use. For those who wish to learn about RLAs without having to write any
code themselves, R2B2 provides a command line tool for both interactive auditing and
generating audit results and statistics for larger data sets.

\subsection{Simulation Software}
As described above R2B2 has implementations of several ballot polling risk-limiting audits as well as a simulator, 
all written in Python.
For each of these audits, the software can compute the stopping condition for a given sample and estimates
of the next round size to achieve a desired stopping probability. 
For a given audit and random seed, the simulator draws random samples, with replacement, using a pseudorandom number generator.
given the number of votes for each candidate, and the number of invalid votes, in the underlying election (these need not be chosen to be those announced). 

When the number of candidates is more than two, the audit is carried out pairwise for each candidate pair, and votes for all other candidates are considered invalid votes. 

After drawing a simulated sample of ballots, the simulator evaluates the given audit's stopping condition for this sample.
If the audit stops, the simulation stops, and if the audit continues, the simulation draws another round. 
The abstract simulator class does not prescribe any one method for choosing round sizes. 
We implement several classes to support various round size choices: 
round sizes from an estimate to achieve a desired probability of stopping, 
predetermined round sizes, and pseudorandomly-generated round sizes. 

\subsection{Testing}

The R2B2 software is used to compute stopping conditions and next round estimates. It is intended for use by us and other researchers, and designed for this purpose. We have also independently implemented all the functionality in matlab \cite{brla_explore_anon} (the two codebases are written by different individuals) and have extensively checked the results of the two codebases. Additionally, for use in regular election audits by election officials, we have written an add-on \cite{athena_anon} to the {\em Arlo} risk-limiting audit software, the results of which have also been extensively checked against the other two codebases. 




\section{Experiments}
\label{sec:expts}
%Definitions
In this section, we motivate and describe the experiments. We consider a two candidate plurality contest, and assume that ballots are sampled with replacement, as is common in the literature. 

We first present relevant definitions.

\begin{definition}
An audit $\mathcal{A}$ takes a sample of ballots $X$ as input and gives one of the 
following decisions
\begin{enumerate}
\item
$Correct$: the audit is complete
\item
$Uncertain$: continue the audit
\end{enumerate}
\end{definition}

All of the audits discussed in this paper are modeled as binary hypothesis tests. Under the alternate hypothesis, $H_a$, the announced outcome is correct. That is, the true underlying ballot distribution is given by the announced ballot tallies. Under the null hypothesis, $H_0$, the true outcome is a tie (or the announced winner lost by one vote, and the number of ballots is large enough that the probability of drawing a ballot for the winner is that of drawing one for the winner).

The maximum risk of an audit is the probability that an audit stops, given that the underlying election is a tie. (Vora show that this is the maximum risk \cite{Bayesian-RLA}.) 
\begin{definition}[Risk]
The maximum risk $R$ of an audit $\mathcal{A}$ is
$$R(\mathcal{A})=\Pr[\mathcal{A}(X)=Correct \mid H_0]$$
\end{definition}

This leads us to the following definition of an $\alpha$-RLA.
\begin{definition}[Risk Limiting Audit ($\alpha$-RLA)]
An audit $\mathcal{A}$ is a Risk Limiting Audit with 
risk limit $\alpha$ iff 
$$R(\mathcal{A}) \le \alpha.$$
\end{definition}

We present measures of stopping probability in the $j^{th}$ round of the audit, given that the underlying election is as announced.
\begin{definition}[Stopping Probability]
The stopping probability $S_j$ of an audit $\mathcal{A}$ in round $j$ is 
$$S_j(\mathcal{A})=\Pr[\mathcal{A}(X)=Correct ~in~round~j~\land \mathcal{A}(X) \neq Correct ~previously \mid H_a]$$
\end{definition}
Experimentally, using our simulations, $S_j$ would be estimated by the fraction of audits that stop in round $j$. 

Note that $\sum _j S_j(\mathcal{A})=1$. We can also consider the cumulative stopping probability: 
\begin{definition}[Cumulative Stopping Probability]
The cumulative stopping probability $C_j$ of an audit $\mathcal{A}$ in round $j$ is 
$$C_j(\mathcal{A})= \sum_{i=1}^j S_j$$
\end{definition}
Experimentally, using our simulations, $C_j$ would be estimated by the fraction of audits that stop in or before round $j$. 

Finally, we are also interested in the probability that an audit will stop in round $j$ given that it did not stop earlier: 
\begin{definition}[Conditional Stopping Probability]
The conditional stopping probability  of an audit $\mathcal{A}$ in round $j$ is 
$$\chi_j (\mathcal{A})=\Pr[\mathcal{A}(X)=Correct ~in~round~j~\mid H_a \land \mathcal{A}(X) \neq Correct ~previously]$$
\end{definition}
Experimentally, using our simulations, $\chi_j$ would be estimated by the ratio of the audits that stop in round $j$ to those that ``entered'' round $j$, i.e. those that did not stop before round $j$. 

We simulated audits using margins from the 2020 Presidential election, limiting ourselves to pairwise margins for the two main candidates of $0.05$ or larger. Round sizes increase roughly proportional to the inverse
square of the margin, so 
smaller margins are computationally much more expensive to simulate.
For each of these states, we simulated 
$10,000=10^4$ audits assuming the underlying election was as announced ($H_a$),  
and an additional $10,000=10^4$ audits assuming the underlying election was a tie ($H_0$). 

We ran simulations for: (a) $90\%$ probability of stopping in the first round, enabling election officials to be done in the first round with very high probability if the election is as announced and (b) $25\%$ probability of stopping in the first round which is more favorable to \BRAVO. We ran our simulations for up to five rounds. For \Minerva, we chose two round schedules: one where the additional number of ballots chosen in a round is the same as in the previous round (multiplying factor of $1.0$) and the second where the multiplying factor is $1.5$ (this version is integrated into {\em Arlo}, and the multiplying factor was chosen as it roughly ensures a $90\%$ conditional stopping probability in the second round for a first round stopping probability of $90\%$). For both versions of \BRAVO, we chose a single round schedule: each round size has the same conditional stopping probability as the first one. 


\section{Stopping Probability and Risk}
\label{sec:results1}
%Results
\subsection{Stopping Probability as a Function of Round and Margin}
For both SO and EoR \BRAVO simulations, our software estimated round sizes that would give $\chi_j(\mathcal{A}) = 0.9$ and used those for the simulations. In Figure \ref{fig:eor_bravo_sprob}, we display the proportion of EoR \BRAVO audits that stopped in the $j^{th}$ round
to all audits which had not stopped before the $j^{th}$ round, for $j=1,2,3$. Though we carried out the simulations for $5$ rounds we show only the first three rounds of the simulations because very few audits, $(.1)^{j-1}\cdot(10^4)$ on average, 
make it to the $j^{th}$ round for $j \geq 4$. In Figure \ref{fig:so_bravo_sprob}, we display the same proportions for SO \BRAVO audits. 
In both cases, these proportions are estimates of the true value of $\chi_j(\mathcal{A})$ for $j=1,2,3$ as a function of margin. 
We see that, especially in earlier rounds for which 
the values are more representative of true audit behavior because fewer simulated audits have stopped, 
our round size predictions are accurate (the proportions are close to $0.9$).

\begin{figure}
\begin{centering}
\includegraphics[width=0.8\textwidth]{eor_bravo_90perc_10^4_corrected/sprob_first_three_cropped.png}\caption{
This plot shows, for each state margin, when the underlying election is as announced, the number of EoR \BRAVO audits that stopped in the $j^{th}$ round,
as a fraction of all EoR \BRAVO audits which had not yet stopped before the $j^{th}$ round for $j=1,2,3$ and $S_1=0.9$.}
\label{fig:eor_bravo_sprob}
\end{centering}
\end{figure}

\begin{figure}
\begin{centering}
\includegraphics[width=0.8\textwidth]{so_bravo_90perc_10^4/sprob_first_three.png}\caption{
This plot shows, for each state margin, when the underlying election is as announced, the number of SO \BRAVO audits that stopped in the $j^{th}$ round,
as a fraction of all SO \BRAVO audits which had not yet stopped before the $j^{th}$ round for $j=1,2,3$ and $S_1=0.9$.}
\label{fig:so_bravo_sprob}
\end{centering}
\end{figure}

Figure~\ref{fig:minerva1_sprob} and Figure~\ref{fig:minerva1p5_sprob} show the same proportions for \Minerva round multipliers of $1.0$ and $1.5$ respectively. We see that the first round size estimates were fairly accurate, with first round stopping probabilities being very close to $.9$. For subsequent rounds, the multipliers of $1.0$ achieved smaller stopping probabilities, as it was not chosen so as to obtain $\chi_j({\mathcal A}) = 0.9$. The $1.5$ multiplier is a good estimate for $j=2$, but the stopping probability for $j=3$ is slightly smaller than $0.9$. Note that we chose a simple multiplier for future rounds, but one could make more accurate round size estimates before the audit begins. 

\begin{figure}
\begin{centering}
\includegraphics[width=0.8\textwidth]{minerva_multiround_1x_10^4/sprobs_first_three.png}
\caption{This plot shows, for each state margin, when the underlying election is as announced, the number of \Minerva audits that stopped in the $j^{th}$ round,
as a fraction of all \Minerva audits which had not yet stopped before the $j^{th}$ round for $j=1,2,3$, round size multiple of $1.0$ and $S_1=0.9$.}
\label{fig:minerva1_sprob}
\end{centering}
\end{figure}

\begin{figure}
\begin{centering}
\includegraphics[width=0.8\textwidth]{minerva_multiround_1p5x_10^4/sprobs_first_three.png}
\caption{This plot shows, for each state margin, when the underlying election is as announced, the number of \Minerva audits that stopped in the $j^{th}$ round,
as a fraction of all \Minerva audits which had not yet stopped before the $j^{th}$ round for $j=1,2,3$, round size multiple of $1.5$ and $S_1=0.9$.}
\label{fig:minerva1p5_sprob}
\end{centering}
\end{figure}

Finally, we can perform a similar study for $S_1=0.25$. See Figure~\ref{fig:minerva_25} for an example, \Minerva with round mutiplier $1.5$. 

\begin{figure}
\begin{centering}
\includegraphics[width=0.8\textwidth]{minerva25percthen1p5_sprob.png}
\caption{This plot shows, for each state margin, when the underlying election is as announced, the number of \Minerva audits that stopped in the $j^{th}$ round,
as a fraction of all \Minerva audits which had not yet stopped before the $j^{th}$ round for $j=1,2,3$, round size multiple of $1.5$ and $S_1 = 0.25$.}
\label{fig:minerva_25}
\end{centering}
\end{figure}
\subsection{Maximum Risk as a Function of Round and Margin}
We also study the proportion of audits that stopped when the underlying election was a tie.
This proportion should approach a value less than the risk limit, $10\%$, as more audits are performed.

%\begin{figure}
%\includegraphics[width=0.8\textwidth]{eor_bravo_90perc_10^4_corrected/total_risk.png}
%\caption{This plot shows, for each state margin,
%the fraction of EoR \BRAVO audits that stopped in any of the $5$ rounds when the underlying election was a tie.}
%\label{fig:eor_bravo_risk}
%\end{figure}

\begin{figure}
\begin{centering}
\includegraphics[width=0.6\textwidth]{bravo_risks_same_plot.png}
\caption{This plot shows the fraction of EoR \BRAVO audits (all states with margins at least $0.05$) and SO \BRAVO audits (the 13 states for which our simulations are complete so far) that stopped in any of the $5$ rounds when the underlying election was a tie.}
\label{fig:bravo_risk}
\end{centering}
\end{figure}

We observe that the risk of EoR \BRAVO is roughly
an order of magnitude less than the risk limit. 
These results are as expected, because EoR \BRAVO is known to be too conservative \cite{usenix_minerva}.  

In Figure~\ref{fig:bravo_risk} we show only the results for the $13$
states for which our simulations with an underlying tied election have completed.
To estimate the next round size that achieves a desired stopping probability,
the SO \BRAVO software generates the probability distribution on the number of ballots in the sample ballot by ballot (see \cite{usenix_minerva}) since
the stopping condition needs to be evaluated for each individual ballot drawn.
Because the underlying tied election causes audits to move on to larger rounds, 
the simulations are computationally expensive. SO \BRAVO is proven to be a Risk-Limiting Audit,
and we observe in Figure~\ref{fig:bravo_risk},
that the risk of SO \BRAVO is much
nearer the risk limit than that of EoR \BRAVO, as expected. 

%\begin{figure}
%\includegraphics[width=0.8\textwidth]{so_bravo_90perc_10^4/total_risk.png}
%\caption{This plot shows, for each state margin,
%the fraction of SO \BRAVO audits that stopped in any of the $5$ rounds when the underlying election was a tie.}
%\label{fig:so_bravo_risk}
%\end{figure}

%TODO SO and EoR macros

Figure~\ref{fig:minerva1p5_risk} show that fewer than $0.1$ of the audits stopped when the underlying election was a tie, for round multiples $1.5$, as would be expected for an RLA with risk limit $10\%$. 
Unlike EOR \BRAVO, the experimental risks here are much closer to the risk limit,
showing that \Minerva stops on average with a less conservative risk; \Minerva is sharper. The plot for round multiple $1.0$ is very similar. 

%\begin{figure}
%\begin{centering}
%\includegraphics[width=0.6\textwidth]{minerva_multiround_1x_10^4/total_risk.png}
%\caption{This plot shows, for each state margin,
%the fraction of \Minerva audits with a round size multiple of $1.0$ that stopped in any of the $5$ rounds when the underlying election was a tie.}
%\label{fig:minerva1_risk}
%\end{centering}
%\end{figure}

%For the Minerva simulations with a round size multiple of $1.5$,
%we increased the number of simulations to $10^6$ per state for 
%both an underlying tie and underlying announced outcome. 

%\includegraphics[width=\textwidth]{minerva_multiround_1p5x_10^6/total_risk.png}

% we could suggest that predicting accurate multiples is possible (same curve rather than line)

\begin{figure}
\begin{centering}
\includegraphics[width=0.6\textwidth]{minerva_multiround_1p5x_10^4/total_risk.png}
\caption{This plot shows, for each state margin,
the fraction of \Minerva audits with a round size multiple of $1.5$ that stopped in any of the $5$ rounds when the underlying election was a tie.}
\label{fig:minerva1p5_risk}
\end{centering}
\end{figure}






\section{Number of Ballots}
\label{sec:results2}
In this section we present our data on the expected number of ballots drawn as the number of rounds increases, and on the fraction of audits that stop (an estimate of cumulative stopping probability, $C_j$) for the states of Texas, Missouri and Massachusetts, with margins of $0.057$, $0.157$ and $0.342$ respectively. Interestingly, we observe that $\Minerva$ has an advantage for a first round size with stopping probability $S_1=0.25$, but it is not as large as that for $S_1=0.9$. On all our plots we mark ASN, the Average Sample Number for B2 \BRAVO for context. Notice that, in all the plots, both instances of \Minerva show a higher probability of completion than does either \BRAVO audit when the average number of ballots drawn is ASN. 

\begin{figure}
\begin{centering}
\includegraphics[width=0.6\textwidth]{texas25.png}
\caption{This plot shows the cumulative fraction of audits that stopped as a function of average number of sampled ballots for all four audits we studied, for the state of Texas, margin $0.057$, and first round stopping probability $S_1=0.25$.}
\label{fig:texas_25}
\end{centering}
\end{figure}

We observe that the behavior of both \Minerva audits is similar, and that the plot for SO \BRAVO is to the right of (more ballots) and below (lower probability of stopping) those for \Minerva, even for a stopping probability as low as $0.25$. We observe that the plot for EoR \BRAVO shows the worst performance, which is not surprising. We observe similar behavior across margins (see Figures \ref{fig:missouri_25} and \ref{fig:mass_25}), though the improvement due to \Minerva reduces as margins get larger. We see also that the improvement due to using \Minerva is not as large as that seen for $S_1=0.9$ (see Figure \ref{fig:texas_90}). 

For $S_1=0.25$, the ratio of first round size of EoR \BRAVO to \Minerva is $1.45$, $1.37$, $1.23$ for states Texas, Missouri and Massachusetts, and margins $0.057$, $0.157$ and $0.342$ respectively. This may be compared to $2.03$, $1.99$ and $1.8$ respectively for $S_1=0.9$. Similarly, for $S_1=0.25$, the ratio of first round size of SO \BRAVO to \Minerva is $1.13$, $1.08$, $1.12$ for states Texas, Missouri and Massachusetts, and margins $0.057$, $0.157$ and $0.342$ respectively. This may be compared to $1.38$, $1.38$ and $1.30$ respectively for $S_1=0.9$. Note that the effect of such improvements on workload depends greatly on the number of ballots being sampled. For example, a $20\%$ reduction in sample size in Massachusetts might save election officials $10$ ballots, whereas the same reduction in Texas could save thousands.

\begin{figure}
\begin{centering}
\includegraphics[width=0.6\textwidth]{missouri25.png}
\caption{This plot shows the cumulative fraction of audits that stopped as a function of average number of sampled ballots for all four audits we studied, for the state of Missouri, margin $0.157$, and first round stopping probability $S_1=0.25$.}
\label{fig:missouri_25}
\end{centering}
\end{figure}

\begin{figure}
\begin{centering}
\includegraphics[width=0.6\textwidth]{massachusetts25.png}
\caption{This plot shows the cumulative fraction of audits that stopped as a function of average number of sampled ballots for all four audits we studied, for the state of Massachusetts, margin $0.342$, and first round stopping probability $S_1=0.25$.}
\label{fig:mass_25}
\end{centering}
\end{figure}

\begin{figure}
\begin{centering}
\includegraphics[width=0.6\textwidth]{texas90.png}
\caption{This plot shows the cumulative fraction of audits that stopped as a function of average number of sampled ballots for all four audits we studied, for the state of Texas, margin $0.057$, and first round stopping probability $S_1=0.9$.}
\label{fig:texas_90}
\end{centering}
\end{figure}



\section{Conclusions and Future Work}
\label{sec:conc}
%Conclusions
We describe the use of the R2B2 library and simulator to characterize the maximum risk, stopping probability and average number of ballots required across round schedules which may be specified through conditional stopping probabilities (as with the \BRAVO audits) or pre-determined round sizes (as with \Minerva). We use simulations to study the number of ballots drawn when the first round size is small (stopping probability of 0.25) and when it is large (stopping probability of 0.9) for a risk limit of $0.1$. We observe that the advantage of using \Minerva is smaller for the smaller stopping probability of the first round, as would be expected. We also observe that \Minerva does still require fewer ballots all the way through five rounds.  

A promising direction for future work would be a more detailed study of the impact of first round stopping probability and different round schedules on overall stopping probability and number of ballots for both \Minerva and \BRAVO. 


%
% ---- Bibliography ----
%
% BibTeX users should specify bibliography style 'splncs04'.
% References will then be sorted and formatted in the correct style.
%
\bibliographystyle{splncs04}
\bibliography{audits}
%
\end{document}

\begin{thebibliography}{8}
\bibitem{ref_article1}
Author, F.: Article title. Journal \textbf{2}(5), 99--110 (2016)

\bibitem{ref_lncs1}
Author, F., Author, S.: Title of a proceedings paper. In: Editor,
F., Editor, S. (eds.) CONFERENCE 2016, LNCS, vol. 9999, pp. 1--13.
Springer, Heidelberg (2016). \doi{10.10007/1234567890}

\bibitem{ref_book1}
Author, F., Author, S., Author, T.: Book title. 2nd edn. Publisher,
Location (1999)

\bibitem{ref_proc1}
Author, A.-B.: Contribution title. In: 9th International Proceedings
on Proceedings, pp. 1--2. Publisher, Location (2010)

\bibitem{ref_url1}
LNCS Homepage, \url{http://www.springer.com/lncs}. Last accessed 4
Oct 2017
\end{thebibliography}

