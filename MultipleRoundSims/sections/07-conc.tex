%Conclusions
We describe the use of the R2B2 library and simulator to characterize the maximum risk, stopping probability and average number of ballots required across round schedules which may be specified through conditional stopping probabilities (as with the \BRAVO audits) or pre-determined round sizes (as with \Minerva). We use simulations to study the number of ballots drawn when the first round size is small (stopping probability of 0.25) and when it is large (stopping probability of 0.9) for a risk limit of $0.1$. We observe that the advantage of using \Minerva is smaller for the smaller stopping probability of the first round, as would be expected. We also observe that \Minerva does still require fewer ballots all the way through five rounds.  

A promising direction for future work would be a more detailed study of the impact of first round stopping probability and different round schedules on overall stopping probability and number of ballots for both \Minerva and \BRAVO. 