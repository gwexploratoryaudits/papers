% Model

%We now summarize the model drawing largely from the notation and terminology of \cite{usenix_minerva,arxiv_athena,simulations,bravo}. The model is related work and not claimed to be original to this work. 

%An audit $\mathcal{A}$ is a function that takes as input the sample of ballots and outputs either (1) \emph{Correct: stop the audit} or (2) \emph{Undetermined: sample more ballots}. As described in the introduction, we ignore a third option of moving to a hand count, leaving that decision to election officials. This is standard in the literature, see, for example, \cite{bravo}. We provide more detail on the use of this option in section \ref{sec:related}, on related work. 

\BRAVO and \Minerva are modeled as binary hypothesis tests where the null hypothesis $H_0$ corresponds to a tied election and the alternative hypothesis $H_a$ to an election tally as announced. 
(When the number of ballots is odd, $H_0$ corresponds to the announced loser winning by one ballot.)
%Thus the null hypothesis is the outcome distinct from the announced one which is most difficult to detect; the probability of failing to detect it, given that the null hypothesis is true, is the worst case such probability and should be below the risk limit \cite{Bayesian-RLA}.

%\begin{definition}[Risk Limiting Audit ($\alpha$-RLA)]
%An audit $\mathcal{A}$ is a Risk Limiting Audit with 
%risk limit $\alpha$ iff for sample $X$
%$$
%Pr[\mathcal{A}(X) 
%= \text{Correct} \,|\, H_0]\le \alpha
%$$
%\end{definition}

The stopping conditions of \BRAVO and \Minerva rely on the following ratios.

\begin{definition}[\BRAVO Ratio] \label{def:bravo-ratio} The \BRAVO audit uses the ratio $\sigma$. Consider a sample size of $n$ ballots with $k$ for the reported winner. The proportion of ballots for the reported winner under the alternative hypothesis and null hypothesis are $p_a$ and $p_0$ respectively.
\begin{equation}
    \sigma(k, p_a, p_0, n) \triangleq \frac{p_a^{k} (1-p_a)^{n-k}}{p_0^{k} (1-p_0)^{n-k}} 
    \label{eqn:bravoratio}
\end{equation}
\end{definition}

In \BRAVO, $p_0=\frac{1}{2}$. A \BRAVO audit outputs correct if and only if
$$\sigma(k,p_a,\frac{1}{2},n)\ge \frac{1}{\alpha}.$$

%If testing the \BRAVO stopping condition after each individual ballot
% is drawn (a \B \BRAVO audit), 
If $K$ is the random variable indicating the number of ballots in the sample that contain a vote for the reported winner, it is easy to see that the ratio $\sigma$ is the likelihood ratio:
$$
\frac{Pr[K=k|H_a,n]}{Pr[K=k|H_0,n]}= \frac{\binom{n}{k}p_a^{k} (1-p_a)^{n-k}}{\binom{n}{k}(\frac{1}{2})^n} =\sigma(k, p_a, \frac{1}{2}, n)
$$

\begin{comment}
\begin{definition}[$(\alpha,p)$-\BRAVO ]\label{def:bravo}  An audit $\mathcal{A}$ is the \B~$(\alpha, p)$-\BRAVO audit iff the following stopping condition is tested at each ballot draw. If the sample $X$ is of size $n$ and has $k$ ballots for the winner,  
\begin{equation}
    \mathcal{A}(X) =  \left\{ \begin{array}{ll} \text{Correct} & ~\sigma(k, p, \frac{1}{2}, n) 
         %\triangleq \frac{p^{k} (1-p)^{n-k}}{(\frac{1}{2})^n} 
        \geq \frac{1}{\alpha}\\
        Undetermined & ~else 
    \end{array}
    \right .
    \label{eqn:bravo}
\end{equation}
\end{definition}
\end{comment}

\BRAVO is an instance of Wald's Sequential Probability Ratio Test (SPRT) \cite{wald}. Applying the general SPRT to RLAs, there would be a third output, {\em Full Manual Hand Count}, in addition to {\em Correct} and {\em Undetermined}. The test requires an additional error parameter $\beta$ (of audit error when the outcome is correct): the probability of requiring a (unnecessary) hand count when the outcome is correct. The test is:
\begin{equation}
    \mathcal{A}(X) =  \left\{ \begin{array}{ll} \emph{Correct} & ~\sigma(k, p_a, p_0, n) 
        \geq \frac{1-\beta}{\alpha}\\ \\ \\
       \emph{Hand~Count} & ~\sigma(k, p_a, p_0, n) 
        < \frac{\beta}{1-\alpha}\\ \\ \\
        \emph{Undetermined} & ~else 
    \end{array}
    \right .
    \label{eqn:wald}
\end{equation}

An example of the above use is \cite{RLA}. In the more recent literature, for example \cite{bravo}, ballot polling audits do not include this possibility (i.e. they set $\beta=0$) in order to give maximum flexibility to election officials in choosing when to proceed to a full manual count. For a given election, $\sigma$ is a function only of $k$, and increases with $k$. Any non-zero value of $\beta$ would reduce the value $\sigma$ is being compared with, and allow $\mathcal{A}$ to declare \emph{Correct} for lower values of $k$. Thus $\beta = 0$ is the most stringent of this form of audit. The choice of going to a full hand count when $\beta=0$ will only reduce the risk because it never results in an error (by definition, the true election outcome is one that would be obtained by a full manual hand count of the ballots) and prevents future errors. 
%In principle, the stopping condition could indicate a third option too: (c) the audit proceeds to a full manual hand count. This option is generally not incorporated \cite{bravo}. 
% Election officials would typically decide to perform a full manual hand count if the audit does not stop in spite of drawing a large number of ballots, typically over multiple rounds. 

A manual hand count presents numerous logistical challenges. The decision to move to one would be influenced by the certification deadline\footnote{As mentioned earlier, audits going towards statutory or legal requirements would need to be completed by the certification deadline. Pilot audits, performed after certification, usually end after a single round or fixed number of rounds, providing a measured risk (the statistical p-value) at the end of the round, and no decision regarding a full hand count needs to be made.}, the estimated number of human hours required for another round, the logistical costs of a full hand count, and the impact of any decision on citizen confidence. A better workload model would better inform the decision. %The only disadvantage of leaving the decision to election officials would be when the outcome were incorrect, and election officials would continue drawing rounds hoping that they would not need to escalate to a full hand count. 
Election officials should announce ahead of time their plans for when they would abort the audit procedure and go to a full hand count, justifying it with an eye towards completing certification requirements. This would provide a more transparent process and protect them from political pressures. 
%See section \ref{sec:sims} for a discussion of the considerations informing the choice between drawing more rounds and performing a full hand count for the \Providence audit and large round size. 

Where \BRAVO uses the ratio of the values of the probability distribution functions, \Minerva uses the ratio of their \emph{tails}. Now it becomes useful to have shorthand for a sequence of cumulative round sizes and the corresponding sequence
of cumulative winner ballot tallies.
We use:
$$\bm{n_j}\triangleq(n_1,n_2,\ldots,n_j) \quad\text{and}\quad \bm{k_j}\triangleq(k_1,k_2,\ldots,k_j)$$
Also, let $K_j$ be the random variable indicating the cumulative number of ballots in the sample after the $j$th round is drawn.

\begin{definition}[\Minerva Ratio] \label{def:minerva_ratio} The \R \Minerva audit uses the ratio $\tau_j$. We use cumulative round sizes $\bm{n_j}$, with corresponding $\bm{k_j}$ ballots for the reported winner in each round. The proportion of ballots for the reported winner under the alternative hypothesis and null hypothesis are $p_a$ and $p_0$ respectively.
         \begin{equation}
         \begin{aligned}
             \label{eqn:tau}
                 \tau_{j}(k_{j}, p_a,p_0, \bm{n_j}, \alpha )  \triangleq\\
                 \frac{Pr[K_{j} \geq k_{j} \wedge \forall_{i < j} ({\mathcal{A}}(X_i) ~\neq \text{Correct}) \mid H_a, \bm{n_j}]}{Pr[K_{j} \geq k_{j} \wedge \forall_{i < j} ({\mathcal{A}}(X_i) ~\neq \text{Correct}) \mid H_0, \bm{n_j}]}
         \end{aligned}
         \end{equation}
\end{definition}

Note that $\tau$ depends only on the most recent cumulative votes for the winner, $k_j$ and not on older values. Here, if we were to condition the ratio on the vector of cumulative winner votes $\bm{k_{j-1}}$, or, equivalently, on the most recent cumulative winner votes ${k_{j-1}}$, we would get the binomial distributions and be running a new first round each time. 

The ratios for \BRAVO and \Minerva are for two candidates. As mentioned when describing the example in section \ref{sec:intro-model}, audits of multiple-candidate contests are treated as multiple audits of the pairwise contest between the announced winner and all other candidates. 

\begin{comment}
\begin{definition}[$ (\alpha, p, \bm{n_j} ) $-\Minerva]
     \label{def:minerva}
     Given \B $(\alpha, p)$-\BRAVO and cumulative round sizes\\ $\bm{n_j}$, the corresponding \R \Minerva stopping rule for the $j^{th}$ round is:
 \begin{equation}
     \mathcal{A}(X_{j})=  \left\{ \begin{array}{ll} \text{Correct} ~~~~ \tau_{j}(k_{j}, p_a, \frac{1}{2}, \bm{n_j}, \alpha ) \geq \frac{1}{\alpha}\\
             % & \\
             % incorrect& ~~~ \sigma_n < \frac{\beta}{1-\alpha} \\
             % & \\
             Undetermined ~~else \\
         \end{array}
         \right .
         \label{eqn:minerva-test}
 \end{equation}
\end{definition}
\end{comment}


