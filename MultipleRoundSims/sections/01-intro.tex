%Intro
The literature contains numerous descriptions of vulnerabilities in deployed voting systems, and it is not possible to be certain that any system, however well-designed, will perform as expected in all instances. For this reason, 
{\em evidence-based elections} \cite{evidence-based} aim to produce trustworthy and compelling evidence of the correctness of election outcomes, enabling the detection of problems with high probability. One way to implement an evidence-based election is to use a well-curated voter-verified paper trail, compliance audits, and a rigorous tabulation audit of the election outcome, known as a risk-limiting audit (RLA) \cite{RLA}. An RLA is an audit which guarantees that the probability of concluding that an election outcome is correct, given that it is not, is below a pre-determined value known as the risk limit of the audit, independent of the true, unknown vote distribution of the underlying election. Over a dozen states have seriously explored the use of RLAs---some have pilot programs, some allow RLAs to satisfy a general audit requirement and some have RLAs in statute.     

This paper provides insight into the main approach to ballot polling RLAs, the \BRAVO audit \cite{bravo}, and the newer \Minerva \cite{usenix_minerva} ballot polling RLA, through the presentation of simulation results. While some properties of the two audits may be theoretically derived, for other properties theoretical results are not available. This paper examines the number of ballots drawn over multiple rounds of both audits, the related probabilities of stopping if the election is as announced, and the maximum risk of the audit. 

\subsection{Background}
Of the multiple types of RLAs, this paper focuses on ballot-polling RLAs, which require a large number of ballots but do not rely on any special features of the election technology. In the general ballot-polling RLA, a number of ballots are drawn and tallied in what is termed a {\em round} of ballots \cite{usenix_minerva}. A statistical measure is then computed to determine whether there is sufficient evidence to declare the election outcome correct within the pre-determined risk limit. Because the decision is made after drawing a round of ballots, the audit is termed a {\em round-by-round (R2)} audit. The special case when round size is one---that is, stopping decisions are made after each ballot draw---is a {\em ballot-by-ballot (B2)} audit. 

The \BRAVO audit is designed for use as a B2 audit: it requires the smallest expected number of ballots when the true tally of the underlying election is as announced, and stopping decisions are made after each ballot draw. In practice, election officials draw many ballots at once, and the \BRAVO stopping rule needs to be modified for use in an R2 audit that is not B2. There are two obvious approaches. The B2 stopping condition can be applied once at the end of each round: End-of Round (EoR) \BRAVO.  Alternatively, the order of ballots in the sample can be tracked by election officials and the B2 \BRAVO stopping condition can be applied retroactively after each ballot drawn: Selection-Ordered (SO) \BRAVO. SO \BRAVO requires fewer ballots on average than EoR \BRAVO but requires the work of tracking the order of ballots rather than just their tally. 

\Minerva was designed for R2 audits and applies its stopping rule once for each round. Thus it does not require the tracking of ballots that SO \BRAVO does. Zag{\'o}rski {\em et al.} \cite{usenix_minerva} prove that \Minerva is a risk-limiting audit and requires fewer ballots to be sampled than EoR \BRAVO when an audit is performed in rounds, the two audits have the same pre-determined (before any ballots are drawn) round schedule and the underlying election is as announced. They also present first-round simulations which show that \Minerva draws fewer ballots than SO \BRAVO in the first round for first round sizes with a large probability of stopping when the (true) underlying election is as announced. 

There are no results, either theoretical or based on simulations, regarding the number of ballots drawn over multiple rounds in a \Minerva audit with a pre-determined schedule. Because \BRAVO does not need to work on a pre-determined round schedule, it can optimize the size of the next round based on the sample drawn so far. Thus an open question is whether the constraint of a predetermined round schedule limits the efficacy of \Minerva in future rounds, and there is no literature comparing the number of ballots drawn by \Minerva and SO \BRAVO over multiple rounds. Note that the Average Sample Number (ASN) computations for \BRAVO \cite{bravo} apply only for B2 audits and are especially misleading as estimates of the number of ballots drawn over multiple rounds when first round sizes are large. 

Both \BRAVO and \Minerva have been integrated into election audit software {\em Arlo} \cite{arlo}, and, as such, are available for use in real election audits. Both have been used in real election audits. For this reason, it is very important to understand their properties over multiple rounds. 

\subsection{Our Contributions}
We show the following for a risk limit of $0.1$: 
\begin{enumerate}
\item Even when the first round stopping probability is as small as $0.25$, the number of ballots required for \Minerva is smaller than that required by SO \BRAVO and EoR \BRAVO. However, the improvement is considerably smaller than that when the stopping probability is $0.9$. 
\begin{itemize}
\item The number of ballots required by SO \BRAVO for a first round stopping probability of $0.9$ is about a third more than that required by \Minerva. On the other hand, for a first round stopping probability of $0.25$, it requires only about a tenth more ballots than does \Minerva.  
\item The number of ballots required by EoR \BRAVO for a first round stopping probability of $0.9$ is about twice those required by \Minerva. On the other hand, for a first round stopping probability of $0.25$, it requires only about a fourth to a half more ballots (depending on margin) than does \Minerva.  
\end{itemize}
\item For a first round stopping probability of $0.9$, when consequent \Minerva rounds are the same size (multiplying factor $1$), consequent conditional stopping probabilities are about $0.75$ and $0.74$ respectively for rounds two and three. When the multiplying factor is $1.5$, the conditional stopping probabilities for rounds two and three are $0.91$ and $0.83$ respectively. Both our simulator and the code estimating probabilities and round sizes are flexible enough to enable the study of various predetermined round schedules. 
\end{enumerate}

\subsection{Organization} Section \ref{sec:related} describes related work. The experiments we performed are described in section \ref{sec:expts} and sections \ref{sec:results1} and \ref{sec:results2} present our results. Section \ref{sec:conc} has our conclusions. 

