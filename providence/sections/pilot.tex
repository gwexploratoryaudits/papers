% pilot
The Rhode Island Board of Elections performed a pilot audit in Providence 
in February 2022. The contest audited was a single yes-or-no question in the November 202 1 election: Portsmouth's
Issue 1, "School Construction and Renovation Projects". The question had announced margin $25.67\%$ and the audit used risk-limit $10\%$.

A first round size of $140$ ballots with large probability of stopping ($95\%$) was selected
in order to give the potential for more interesting analysis afterwards; the large margin 
made a first round with large stopping probability practical.
Selection order was tracked for the sake of analysis.
As expected, the audit concluded in the first round with a \Providence risk of $4.18\%$. Table~\ref{tab:pilot-risks} shows risk measures for the drawn sample using \Minerva and \BRAVO (both EoR and SO).

\begin{table}
\begin{center}
\begin{tabular}{ |c|c|c|c|c| } 
\hline
%\diagbox[dir=NW]{First \\Round \\Size}{RLA}
ballots& \rotatebox{45}{\Providence} & \rotatebox{45}{\Minerva} & \rotatebox{45}{SO \BRAVO} & \rotatebox{45}{EoR \BRAVO} \\
\hline
140 & \bf{4.18\%} & \bf{4.18\%} & \bf{5.41\%} & 36.6\% \\
\hline
\end{tabular}
\end{center}
\caption{Risk measures for the drawn first round of $140$ ballots in the Providence, RI pilot audit. Risks in bold meet the risk-limit ($10\%$) and thus correspond to audits that would stop.}
\label{tab:pilot-risks}
\end{table}

Note that the risk measures shown in Table~\ref{tab:pilot-risk} imply that, for the sample obtained in the pilot audit, an EoR \BRAVO audit would not have been able to stop in the first round, despite the large round size. Further, if the risk limit had been $0.05$ instead of $0.10$, SO \BRAVO also would have required moving on to a second round. 

We can use simulations to better understand typical audit behavior for the margin of this pilot audit and contextualize the results we obtained in the pilot. We run $10^4$ trial audits for several stopping probabilities $p$. Each round size is chosen to give a probability of stopping $p$ assuming the announced tally and given the results of previous rounds. We use the same $10\%$ risk limit and margin of $25.57\%$. Figure~\ref{fig:pilot_sims} shows the average number of ballots sampled for each value of $p$.

Section~\ref{sec:workload} addresses more appropriate measures of workload
