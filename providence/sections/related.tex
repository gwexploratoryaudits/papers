%Related Work
In this section, we describe related work. 

\npara{\bf Ballot Polling RLA Process:} Bernhard provides a good description of the RLA and its assumptions, and also describes the process on the ground \cite{bernhard-sok}. 

Election officials typically draw ballots in large round sizes, see for example \cite{va-2022,RI-report}. Note also that, in addition to allowing users to directly enter a round size or choose the expected number of ballots drawn by \BRAVO, Arlo provides choices of stopping probabilities of $0.9$, $0.8$ and $0.7$. For the two audits we attended, election officials chose stopping probabilities of at least $0.9$. Estimates of round sizes with stopping probability 0.9 for each state in the 2020 US Presidential election may be found in \cite{usenix_minerva}. Thousands of ballots is quite a common estimate; many estimates are as large as tens and hundreds of thousands of ballots. We are not aware of any ballot polling RLA performed on ballots cast in a governmental election that drew ballots one at a time (though the stopping condition can be computed one ballot at a time, the ballots are drawn in rounds). 

\npara{\bf R2 and B2 Audits and the Classical \BRAVO audit:} A {\em round-by-round (R2)} audit is the general audit, where the decision of whether to draw more ballots or not is taken after drawing a round of ballots. A {\em ballot-by-ballot (B2)} audit is the special case of round size one---when the decision is made after each ballot is drawn. The popular \BRAVO audit requires the smallest expected number of ballots when the announced tally of the election is correct, and stopping decisions are taken a ballot at a time (that is, when it is used as a B2 audit). However, \BRAVO cannot be used as a B2 audit in the scenarios described in the previous paragraph. 

For use as an R2 audit, the \BRAVO stopping condition can be applied once at the end of each round (End-of Round (EoR)), or retroactively after each ballot drawn if ballot order is retained (Selection-Ordered (SO)). SO \BRAVO is closer to the original B2 \BRAVO, and requires fewer ballots on average than EoR \BRAVO. But it requires the additional effort of tracking the order of ballots, and should be expected to be inefficient because it does not use the information in the ballots drawn towards the end of the round. 

%The \BRAVO audit \cite{bravo} is the most popular ballot polling audit. When ballots are sampled one at a time, it is the audit with the smallest expected number of ballots drawn. 
\npara{\bf Newer Ballot Polling Audits:} The \Minerva audit \cite{usenix_minerva,arxiv_athena} does not need ballot order and relies only on sample and round tallies. It was developed for use with large first round sizes, and has been proven to be risk limiting when the round schedule for the audit is fixed before any ballots are drawn. First-round sizes for a stopping probability of $0.9$ when the announced tally is correct have been shown to be smaller than those for EoR and SO \BRAVO for a wide range of margins. 

The \ALPHA audit \cite{alpha} generalizes \BRAVO to gain efficiency in cases where the reported outcome is correct but the reported margin is erroneous.

\npara{\bf Simulations:} Ballot polling audit simulations provide a means of educating the public and election officials \cite{dice} and understanding audit properties \cite{mclaughlin_thesis,simulations_house, blom_IRV, DBLP:conf/evoteid/HuangRSTV20,bravo}. There is work measuring the amount of time taken to examine a single ballot \cite{RI-report}. 
Simple workload estimates may be obtained by using the number of ballots drawn \cite{bernoulli-ballot-polling}, a more thorough workload estimation model includes the time taken to access individual ballots\cite{bernhard-diss}. 

Zag\'{o}rski {\em et al.}  present first-round simulations demonstrating that \Minerva draws fewer ballots than SO \BRAVO in the first round for large first round sizes when the true tally is as announced. Broadrick {\em et al.} provide further simulations showing that \Minerva requires fewer ballots than EoR and SO \BRAVO over multiple rounds and for smaller stopping probability. As expected, the advantage of \Minerva decreases for smaller stopping probability (smaller round sizes) as it approaches a B2 audit, 
%round schedule (1,1,1,\ldots) 
for which \BRAVO is known to be most efficient. 

\npara{\bf Ballot and Batch Comparison Audits:} In a ballot comparison RLA \cite{RLA}, the manual interpretation of each sampled ballot is compared to the corresponding Cast Vote Record (CVR), which is the machine interpretation of the ballot. Ballot comparison RLAs require the fewest ballots of all known RLA approaches, but also require a means of identifying the CVR corresponding to a particular ballot. Not all voting systems record CVRs and their use can present privacy challenges. %A typical approach is to use a ballot serial number on both paper ballot and CVR. When voters vote in precincts, however, serial numbers on ballots can enable the correlation of ballots with voters, and ballots are typically not numbered. Additionally, some voting systems do not record a CVR. One may perform a transitive audit by rescanning unnumbered voted ballots with special scanners which produce CVRs and also print numbers on the ballots as they are scanned. This requires an investment in a sufficiently large number of such printers and in the human effort of rescanning ballots, and is not always feasible. There is recent work on {\em adaptive comparison} RLAs, which relies on scanning only the batches necessary \cite{adaptive-comparison}. In either case, the workload of a transitive ballot comparison audit, measured in hours, would not be simply a linear function the number of ballots drawn, but would include a term which would depend on the total number of ballots rescanned. 
A batch comparison RLA \cite{RI-report} samples batches of ballots (typically, a batch is a storage box of ballots) and compares the manual tally of each sampled batch with the announced tally of that batch. %Thus, while this type of audit does not need CVRs, it does need the publication of the tally of each batch. 
Batch comparison typically requires the sampling of a very large number of ballots, larger than polling audits except for small enough margins. 
%However, the process is most similar to that which election officials already use when they perform fixed-percentage post-election audits, where, for example, 2\% of the batches are manually tallied. (Note that, for an RLA, the number of batches tallied is not fixed; the risk limit is. A smaller number of batches might be sufficient, or a larger number necessary, for an audit with the required risk limit.)

In this work, we consider ballot polling RLAs only and thus compare \Providence with \BRAVO and \Minerva.
