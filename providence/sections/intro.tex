%Intro
It is well-known that electronic voting systems are vulnerable to software errors and manipulation which may be undetected. Errors and/or manipulation may not always change an election outcome, but we would want to know when they do. {\em Software independent} voting systems \cite{SI-Wack,rivest2008notion} are ones where an undetected change in the software cannot lead to an undetectable change in the election outcome. The {\em evidence-based elections} \cite{evidence-based} approach requires that election outcomes be supported by strong affirmative evidence. The evidence is generated during the election and publicly examined at the end, enabling citizens to determine whether it provides strong support of the election outcome or is not sufficiently convincing. 
%by requiring not merely the use software independent systems to produce trustworthy evidence of outcome correctness; but also the examination of the evidence so that incorrect outcomes may be detected with high probability. 
One approach to evidence-based elections is to use voter-verified paper ballots, store them securely, and perform public audits---a compliance audit to determine whether the ballots were stored securely and procedures were followed; and a rigorous tabulation audit, known as a risk-limiting audit (RLA) \cite{RLA}, to determine whether the outcome is correctly computed from the stored ballots.  
%Many US states have had pilot RLA programs. Additionally, some states allow RLAs to be used towards audit requirements, and some states require RLAs before elections can be certified. 
%We propose the \Providence audit, a new approach to a particular type of RLA---the ballot polling RLA described below---and propose a new model for the work load of an election. We show that \Providence is superior to the popular ballot polling RLA \Bravo for real elections, and describe the use of our open source implementation by the Rhode Island Board of Elections for an audit of their 2021 elections. \Providence was recently integrated as an option in Arlo \cite{arlo}, the most popular election audit software. %Our open-source implementations of \Providence and our audit planning tools are likely to be useful later this year; ballot polling audits are expected to be used as pre-certification RLAs in at least one statewide contest in both Georgia and Pennsylvania for the 2022 general elections in the US.   

As a formalized approach to the examination of evidence supporting vote tabulation, an RLA is an important part of an evidence-based election. When correctly implemented, it serves to vastly improve the trustworthiness of the election. 

Significant efforts by nationwide organizations (such as Verified Voting, Brennan Center for Justice, Common Cause and Democracy Fund), local organizations, citizen advocates and experts have led to great progress towards the use of RLAs. Nonprofit VotingWorks has developed open-source election audit software, Arlo \cite{arlo}, and provides training in its use. Six states (Colorado, Georgia, Nevada, Pennsylvania, Rhode Island, Virginia) now require RLAs; three have statutory pilot programs (Indiana, Kentucky, Texas); four allow RLAs to satisfy a more general audit requirement (California, Ohio, Oregon, Washington); and two have an administrative pilot program (Michigan, New Jersey) \cite{vv_audits}. The effort to broaden the reach of RLAs continues, as election officials appear very keen to improve the trustworthiness of the elections they administer. 

\subsection{Background on RLAs}
A tabulation audit will either end with a declaration that an election outcome is correct, or escalate to a full hand count. The integrity of an audit may be judged by how it deals with incorrect outcomes. A {\em risk-limiting audit (RLA)} guarantees a minimum probability that it will perform as it is supposed to (escalate to a full hand count) if the outcome is incorrect. Equivalently, this is also a guarantee of a maximum probability with which it will perform erroneously (declare the audit correct) when the outcome is incorrect. The {\em risk limit} of an RLA is the (guaranteed) maximum probability that an incorrect election outcome would be declared correct. Lower risk limits are better.  

There are three types of RLAs: ballot comparison RLAs, batch comparison RLAs and ballot polling RLAs. This paper focuses on ballot polling RLAs which have been used in a number of US state pilots (California, Georgia, Indiana, Michigan, Ohio, Pennsylvania, Virginia and elsewhere), real statewide audits (Georgia, Virginia) \cite{vv_audits} and  audits of smaller jurisdictions, such as Montgomery County, Ohio \cite{usenix_minerva}. Ballot and batch comparison RLAs are described in section \ref{sec:related}. 

\subsubsection{The Workflow of a Ballot Polling RLA}
\label{sec:polling}
%Ballot and batch comparison RLAs compare information gathered through manual examination of the sampled ballots with the corresponding information claimed by the voting system. For example, the information could be the vote on a particular ballot (stored by the voting system as a CVR or Cast Vote Record) or the tally of a particular box of ballots. 
A ballot polling RLA \cite{RLA} is based on manual examination of the sampled ballots, and does not require any information from the tabulating system other than the tally. 
%, on the other hand, simply tallies the sampled ballots. It requires a much larger number of ballots than a ballot comparison RLA, but tends to be more feasible because it does not require any additional functionality of the voting system; in particular it does not require the system to generate or provide CVRs. 
More detail about the storage of ballots is required, however: a complete ballot manifest (a list of ballot storage containers and the number of ballots in each) which enables the creation of a well defined list of the ballots and their locations (the fifth ballot in box number 20, for example) to enable the sampling of specific ballots from the list. 

All RLAs draw one or more ballots at a time; each such set of ballots is referred to as a {\em round}. We use notation and terminology from \cite{usenix_minerva,arxiv_athena,simulations,bravo} and also assume ballots are drawn with replacement. 

Ballot polling audits proceed as follows. 
\begin{enumerate}
\item The ballot manifest is published. 
\item A first round size \cite{usenix_minerva} is chosen. %\fpo{the sentence is hard to understand}. pv: fixed
\item Ballots on the ballot manifest are sampled uniformly at random, with replacement, using a pseudorandom number generator, which is typically seeded by a natural source of randomness like rolling dice.
\item The physical ballots are found and manually interpreted, recording the manual interpretations. 
\item The stopping condition $\mathcal{A}$, a function of the manual interpretations of the current cumulative sample of ballots $X$, is computed. It outputs: 
\begin{enumerate}
\item \emph{Correct: stop the audit} or 
\item \emph{Undetermined: sample more ballots}. 
\end{enumerate}
Election officials may choose to abort this procedure and go to a full hand count at any time and should have a plan for how to decide whether to do so; we discuss this in more detail below and in section \ref{sec:model}.     
\item If more ballots are to be drawn, the next round size is chosen, and we go back to step 2. 
\end{enumerate} 

Round sizes, including the first one, may be computed based on a desired probability of audit completion at the end of the round, and may take into consideration loose estimates of the resources required. For RLAs required by statute or legislation, the successful completion of the RLA (or a full hand count confirming the outcome) is usually necessary before certification\footnote{If an RLA were performed after certification and determined that the outcome was incorrect, there may not be a legal means of changing the outcome and this could significantly impact citizen confidence. For example, till recently, the state of Virginia required RLAs but they were to be performed after certification and could not be used to change an outcome. This was corrected through new legislation passed in April 2022.} and certification deadlines would play a large role in round size and hand count decisions. 

%In principle, the stopping condition could indicate a third option too: (c) the audit proceeds to a full manual hand count. This option is generally not incorporated \cite{bravo}. A manual hand count presents significant logistical challenges, and there is always a chance that it will have been unnecessary. 
% They would be influenced by the certification deadline, the estimated number of human hours required for another round, the logistical costs of a full hand count, and the impact of any decision on citizen confidence. 
%This value is the smallest risk limit of the RLA that would stop at the end of the round. 
%The stopping condition for \BRAVO, \Minerva and \Providence is described for a contest with two participants and formulated as a binary hypothesis test. Audits of multiple-candidate contests are treated as multiple audits of the pairwise contest between the announced winner and all other candidates. The \BRAVO and \Minerva stopping conditions are provided in section \ref{sec:model}. 

%\BRAVO is also used for what are termed {\em stratified} audits [REF], where a single contest is audited using different approaches (ballot polling, ballot comparison, batch comparison) in different counties. When a ballot polling audit is used for the entire set of ballots, the null hypothesis is a tied outcome (the hardest incorrect outcome to detect). In a stratified audit, however, when testing individual units such as counties, one may need to assume a different null. This is because, for example, if a county with an announced large number of ballots for the winner were actually tied, the overall outcome would be incorrect. The proof of \Minerva's risk-limiting property assumes a tied null, and it is not immediately obvious how the proof may be modified to allow other possibilities. Thus, while \Minerva may be used for multi-candidate contests when all counties participate in the ballot polling audit, it cannot be used for stratified audits. \BRAVO, and \Providence, on the other hand, can. 

%Using notation and terminology from , an audit $\mathcal{A}$ is a function that takes as input the current sample of ballots and outputs either (1) \emph{Correct: stop the audit} or (2) \emph{Undetermined: sample more ballots}. As mentioned earlier, we ignore a third option of a full hand count, leaving that decision to election officials, see section \ref{sec:related} for more detail. 

\subsubsection{The Audit Model}
\label{sec:intro-model}
An audit is typically defined as a binary hypothesis test. If the null, $H_0$, is defined as the incorrect outcome hardest to detect (generally a tie, see \cite{Bayesian-RLA}), we have the following definition of an RLA. 

\begin{definition}[Risk Limiting Audit ($\alpha$-RLA)]
\label{def:RLA}
An audit $\mathcal{A}$ is a Risk Limiting Audit with 
risk limit $\alpha$ iff for sample $X$
$$
Pr[\mathcal{A}(X) 
= \text{Correct} \,|\, H_0]\le \alpha
$$
\end{definition}

Definition \ref{def:RLA} is valid at the end of the RLA, and not at the end of each round. Thus it is not the case that an incorrect election will pass the audit if a sufficient number of rounds is drawn. In fact, the larger a sample, the more accurate the estimate of election correctness can be, and the more the audit will diverge from one that would stop (see the example below). It is possible to declare an incorrect outcome as correct even after drawing a large sample, but a good audit is less likely to make this error. 

%When $X$ is used to represent a sample while the RLA is ongoing, it is understood to be the cumulative sample, representing all the information obtained thus far about the underlying vote collection.  

\npara{\bf Full Hand Count:} It is worth noting here that if election officials do not have a plan for when they will move to a full hand count, and the election outcome is incorrect, at least a fraction $(1-\alpha)$ of the audits will {\em never} stop, and new rounds will continue to be drawn. See Figure \ref{fig:prov-risk} in section \ref{sec:sims} for the results of 10,000 simulations of \Providence on tied elections (risk limit of 0.1, across election margins of 0.05 and larger in the statewide contests for US President in 2020), each simulation stopping after five rounds. Observe that, for each margin studied, more than 90\% of the audits do not stop after five rounds. We present two example simulations from among the ones reported for the state of Texas to illustrate this idea. 

A total of 11, 315,056 votes were cast in the 2020 US President contest in the state of Texas \cite{MIT-Data-Lab}; candidate Trump won with 5,890,347 votes, and the second highest vote count was that of candidate Biden, who received 5,259,126 votes, for a margin of 0.0566 in the pairwise contest. A pairwise contest between two candidates treats invalid votes and those received by other candidates as irrelevant to determining which of the two won the pairwise contest. Its {\em margin} is the difference between their vote counts as a fraction of the sum of their votes. Ballot polling audits used to audit government elections (such as \BRAVO, \Minerva and \Providence) audit a multi-candidate contest by conducting multiple audits of the pairwise contests between the winner and every other candidate. 

A first round size of 2,217 corresponds to a stopping probability of 0.9 and a risk limit of $\alpha=0.1$ for \Providence (see \cite{usenix_minerva} for a description of how first round size may be computed, \Minerva and \Providence are identical in the first round). The audit stops when the inverse of the \Providence ratio (see section \ref{sec:prov}) is smaller than $\alpha$. We describe two audit simulations using these parameters. 

The first simulation assumed that the tally is correct and resulted in 1,138 votes for Trump and 1,054 for Biden (the rest were irrelevant votes). The inverse of the \Providence ratio was $0.047 < \alpha$ and the audit stopped. 

The second simulation assumed a tied election and did not stop at the end of five rounds. 
%The design of the \Providence audit (see section \ref{sec:prov}) ensures that the risk of the first round is no more than 0.9 times the risk limit, and the risk expended in the first round is roughly 0.09. 
Table \ref{tab:toy} lists the cumulative round sizes, cumulative votes for both candidates and the inverse \Providence ratio after each draw. Each round size was chosen for a conditional stopping probability of 0.9, given that the audit did not stop so far. For example, this corresponds to an approximate probability of 0.09 that the audit stops in the second round (that it does not stop in the first round and stops in the second), and so on. 
\begin{table}[h]
\centering
%\scriptsize
\begin{tabular}{|crrrc|}
\hline
\hline
Round& Cumulative& Trump & Biden & Inverse  \\
 No. & Round Size &  & & Ratio \\
\hline
\hline
1 & 2,217 & 1,111 & 1,079 & 0.256\\
2 & 5,970 & 2,940 & 2,953 & 4.251\\
3 & 16,685 & 8,281 & 8,171 & 3,150 \\
4 & 35,096 & 17,320 & 17,264 & 380,220,376 \\
5 & 76,979 & 37,943 & 37,868 & $2.5e^{+22}$ \\
\hline
\end{tabular}
\caption{A simulation of a \Providence audit of a tied contest with an announced margin of 0.0566, and $\alpha=0.1$. Each round size is computed for a conditional stopping probability of $0.9$. The stopping condition is Inverse Ratio < $\alpha$}
\label{tab:toy}
\end{table}

To avoid a scenario where election officials fruitlessly draw round after round hoping the audit will stop when the election outcome is incorrect, it might be worthwhile for election officials to determine, before the audit begins, the latest date by which either: the audit stops, or a hand count begins, so as to be completed before certification. 

\npara{\bf The Adversary in an RLA:} The goal of the adversary is to increase the true risk beyond the declared risk limit---that is, make the audit declare an incorrect outcome as correct with a chance larger than the risk limit. An audit is an RLA only if there is a proof of its risk-limiting property. Hence, at the very least, the adversary would need to invalidate an assumption of the proof to obtain:  
$$
Pr[\mathcal{A}(X) 
= \text{Correct} \,|\, H_0] > \alpha
$$

\subsection{Related Work}
\label{sec:related}
%Related Work
The \BRAVO audit \cite{bravo} is a well-known ballot polling audit which has been used in numerous pilot and real audits. When used to audit a two-candidate election, it is an instance of Wald's sequential probability ratio test (SPRT) \cite{wald}, and inherits the SPRT property of being the most efficient test (requiring the smallest expected number of ballots) if the election is as announced. The model for \BRAVO and the SPRT is, however, that of a sequential audit: a sample of size one is drawn, and a decision of whether to stop the audit or not is taken. Real election audits invest in drawing large numbers of ballot, called rounds, before making stopping decisions because sequentially sampling individual ballots has significant overhead (unsealing storage boxes and searching for individual ballots). It is possible to apply \BRAVO to the sequence of ballots in a round if the sequential order is retained. This is not, however, the most efficient possible use of the drawn sample because information in consequent ballots is ignored when applying \BRAVO to ballots that were drawn earlier in the sample. 

We do know a great deal about the properties of \BRAVO. The risk limiting property of \BRAVO follows from the similar property of the SPRT. Stopping probabilities for \BRAVO may be estimated as implemented in \cite{arlo}; this method is due to Mark Lindeman and uses quadratic approximations. A later method for stopping probability estimates presented by Zag{\'o}rski {\em et al.}\cite{usenix_minerva,arxiv_athena} uses a similar technique for narrow margins and a separate algorithm for wider margins, the results of which match simulation results reported by Lindeman {\em et al.} \cite[Table 1]{bravo}.  

The \Minerva audit \cite{usenix_minerva,arxiv_athena} was developed for large first round sizes which enable election officials to be done in one round with large probability. It uses information from the entire sample, and has been proven to be risk limiting when the round schedule for the audit is determined before the audit begins. That is, information about the actual ballots drawn in the first round cannot inform future round sizes. First-round sizes for a $0.9$ stopping probability when the election is as announced have been computed for a wide range of margins and are smaller than those for EoR and SO \BRAVO. First round simulations of \Minerva \cite{arxiv_athena} demonstrate that its first-round properties---regarding the probabilities of stopping when the underlying election is tied and when it is as announced---are as predicted for first round sizes with stopping probability $0.9$. 

Ballot polling audit simulations have been used to familiarize election officials and the public with the approach \cite{dice}. McLaughlin and Stark \cite{mclaughlin_thesis,simulations_house} compare the workload for the Canvass Audits by Sampling and Testing (CAST) and Kaplan-Markov (KM) audits using simulations. Blom {\em et al.} demonstrate the efficiency of their ballot polling approach to audit instant runoff voting (IRV) using simulations \cite{blom_IRV}. Huang {\em et al.} present a framework generalizing a number of ballot polling audits and compare their performance (round sizes and stopping probabilities) using simulations \cite{DBLP:conf/evoteid/HuangRSTV20}. This work was prior to the development of \Minerva, and focuses on the comparison between Bayesian audits \cite{bayesian-audits} and \BRAVO, essentially studying the impact of the prior of the Bayesian RLA. Some workload measurements have been made\cite{RI-report}. While total ballots sampled can give naive workload estimates\cite{bernoulli-ballot-polling}, Bernhard presents a more complex workload estimation model\cite{bernhard-diss}. 

\subsection{Model}

\begin{definition}[Risk Limiting Audit ($\alpha$-RLA)]
An audit $\mathcal{A}$ is a Risk Limiting Audit with 
risk limit $\alpha$ iff for sample $X$
$$
Pr[\mathcal{A}(X) 
= \text{Correct} \,|\, H_0]\le \alpha
$$
\end{definition}

\begin{definition}[\BRAVO Ratio] \label{def:bravo-ratio} The \BRAVO audit uses the ratio $\sigma$. Consider a sample size of $n$ ballots with $k$ for the reported winner. The proportion of ballots for the reported winner under the alternative hypothesis and null hypothesis are $p_a$ and $p_0$ respectively.
\begin{equation}
    \sigma(k, p_a, p_0, n) \triangleq \frac{p_a^{k} (1-p_a)^{n-k}}{p_0^{k} (1-p_0)^{n-k}} 
    \label{eqn:bravoratio}
\end{equation}
\end{definition}

In \BRAVO, $p_0=\frac{1}{2}$. If testing the \BRAVO stopping condition after each individual ballot
is drawn (a \B \BRAVO audit), $\sigma$ is equivalent to the 
likelihood ratio:
$$
\frac{Pr[K=k|H_a,n]}{Pr[K=k|H_0,n]}= \frac{\binom{n}{k}p_a^{k} (1-p_a)^{n-k}}{\binom{n}{k}(\frac{1}{2})^n} =\sigma(k, p_a, \frac{1}{2}, n)
$$

\begin{definition}[$(\alpha,p)$-\BRAVO ]\label{def:bravo}  An audit $\mathcal{A}$ is the \B~$(\alpha, p)$-\BRAVO audit iff the following stopping condition is tested at each ballot draw. If the sample $X$ is of size $n$ and has $k$ ballots for the winner,  
\begin{equation}
    \mathcal{A}(X) =  \left\{ \begin{array}{ll} \text{Correct} & ~\sigma(k, p, \frac{1}{2}, n) 
         %\triangleq \frac{p^{k} (1-p)^{n-k}}{(\frac{1}{2})^n} 
        \geq \frac{1}{\alpha}\\
        Undetermined & ~else 
    \end{array}
    \right .
    \label{eqn:bravo}
\end{equation}
\end{definition}

It becomes useful to have shorthand for a sequence of round sizes and a sequence
of winner ballot tallies.
We use:
$$\bm{k_j}\triangleq(k_1,k_2,\ldots,k_j)$$
$$\bm{n_j}\triangleq(n_1,n_2,\ldots,n_j)$$

\begin{definition}[\Minerva Ratio] \label{def:minerva_ratio} The \R \Minerva audit uses the ratio $\tau_j$. We use cumulative round sizes $\bm{n_j}$, with corresponding $\bm{k_j}$ ballots for the reported winner in reach round. The proportion of ballots for the reported winner under the alternative hypothesis and null hypothesis are $p_a$ and $p_0$ respectively.
         \begin{equation}
             \label{eqn:tau}
                 \tau_{j}(k_{j}, p_a,p_0, \bm{n_j}, \alpha )  \triangleq
                 \frac{Pr[K_{j} \geq k_{j} \wedge \forall_{i < j} ({\mathcal{A}}(X_i) ~\neq \text{Correct}) \mid H_a, \bm{n_j}]}{Pr[K_{j} \geq k_{j} \wedge \forall_{i < j} ({\mathcal{A}}(X_i) ~\neq \text{Correct}) \mid H_0, \bm{n_j}]}
         \end{equation}
\end{definition}

\begin{definition}[$ (\alpha, p, \bm{n_j} ) $-\Minerva]
     \label{def:minerva}
     Given \B $(\alpha, p)$-\BRAVO and cumulative round sizes\\ $\bm{n_j}$, the corresponding \R \Minerva stopping rule for the $j^{th}$ round is:
 \begin{equation}
     \mathcal{A}(X_{j})=  \left\{ \begin{array}{ll} \text{Correct} ~~~~ \tau_{j}(k_{j}, p_a, \frac{1}{2}, \bm{n_j}, \alpha ) \geq \frac{1}{\alpha}\\
             % & \\
             % incorrect& ~~~ \sigma_n < \frac{\beta}{1-\alpha} \\
             % & \\
             Undetermined ~~else \\
         \end{array}
         \right .
         \label{eqn:minerva-test}
 \end{equation}
\end{definition}

%\subsection{Adversarial Model for RLAs}
%\label{sec:adv}
%Detailed descriptions of best practices for post-election audits may be found in \cite{best-practices,why-and-how}. For our purposes, we will assume that the best practices are followed: the paper trail consists of hand-marked paper ballots and is secured; a public compliance audit is carried out before the RLA to ensure that the processes for securing the paper trail were followed; voter authentication and registration processes are verified, and only legitimate voters cast no more than a single vote each; the risk-limiting audit is public. We will further assume that all software used in the RLA is open source and well-defined, so its output may be reproduced and thus verified by an observer wishing to do so with their own software. 

%Referring to the ballot polling audit steps described in section \ref{sec:polling} above, we further assume that the ballot manifest is verified by the compliance audit; a secure PRNG is used; the seed is generated uniformly at random in a public process; the process of locating ballots is publicly observable and the located ballots can be viewed by the public. Because the PRNG is well-defined, as is the stopping condition, we may assume that the stopping condition is correctly computed, because it can be checked by the public through knowledge of the seed and the drawn ballots. Thus the only variable is round size. 

%We define a {\em weak adversary} as one who can choose the first and subsequent round sizes before the audit begins and a {\em strong adversary} as one who can choose any round size at any time (before, of course, that particular round begins). In particular, a strong adversary may use information about the sample drawn thus far to decide the next round size, but a weak adversary may not. 

\subsection{Gaps in Prior Work and Our Contributions}
We describe relevant gaps in our knowledge of audits and then describe our contributions. 
\subsubsection{Limitations of \Minerva}
Zag\'{o}rski {\em et al.}  prove that \Minerva \cite{usenix_minerva} is risk-limiting when the number of relevant ballots drawn in each round is pre-determined before any ballots are examined. They do not address the case of a stronger adversary (such as an audit insider) who can determine the size of the next round after knowing what votes are on the ballots sampled thus far. An open question about \Minerva is whether its RLA proof holds in this case. Can the audit insider increase the audit's error probability beyond its declared risk limit? Or is there no probabilistic adversarial advantage to being able to compute next round sizes after knowing the drawn sample? We do not answer this question, and to our knowledge, it remains open. 

Until \Minerva is proven to be risk-limiting to a given risk limit for the adversary who can determine next round size after examining the current sample, it may not be used in audits whose round sizes are not pre-determined. This presents a major limitation, because the stopping probability of the next round is better estimated using information of the sample drawn thus far, but this would not be allowed for \Minerva. The current implementation of \Minerva integrated as an option in Arlo uses a fixed multiplier of the current round size to compute the next round size, thus allowing the first round to be computed as desired, and fixing the next round sizes thereafter\footnote{Note that every draw may contain irrelevant ballots, and thus the true number of relevant ballots can never be predetermined. However, because this is random, and not controllable by an adversary once the size of the draw is fixed, we assume that differences in the number of ballots average out, and that a fixed draw size is sufficient, though this is not explicitly proven in \cite{usenix_minerva}.}. This could lead to the drawing of too few or too many ballots and greatly constrains audit planning.  

% In MINERVA, the number of ballots drawn in each round is determined before any ballots are drawn. Because invalid ballots and ballots that are inconsequential for the contest being audited would be drawn in addition to relevant ballots, the assumption used by the proof is not true in general. (We are grateful to Philip Stark for drawing our attention to this.) However, any variation in number of relevant ballots drawn for a fixed round size would be random and not chosen by an adversary; the proof showing the risk-limiting property of MINERVA could hence be extended.

The risk limit for B2, EoR and SO \BRAVO is fixed independent of whether next round sizes are determined with or without knowledge of the current sample. This allows \BRAVO audits the flexibility of choosing smaller subsequent round sizes if the sample drawn so far is a ``good'' sample. An open question is whether a ballot polling RLA exists with the efficiency of \Minerva and this flexibility of \BRAVO.

\subsubsection{Limitations in Existing Workload Measures}
A major limitation of our understanding of the ballot polling problem as a community is that we use the number of ballots drawn or values proportional to this number \cite{mclaughlin_thesis,bernhard-diss,RI-report} as measures of the workload of an audit. If this were a correct measure of the workload of an audit, we would want to use B2 audits (round size is one) and make decisions about stopping the audit after drawing each ballot, because this leads to the smallest expected number of ballots. As described in \ref{sec:related}, election officials, on the other hand, greatly prefer drawing many ballots at once. From conversations with election officials and staff members of Verified Voting, Brennan Center and Common Cause who have been training election officials to perform RLAs, we estimate that this preference is likely due to the following. 

Firstly, each round has an overhead workload as well, including setting up the round and communicating among the various localities involved in conducting the audit (for example, audits of statewide contests involve the drawing of ballots at county offices where the ballots are stored). Secondly, there is an overhead to finding a storage box and unsealing it. For large round sizes, multiple ballots may be drawn at once from a box, and the number of boxes retrieved is smaller than the number of ballots (storage boxes commonly contain many hundreds of ballots each). 
%For smaller round sizes, the number of times a box is retrieved would be roughly identical to the number of ballots drawn, as it is unlikely that a single box will hold multiple ballots from the sample. 
Finally, in the current environment of misinformation, election officials fear a misleading audit sample (with more votes for an announced losing candidate than the winner), preferring to structure audits to reduce the chances of such samples, thus implicitly choosing larger round sizes. 

Thus the workload of an audit is not simply a linear (or affine) function of the number of ballots drawn. Relatedly, an optimal round schedule is not completely determined by the expected number of ballots drawn. It depends on other variables as well, the consideration of which is necessary while planning an audit. 

\subsubsection{Our Contributions}
Our primary contribution is a new RLA, \Providence, which gives the efficiency of \Minerva and is also resistant to an adversary who can choose the next round size after knowing the current sample. 
%Additionally, \Providence does not assume that the null hypothesis is a tied election and hence is more broadly useful than \Minerva. 
\BRAVO is a very flexible audit enabling audit planning; with the introduction of \Providence, similar planning is possible with greater efficiency. In a contest with a narrow margin (in the 2020 US Presidential election, eight states had margins smaller than 0.03) the difference in number of ballots sampled using \Providence over \BRAVO could correspond to many days of work which would need to be completed before a certification deadline.

The stopping condition for \Minerva does not take into account the sample obtained in previous rounds, and, in \cite{usenix_minerva}, its risk limit is estimated through weighted averages across multiple rounds, assuming that round sizes do not depend on the previous sample. 
%Further, it assumes that the null hypothesis is a tie. 
Attempting to simplify the proof of \Minerva's risk-limiting property, we were able to define a different audit \Providence. The RLA proof for \Providence does not make an assumption about round sizes.  

We provide the following:
\begin{enumerate}
\item Proof that \Providence is an RLA and resistant to an adversary who can choose next round sizes after knowing the current sample.
\item Simulations of \Providence, \Minerva, SO \BRAVO, and EoR \BRAVO which show that \Providence uses number of ballots similar to those of \Minerva, both fewer than either version of \BRAVO.
\item Results and analysis from the use of \Providence in a pilot audit in Rhode Island.
\item A model of workload that includes the overhead effort of each round and the overhead effort of retrieving a storage unit of ballots; simulations that illustrate the use of this model to compare the different types of ballot polling audits and to plan an audit with minimal workload.
\item An analysis of round size as a function of the maximum acceptable probability of a misleading audit sample.
\item Open source implementation of \Providence and audit planning tools. The implementation of \Providence has been integrated as an option in Arlo. 
%including the novel metric Probability of Misleading(name?)
\end{enumerate}

\Providence may be used in any audit where sampling is with replacement and audited contests may be expressed as pairwise plurality contests: for example, it can be used in plurality and majoritarian elections. 

%Our results demonstrate the superiority of \Providence over the other audits. Our work may be used by election officials to plan ballot polling audits, including in Georgia and Pennsylvania in 2022. 

\subsubsection{Organization} 
Sections \ref{sec:model} provides preliminaries on the \BRAVO and \Minerva audits. Section \ref{sec:prov} describes the \Providence audit, section \ref{sec:sims} the simulations comparing the number of ballots drawn using various ballot polling audits and section \ref{sec:pilot} the use of \Providence in an audit carried out by the Board of Elections of Rhode Island. Section \ref{sec:workload} presents our workload model and describes its use for a ballot polling audit using details of the 2016 US Presidential election in the state of Virginia. Our conclusions, the availability of an audit implementation and acknowledgements may be found in sections \ref{sec:conc}, \ref{sec:avail} and \ref{sec:ack} respectively. 

