In this section we introduce the stopping condition of \Providence and prove some properties.

Recall that the proof that the \Minerva audit is risk-limiting assumes that the round schedule of \Minerva is predetermined and that, in particular, an adversarial auditor cannot determine the next round size after drawing a sample (we termed such an adversary a weak adversary in section \ref{sec:adv}). This presents difficulties because a non-adversarial election official might want to draw a small next round if the current sample comes close to satisfying the risk limit. Because the \Minerva round size is predetermined, however, the election official would be required to draw a larger round size than necessary for the sample. Conversely, if the current sample is not at all close to satisfying the risk limit, it would be advantageous to draw a larger round than the predetermined round size. Thus it would be advantageous to have an RLA with \Minerva's efficiency advantages in the face of an adversarial auditor who could choose next round sizes after drawing the sample (we termed such an adversary a strong adversary). This would enable honest participants to better plan audits. 

Before proceeding to define the audit \Providence which is an RLA in the presence of the stronger adversary, we formalize the notions of weak and strong adversaries introduced in section \ref{sec:adv}.

\begin{definition}[Weak Adversary]
A {\emph weak adversary} may choose the first and consequent round sizes as a pre-determined function of audit parameters. That is, the $j^{th}$ round size is a function $$n_j(\alpha, p_a, p_0)$$ determined before the audit begins. 
\end{definition}

\begin{definition}[Strong Adversary]
A {\emph strong adversary} may choose any round size as any function of audit parameters and all preceding samples. That is, the first round size is a function 
$$n_1(\alpha, p_a, p_0),$$
and for all rounds $j\ge 2$, the round size is a function
$$n_{j}(\alpha, p_a, p_0, \bm{k_{j-1}}, \bm{n_{j-1}})$$ The functions $n_j$, $j \geq 1$ may be chosen at any time before the $j^{th}$ round begins. 
\end{definition}

Before defining \Providence, we give some intuition for how it is designed to avoid the problem of \Minerva.

For any ballot polling audit, in round $j$, designed to stop for some value of $K_j$, however large $K_j$ is, there is, in general, a probability, however small, that the value of $K_j$ has resulted from an underlying incorrect election outcome. Thus, corresponding to each such value of $K_j$, a risk---probability that the audit will stop given that the outcome is incorrect---is incurred. To obtain the total risk for the round, one adds the risks corresponding to each value of $K_j$ for which the audit can stop, weighted by the probability of drawing that value of $K_j$. The stopping condition may provide relationships among the various quantities.

In \Minerva, the stopping condition relates the weighted average of the risks to the weighted average of the stopping probabilities over all values of $K_j$, for a given round size. Separate relationships between risk and stopping probability are not available for individual values of $K_j$. If the next round size depends on $K_j$, expressions relating the risks are not available, and so we are not able to say if \Minerva is an RLA in this case, or if it is vulnerable to an adversary choosing round sizes \cite{usenix_minerva, arxiv_athena}. 

In \Providence, we choose a stopping condition that applies separately to the risk and stopping probabilities for each value of $K_j$, avoiding the problem of \Minerva, and allowing for optimal round size choices, which depend on the drawn sample. 
The \Providence audit is risk-limiting even if a strong adversarial auditor determines round sizes after drawing the sample, and next round size computations may use knowledge of the current sample. 

\subsection{Definition}
\label{sec:prov_def}
\begin{definition}[$(\alpha,p_a, p_0,k_{j-1},n_{j-1},n_j)$-\Providence]
    \label{def:minervatwo}
    For cumulative round size $n_j$ for round $j$ and a cumulative $k_j$ ballots for the reported winner found in round $j$, where samples are drawn with replacement, the \R \Providence stopping rule for the $j^{th}$ round is:
$$
\mathcal{A}(X_{j})=  \left\{ \begin{array}{ll} \text{Correct} ~~~~ \omega_{j}(k_{j}, k_{j-1}, p_a, p_0, n_j, n_{j-1}) \geq \frac{1}{\alpha}\\
        % & \\
        % incorrect& ~~~ \sigma_n < \frac{\beta}{1-\alpha} \\
        % & \\
        Undetermined ~~else \\
    \end{array}
    \right .
$$
where $\omega _{1}\triangleq \tau_{1}$ and for $j\ge 2$, we define $\omega _{j}$ as follows:
\begin{equation}
    \begin{aligned}
    \omega_{j}(k_{j}, k_{j-1}, p_a, p_0, n_{j}, n_{j-1})
    \triangleq\\
    \sigma(k_{j-1},p_a,p_0,n_{j-1})\cdot \tau_1(k_{j}-k_{j-1},p_a,p_0,n_j-n_{j-1})
    \end{aligned}
\end{equation}
\end{definition}

Notice that \Providence requires the computation of $\tau_j$ for $j=1$ and no other values of $j$. The value of $\tau_1$ is simply the ratio of the the tails of the binomial distributions (the distributions are binomial because the auditing is with replacement) for the two hypotheses and can be fairly efficiently computed. The computation of $\tau_j$ for $j \geq 2$, as required in \Minerva, relies on the convolution of two probability distribution functions and is hence computationally considerably more expensive. Lesser computational complexity makes audit planning and analysis using simulations as in section~\ref{sec:workload} more feasible.

Notice also that \Providence and \Minerva are identical for $j=1$. 

\subsection{Risk-Limiting Property: Proof}
\label{sec:proof}
We now prove that \Providence is risk-limiting against a strong adversary using lemmas from basic algebra which are given in Appendix~\ref{sec:proofs}.

\begin{theorem}
\label{thm:minerva2_is_rla_new}
$(\alpha,p_a, p_0,k_{j-1},n_{j-1},n_j)$-\Providence is an
$\alpha$-RLA in the presence of a strong adversary.
\end{theorem}
\begin{proof}
Let $\mathcal{A}=(\alpha,p_a, p_0,k_{j-1},n_{j-1},n_j)$-\Providence.
Let $\bm{n_j}$ be the cumulative round sizes used in this
audit, with corresponding cumulative tallies of
ballots for the reported winner $\bm{k_j}$.
For round $j=1$, by Definitions \ref{def:minervatwo}
and \ref{def:minerva_ratio}, we see that
the $\mathcal{A}=\text{Correct}$ (the audit stops) only when
$$
\tau_1(k_{1},p_a,p_0,n_1)\\
=\frac{Pr[K_{1} \geq k_{1} \mid H_a, n_1]}{Pr[K_{1} \geq k_{1} \mid H_0, n_1]}
\ge \frac{1}{\alpha}.
$$
By Lemma \ref{lemma:minerva2_kmin_exists} and Definition \ref{def:kmin}, there is 
a value $k_{min,1} = k^{p_a, p_0, \alpha, 0}_{min, 1, 0, n_1}$ such that 
$$
\frac{Pr[K_{1} \geq k_{1} \mid H_a, n_1]}{Pr[K_{1} \geq k_{1} \mid H_0, n_1]}
\ge
\frac{Pr[K_{1} \geq k_{min,1} \mid H_a, n_1]}{Pr[K_{1} \geq k_{min, 1} \mid H_0, n_1]}
\ge 
\frac{1}{\alpha}.
$$

For any round $j\ge 2$, by Definition \ref{def:minervatwo}
and Lemma \ref{lemma:minerva2_kmin_exists},
$\mathcal{A}=\text{Correct}$ (the audit stops) if and only if
\begin{equation*}
\begin{aligned}
\omega_{j}(k_{j}, k_{j-1}, p_a, p_0, n_{j}, n_{j-1}, \alpha )\triangleq\\
\sigma(k_{j-1},p_a,p_0,n_{j-1})\cdot \tau_1(k_{j}-k_{j-1},p_a,p_0,n_j-n_{j-1})
\ge \frac{1}{\alpha}.
\end{aligned}
\end{equation*}
%Let $d_j=k_j-k_{j-1}$ be shorthand for the new draw.
%could do this for simpler notation^
By Lemma \ref{lemma:any_ratio_is_sigma_simple}
and Definition \ref{def:minerva_ratio}, this is equivalent to
$$
\frac{\Pr[K_{j-1} = {k_{j-1}} \mid H_a, n_{j-1}] Pr[K_{j} \ge k_{j} \mid {k_{j-1}}, H_a, n_{j-1}, n_{j}]}{\Pr[K_{j-1} = {k_{j-1}} \mid H_0, n_{j-1}] Pr[K_{j} \ge k_{j} \mid {k_{j-1}}, H_0, n_{j-1}, n_{j}]}
$$$$\ge \frac{1}{\alpha}.
$$
By Lemma~\ref{lemma:minerva2_kmin_exists} and Definition~\ref{def:minervatwo},
we see that there exists a $k_{min, j} = k^{p_a, p_0, \alpha, k_{j-1}}_{min, j, n_{j-1}, n_j}  \leq k_j$ 
%$k_{min, j}\le k_j$ 
for which
$$
\frac{\Pr[K_{j-1} = {k_{j-1}} \mid H_a, n_{j-1}] Pr[K_{j} \ge k_{j} \mid {k_{j-1}}, H_a, n_{j-1}, n_{j}]}{\Pr[K_{j-1} = {k_{j-1}} \mid H_0, n_{j-1}] Pr[K_{j} \ge k_{j} \mid {k_{j-1}}, H_0, n_{j-1}, n_{j}]}\ge
$$
$$
\frac{\Pr[K_{j-1} = {k_{j-1}} \mid H_a, n_{j-1}] Pr[K_{j} \ge k_{min, j} \mid {k_{j-1}}, H_a, n_{j-1}, n_{j}]}{\Pr[K_{j-1} = {k_{j-1}} \mid H_0, n_{j-1}] Pr[K_{j} \ge k_{min, j} \mid {k_{j-1}}, H_0, n_{j-1}, n_{j}]} 
$$$$
\ge 
\frac{1}{\alpha}
$$
%Equivalently,
%$$
%\frac{Pr[K_{j} \ge k_{min, j}\wedge K_{j-1} = k_{j-1} \mid n_{j}^*, k_{j-2}^*, H_a]}{Pr[K_{j}\ge k_{min, j} \wedge K_{j-1} = k_{j-1} \mid n_{j}^*, k_{j-2}^*, H_0]}\ge \frac{1}{\alpha}.
%$$
%Taking the sum over all possible rounds and corresponding preceding values
%of $k$, we get
% Taking the sum over all possible audit histories, we get
The above may be rewritten as
\begin{equation*}
\begin{aligned}
\sum_{{k} = k_{min, j}}^{n_j} Pr[(K_{j} , K_{j-1}) = (k, k_{j-1}) \mid H_0, n_{j-1}, n_{j}] \leq \\
\alpha \sum_{{k} = k_{min, j}}^{n_j} Pr[(K_{j} , K_{j-1}) = (k, k_{j-1}) \mid H_a, n_{j-1}, n_{j}]
\end{aligned}
\end{equation*}
The left hand side above is the probability of stopping in the $j^{th}$ round and $K_{j-1} = k_{j-1}$, given the null hypothesis, which is smaller than $\alpha$ times the same probability given the alternate hypothesis. 
For different possible values of $k_{j-1}$, different round sizes $n_j$ can be used, and this same relationship will hold. 
That is, the relationship holds even if the values of $n_{j}$ depend on $k_{j-1}$, if $n_j$ is a function $n_j(\alpha, p_0, p_1, \bm{k_{j-1}}, \bm{n_{j-1}})$.\footnote{\Minerva enforces a similar relationship between risk and stopping probability but does so at the level of the round rather than for each individual value of $K_{j-1}$. By enforcing this relationship for each value of $K_{j-1}$, \Providence is resistant to a strong adversary.}

Summing both sides over all values of $k_{j-1} < k_{min, j-1}$ gives us a similar relationship between the probabilities of stopping in round $j$ (given the null and alternate hypotheses respectively)\footnote{In fact, this is the relationship \Minerva enforces for its stopping condition, additionally requiring that $n_j$ be fixed for all values of $k_{j-1}$.}. 
When both sides of the inequality are further summed over all rounds, we get:  

$$
Pr[\mathcal{A}=\text{Correct} \mid H_0]
\le
\alpha Pr[\mathcal{A}=\text{Correct} \mid H_a]
$$
Finally, because the total probability of stopping the audit under
the alternative hypothesis is not greater than 1, we get
$$
Pr[\mathcal{A}=\text{Correct} \mid H_0] \le
\alpha.
$$
\end{proof}

\subsection{Consequences of resistance to an adversary choosing round size}
\label{sec:adversary}

% \subsection{Motivation for considering adaptive adversary}
% \fpo{(old notes)}

\Minerva was proven to be a risk-limiting audit for a predetermined round schedule.
As explained earlier, it is not clear that \Minerva is risk-limiting if an adversary can 
adaptively select the round schedule as the audit proceeds. In this section we prove that \Providence does not have this problem, and is risk-limiting even when the adversary can choose next round sizes based on knowledge of the current sample. 

\begin{definition}[Strategy-Proof RLA]
 An audit $\mathcal{A}$ is a Strategy-Proof Risk Limiting Audit with risk limit $\alpha$ iff for
 all strategies of selecting round schedule and for sample $X$ 
 \[
  \Pr\left[\mathcal{A}(X) = \text{Correct} | H_0\right] \leq \alpha.
 \]

\end{definition}

\begin{lemma}
\Providence is a Strategy-Proof RLA. 
\end{lemma}
\begin{proof}
This property follows from Theorem~\ref{thm:minerva2_is_rla_new} and Lemma~\ref{lemma:markov}. Note that, as described in section \ref{sec:proof}, the proof of the risk-limiting nature of the audit does not rely on round sizes $n_j$ being identical for all values of $k_{j-1}$. 
\end{proof}

To illustrate the practical implication of this property, we consider a toy example: an RLA of a two-candidate contest with margin $0.01$ and risk limit $0.1$. 
Suppose we wish to achieve a conditional stopping probability $0.9$ in each round of the audit. For \Providence, we can compute a new round size for each round based on the previous samples. For \Minerva, however, we would have a predetermined round schedule. We use the default \Minerva round schedule of audit software Arlo \cite{arlo} (used by many states performing an RLA), which is $[x, 2.5x, 6.25x, ...]$; that is, the next marginal round size is $1.5$ times the current one. This multiplier of $1.5$ is known to give, over a wide range of margins, a probability of stopping roughly $0.9$ in the second round if the first round size has probability of stopping $0.9$. 

Both the audits of our toy example therefore begin with a first round size of $17,272$ with a $0.9$ probability of stopping, and both will stop in the first round if the sample contains at least $8,725$ ballots for the winner. We now consider two cases for which the audit proceeds to a second round. 
\begin{description}
\item In one case there are $8,724$ votes for the winner in the sample, just one fewer than the minimum needed to meet the risk limit. In the \Minerva audit, we are already committed to a second round size of $43,180$ which, given the nearly-passing sample of the first round is higher than necessary, achieving a stopping probability in the second round of $.954$. The \Providence audit samples more than $9,000$ fewer ballots with a round size of $34,078$, achieving the desired $0.9$ probability of stopping.
\item In a less lucky sample, the winner recieves $8,637$ ballots, few more than the loser recieves. In the \Minerva audit, we again have to use a second round size of $43,180$, but now this round size only achieves a $0.727$ probability of stopping, significantly less than the desired $0.9$. Again, the \Providence audit can scale up the second round size according to the first sample and achieve the desired $0.9$ probability of stopping with $58,007$ ballots.
\end{description}



% \newpage

% \subsection{Round extensions}

% The end of a round of size $n_1$, with $k^{p_a, p_0, \alpha}_{min, 1}$ and observed ballots for
% the winner $k$.


% The size of the next round is selected to be $n_{r+1}$.
% The \Providence k-mins are computed as follows. 
% 
% (1) The probability distribution at the end of $(r+1)^{st}$ round is computed:
% % \[
% %  s_{i, d, [n_1, \ldots, n_{r+1}]} = \sum_{j = \max\{0, i - (n_{r+1} - n_r)\}}^{\min\{i, k^{p_a, p_0, \alpha}_{min, r}-1\}} Bin(n_{r+1} - n_r, p_d, i - j) \cdot s_{j, d, [n_1, \ldots, n_r]},
% % \]
% \[
%  s_{i, d, [n_1, \ldots, n_{r+1}]} = \sum_{j = \max\{0, i - (n_{r+1} - n_r)\}}^{\min\{i, k^{p_a, p_0, \alpha}_{min, r}-1\}} Bin(n_{r+1} - n_r, p_d, i - j) \cdot s_{j, d, [n_1, \ldots, n_r]},
% \]
% where $p_d = p_0 = \frac{1}{2}$ or $p_d = p_A$.
% 
% (2) The $k^{p_a, p_0, \alpha}_{min, r}$ is found as:
% \[
%  k^{p_a, p_0, \alpha}_{min, r} =  k^{p_a, p_0, \alpha}_{min, r, [n_1, \ldots, n_r]} = \min\left\{ k: \sum_{i = k}^{n_r} s_{i, 0, [n_1, \ldots, n_r]} \le \alpha \sum_{i = k}^{n_r} s_{i, A, [n_1, \ldots, n_r]} \right\}.
% \]

% 
% \begin{definition} 
% Let $[n_1, \ldots, n_r]$ be the round schedule of an audit that has not stopped by the round $r-1$. Let us define 
% \begin{small}
% \begin{equation}\label{eq:kMin}
% k^{p_a, p_0, \alpha, k_{r-1}}_{min, r, [n_1, \ldots, n_r]} =
%   \min\left\{k : \omega_r(k, k_{r-1},p_a,p_0,n_r, n_{r-1}) \geq \frac{1}{\alpha}  \right\}.
% %  \min\left\{k : \sigma(k_{r-1},p_a,p_0,n_{r-1})\cdot \tau_1(k-k_{r-1},p_a,p_0,n_r-n_{r-1}) \geq \frac{1}{\alpha}  \right\}$
% \end{equation}
% \end{small}
% Then if $k_r \geq k^{p_a, p_0, \alpha, k_{r-1}}_{min, r, [n_1, \ldots, n_r]}$ then the result of the audit is Correct (\textit{i.e.,} stopping condition in Definition~\ref{def:minervatwo} holds).
% \end{definition}
% 
% 
% 
% \begin{lemma}
% Let $[n_1, \ldots, n_{r-1}, n_r]$ be a round schedule for an execution of  \Providence audit that has not stopped
% in any of its first $r-1$ rounds (\textit{i.e.,} for every $j = 1, \ldots, r-1$:
% $k_j < k^{p_a, p_0, \alpha}_{min, r, [n_1, \ldots, n_j]}$), then: 
% 
% \[ 
% k^{p_a, p_0, \alpha, k_{r-1}}_{min, [n_1, \ldots, n_{r-1}, n_r]} = k^{p_a, p_0, \alpha, k_{r-1}}_{min, [n_{r-1}, n_r]}.
% \]
% \end{lemma}
% \fpo{this can be used to prove that \Providence is more efficient than \Minerva and \BRAVO}
% \begin{proof}
% Let $k_{r-1}$ denote the number of ballots drawn for the declared winner up to the round $r-1$ (out of $n_{r-1}$ sampled ballots). The stopping decision for the round $r$ is made as follows:
% 
% \[
%  k^{p_a, p_0, \alpha, k_{r-1}}_{min, r, [n_1, \ldots, n_r]} = \min\left\{k : \omega_{j}(k, k_{r-1}, p_a, p_0, n_r, n_{r-1}) \geq \frac{1}{\alpha}  \right\} = 
% \]
% \[
%   =  k^{p_a, p_0, \alpha, k_{r-1}}_{min, 2, [n_{r-1}, n_r]}
% \]
% 
% \end{proof}
% Then, from (\ref{eq:prov}) one can write it as:
% 
% % \[
% %     \begin{aligned}
% %     \omega_{j}(k, k_{r-1}, p_a, p_0, n_{r}, n_{r-1})
% %     \triangleq\\
% %     \sigma(k_{r-1},p_a,p_0,n_{r-1})\cdot \tau_1(k-k_{r-1},p_a,p_0,n_r-n_{r-1})
% %     \end{aligned}
% % \]
% \begin{small}
% \[
% \begin{aligned}
%  k^{p_a, p_0, \alpha}_{min, r, [n_1, \ldots, n_r]} =\\ 
%  \min\left\{k : \sigma(k_{r-1},p_a,p_0,n_{r-1})\cdot \tau_1(k-k_{r-1},p_a,p_0,n_r-n_{r-1}) \geq \frac{1}{\alpha}  \right\} =\\
%  \min\left\{k :
% \frac{s_{k_{r-1}, a, n_{r-1}}}{s_{k_{r-1}, 0, n_{r-1}}} \frac{
% \sum_{i = k}^{n_r} s_{i - k_{r-1}, a, [n_r - n_{r-1}]}
% }{
% \sum_{i = k}^{n_r} s_{i - k_{r-1}, 0, [n_r - n_{r-1}]}
% }\geq \frac{1}{\alpha}  \right\}
%  \end{aligned}
%  \]
% \end{small}
% 
% \fpo{should be added: for $k > k_{r-1}$ but it seems to be obvious (from monotonicity of k-min)}
% % \[
% % \sigma(k_{r-1},p_a,p_0,n_{r-1}) = \frac{s_{k_{r-1}, a, n_{r-1}}}{s_{k_{r-1}, 0, n_{r-1}}}
% % \]
% 
% % \begin{small}
% % \begin{equation}\label{eq:kMinLong}
% % k^{p_a, p_0, \alpha}_{min, r, [n_1, \ldots, n_r]} =\\ 
% %  \min\left\{k :
% % \frac{s_{k_{r-1}, a, n_{r-1}}}{s_{k_{r-1}, 0, n_{r-1}}} \frac{
% % \sum_{i = k}^{n_r} s_{i - k_{r-1}, a, [n_r - n_{r-1}]}
% % }{
% % \sum_{i = k}^{n_r} s_{i - k_{r-1}, 0, [n_r - n_{r-1}]}
% % }\geq \frac{1}{\alpha}  \right\}
% % \end{equation}
% % \end{small}
% %  = \min\left\{ k: \sum_{i = k}^{n_r} s_{i, 0, [n_1, \ldots, n_r]} \le \alpha \sum_{i = k}^{n_r} s_{i, A, [n_1, \ldots, n_r]} \right\}.
% % \]
% 
% Now, one can rewrite the enumerator and the denominator as  follows ($d \in \{a, 0\}$):
% 
% \begin{small}
% \[
% \begin{aligned}
%  s_{k_{r-1}, d, n_{r-1}} \sum_{i = k}^{n_r} s_{i - k_{r-1}, d, [n_r - n_{r-1}]} =\\
%  \mathsf{Bin}(n_{r-1}, p_d, k_{r-1}) \sum_{i = k}^{n_r} 
%  \mathsf{Bin}({n_r - n_{r-1}, d, i - k_{r-1}}) =\\
%  {n_{r-1} \choose k_{r-1}} p_d^{k_{r-1}} p_d^{n_{r-1} - k_{r-1}}
%  \sum_{i = k}^{n_r} 
%  {n_r - n_{r-1} \choose i - k_{r-1}} p_d^{i - k_{r-1}} p_d^{n_r - n_{r-1} - (i - k_{r-1})} =\\
%   \sum_{i = k}^{n_r} {n_{r-1} \choose k_{r-1}}
%  {n_r - n_{r-1} \choose i - k_{r-1}} p_d^{i} p_d^{n_r - i} =
%  \sum_{i = k}^{n_r} {n_{r} \choose i}
%  p_d^{i} p_d^{n_r - i} = \sum_{i = k}^{n_r} s_{i, d, [n_r]}.
%  \end{aligned}
% \]
% \end{small}
% 
% \begin{small}
% \[
% k^{p_a, p_0, \alpha}_{min, r, [n_1, \ldots, n_r]} =\\ 
%  \min\left\{k :
% \frac{
% \sum_{i = k}^{n_r} s_{i, a, [n_r]}
% }{
% \sum_{i = k}^{n_r} s_{i, 0, [n_r]}
% }\geq \frac{1}{\alpha}  \right\}
% = k^{p_a, p_0, \alpha}_{min, 1, [n_r]}
% \]
% \end{small}
% \fpo{There is no need to put $r$ in $k_{min}$ definition $[n_1, \ldots, n_r]$ is enough}
% 
% \end{proof}

% \fpo{Corollary 1: For RO-Bravo you need to pay attention.}
% 
% \fpo{Corollary 2: For \Providence you can (1) be late for an audit (round schedule does not
% matter much); (2) you can be distracted: only the end-of-round matters}

% \begin{lemma}
%  I
% \end{lemma}




% \begin{theorem}
%  \Providence is risk limiting for adversarial choices of round schedules.
% \end{theorem}
% \begin{proof}
% 
% \end{proof}

\subsection{Efficiency}
\begin{lemma}
\label{lem:efficiency}
% For any round schedule %$[n_1, \ldots, n_r]$
For any risk-limit $\alpha \in (0, 1)$, for any margin
and for any round schedule $[n_1, \ldots, n_j]$, 
the \Providence RLA stops before or in the same round as EoR \BRAVO.
\end{lemma}
\begin{proof} Appendix~\ref{sec:proofs}\end{proof}



