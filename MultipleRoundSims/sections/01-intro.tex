%Intro
Need motivating paragraph and two sentence summary of results here. 

\subsection{Background}
In the general ballot-polling risk limiting audit (RLA), a number of ballots are drawn and tallied in what is termed a {\em round} of ballots \cite{usenix_minerva}. A statistical measure is then computed to determine whether there is sufficient evidence to declare the election outcome correct within the pre-determined risk limit. Because the decision is made after drawing a round of ballots, the audit is termed a {\em round-by-round (R2)} audit. The special case when round size is one---that is, stopping decisions are made after each ballot draw---is a {\em ballot-by-ballot (B2)} audit. 

The \BRAVO audit is designed for use as a B2 audit: it requires the smallest expected number of ballots when the true tally of the underlying election is as announced, and stopping decisions are made after each ballot draw. In practice, election officials draw many ballots at once, and the \BRAVO stopping rule needs to be modified for use in an R2 audit that is not B2. There are two obvious approaches. The B2 stopping condition can be applied once at the end of each round: End-of Round (EoR) \BRAVO.  Alternatively, the order of ballots in the sample can be tracked by election officials and the B2 \BRAVO stopping condition can be applied retroactively after each ballot drawn: Selection-Ordered (SO) \BRAVO. SO \BRAVO requires fewer ballots on average than EoR \BRAVO but requires the work of tracking the order of ballots rather than just their tally.

\Minerva was designed for R2 audits and applies its stopping rule once for each round. Thus it does not require the tracking of ballots that SO \BRAVO does. It has been proven that \Minerva is a risk-limiting audit and requires fewer ballots to be sampled than EoR \BRAVO when an audit is performed in rounds. Computations of first-round size for a 0.9 stopping probability when the election is as announced have been computed for a wide range of margins and shown to be smaller than those for both EoR and SO \BRAVO. First round simulations of \Minerva \cite{arxiv_athena} demonstrate that its first-round properties---regarding the probabilities of stopping when the underlying election is tied and when it is as announced---are as predicted for first round sizes with stopping probability 0.9. There are no results, either theoretical or based on simulations, regarding the expected number of ballots drawn in a \Minerva audit. Further, there is no literature comparing simulations of \Minerva and EoR or SO \BRAVO, or studying the sizes of multiple-round \Minerva audits. 

\subsection{Our Results}

\subsection{Organization}

