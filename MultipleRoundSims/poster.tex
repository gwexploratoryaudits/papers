% Gemini theme
% See: https://rev.cs.uchicago.edu/k4rtik/gemini-uccs
% A fork of https://github.com/anishathalye/gemini

\documentclass[final]{beamer}

% ====================
% Packages
% ====================
\usepackage{amsfonts}
\usepackage[T1]{fontenc}
\usepackage{lmodern}
\usepackage[size=custom,width=120,height=72,scale=1.0]{beamerposter}
\geometry{paperwidth=42in,paperheight=32.5in}
\usetheme{gemini}
\usecolortheme{ucf}
\usepackage{graphicx}
\usepackage{booktabs}
\usepackage{tikz}
\usepackage{pgfplots}
\pgfplotsset{compat=1.17}

% ====================
% Lengths
% ====================

% If you have N columns, choose \sepwidth and \colwidth such that
% (N+1)*\sepwidth + N*\colwidth = \paperwidth
\newlength{\sepwidth}
\newlength{\colwidth}
\setlength{\sepwidth}{0.025\paperwidth}
\setlength{\colwidth}{0.3\paperwidth}

\newcommand{\separatorcolumn}{\begin{column}{\sepwidth}\end{column}}

% ====================
% Title
% ====================

\title{Title}

%\author{Nitish A. Gupta \inst{1} \and Nitish A. Gupta \inst{2}}
\author{Oliver Broadrick\inst{1} \and
Poorvi L. Vora\inst{1} \and
Filip Zag{\'o}rski\inst{2}\inst{3}}
% First names are abbreviated in the running head.
% If there are more than two authors, 'et al.' is used.
%
\institute[shortinst]{\inst{1} Department of Computer Science, The George Washington University[odbroadrick@gmail.com]%\thanks{Supported in part by NSF Award 2015253} 
\samelineand \inst{2} Wroclaw University of Science and Technology%\thanks{Author was partially supported by Polish National Science Centre contract number DEC-2013/09/D/ST6/03927} and 
\samelineand \inst{3} Votifica
}
%

%\institute[shortinst]{\inst{1} \textit{University of Central Florida, Orlando, FL} \samelineand \inst{2} Another Institute}

% ====================
% Footer (optional)
% ====================

\footercontent{
  \href{https://www.example.com}{https://www.example.com} \hfill
  ABC Conference 2025, New York --- XYZ-1234 \hfill
  \href{mailto:alyssa.p.hacker@example.com}{nitish.gupta@example.com}}
% (can be left out to remove footer)

% ====================
% Logo (optional)
% ====================

% use this to include logos on the left and/or right side of the header:
% \logoright{\includegraphics[height=7cm]{logo1.pdf}}
% \logoleft{\includegraphics[height=7cm]{logo2.pdf}}

% ====================
% Body
% ====================

\begin{document}
\addtobeamertemplate{headline}{}
{
    \begin{tikzpicture}[remember picture,overlay]
      \node [anchor=north west, inner sep=3cm] at ([xshift=0.0cm,yshift=3cm]current page.north west)
      {\includegraphics[height=8.5cm]{logos/ucf_logo2.png}}; % also try shield-white.eps
      \node [anchor=north east, inner sep=3cm] at ([xshift=0.0cm,yshift=3cm]current page.north east)
      {\includegraphics[height=8.5cm]{logos/lab_logo.jpg}};
    \end{tikzpicture}
}

\begin{frame}[t]
\begin{columns}[t]
\separatorcolumn

\begin{column}{\colwidth}

  \begin{block}{A block title}

    Some block contents, followed by a diagram, followed by a dummy paragraph.

    \begin{figure}
      \centering
      \begin{tikzpicture}[scale=6]
        \draw[step=0.25cm,color=gray] (-1,-1) grid (1,1);
        \draw (1,0) -- (0.2,0.2) -- (0,1) -- (-0.2,0.2) -- (-1,0)
          -- (-0.2,-0.2) -- (0,-1) -- (0.2,-0.2) -- cycle;
      \end{tikzpicture}
      \caption{A figure caption.}
    \end{figure}

    Lorem ipsum dolor sit amet, consectetur adipiscing elit. Morbi ultricies
    eget libero ac ullamcorper. Integer et euismod ante. Aenean vestibulum
    lobortis augue, ut lobortis turpis rhoncus sed. Proin feugiat nibh a
    lacinia dignissim. Proin scelerisque, risus eget tempor fermentum, ex
    turpis condimentum urna, quis malesuada sapien arcu eu purus.

  \end{block}

% here
definitions.

\begin{definition}
An audit $\mathcal{A}$ takes a sample of ballots $X$ as input and gives as output either
(1) $Correct$: the audit is complete, or (2) $Uncertain$: continue the audit.
\end{definition}

All of the audits discussed in this paper are modeled as binary hypothesis tests. Under the alternative hypothesis, $H_a$, the announced outcome is correct. In particular, the true underlying ballot distribution is given by the announced ballot tallies. Under the null hypothesis, $H_0$, a tie is the correct outcome \footnote{or the announced winner lost by one vote, and the number of ballots is large enough that the probability of drawing a ballot for the winner is that of drawing one for the loser}.
The maximum risk of an audit is the probability that an audit stops, given that the underlying election is a tie \cite{Bayesian-RLA}. Note that an audit $\mathcal{A}$ includes all audit parameters (maximum risk, round sizes, etc.). 

\begin{definition}[Risk]
The maximum risk $R$ of audit $\mathcal{A}$ with sample $X\in \{0,1\}^*$ drawn from 
the true underlying distribution of ballots is
$R(\mathcal{A})=\Pr[\mathcal{A}(X)=Correct \mid H_0].$
\end{definition}

This leads us to the following definition of an $\alpha$-RLA.
\begin{definition}[Risk Limiting Audit ($\alpha$-RLA)]
An audit $\mathcal{A}$ is a Risk Limiting Audit with 
risk limit $\alpha$ iff 
$R(\mathcal{A}) \le \alpha.$
\end{definition}

We present measures of stopping probability in the $j^{th}$ round of the audit, given that the underlying election is as announced.
\begin{definition}[Stopping Probability]
The stopping probability $S_j$ of an audit $\mathcal{A}$ in round $j$ is 
$$S_j(\mathcal{A})=\Pr[\mathcal{A}(X)=Correct ~in~round~j~\land \mathcal{A}(X) \neq Correct ~previously \mid H_a]$$
\end{definition}
Experimentally, using our simulations, $S_j$ would be estimated by the fraction of audits that stop in round $j$. Note that $\sum _j S_j(\mathcal{A})=1$. We can also consider the cumulative stopping probability: 
\begin{definition}[Cumulative Stopping Probability]
The cumulative stopping probability $C_j$ of an audit $\mathcal{A}$ in round $j$ is $C_j(\mathcal{A})= \sum_{i=1}^j S_j$
\end{definition}
Experimentally, using our simulations, $C_j$ would be estimated by the fraction of audits that stop in or before round $j$. 

Finally, we are also interested in the probability that an audit will stop in round $j$ given that it did not stop earlier: 
\begin{definition}[Conditional Stopping Probability]
The conditional stopping probability  of an audit $\mathcal{A}$ in round $j$ is 
$$\chi_j (\mathcal{A})=\Pr[\mathcal{A}(X)=Correct ~in~round~j~\mid H_a \land \mathcal{A}(X) \neq Correct ~previously]$$
\end{definition}
Experimentally, using our simulations, $\chi_j$ would be estimated by the ratio of the audits that stop in round $j$ to those that ``entered'' round $j$, i.e. those that did not stop before round $j$. 

We simulated audits for a risk limit of $10\%$ (as in \cite{bravo} and \cite{usenix_minerva}) using margins from the 2020 US Presidential election, limiting ourselves to pairwise margins for the two main candidates of $0.05$ or larger. 
Note that both \BRAVO and \Minerva can be extended for multiple-candidate, multiple-winner plurality contests by performing pairwise tests between the winners and the losers\cite{RLA, arxiv_athena}. Therefore, the two candidate plurality contest is a general case, and these simulations provide insight for multiple-candidate and multiple-winner contests too.
Round sizes increase roughly proportional to the inverse
square of the margin, so 
smaller margins are computationally much more expensive to simulate.
For each of these states, we simulated 
$10,000=10^4$ audits assuming the underlying election was as announced ($H_a$),  
and an additional $10,000=10^4$ audits assuming the underlying election was a tie ($H_0$). 
  \end{block}
  \begin{block}{A block containing a list}

    Nam vulputate nunc felis, non condimentum lacus porta ultrices. Nullam sed
    sagittis metus. Etiam consectetur gravida urna quis suscipit.

    \begin{itemize}
      \item \textbf{Mauris tempor} risus nulla, sed ornare
      \item \textbf{Libero tincidunt} a duis congue vitae
      \item \textbf{Dui ac pretium} morbi justo neque, ullamcorper
    \end{itemize}

    Eget augue porta, bibendum venenatis tortor.

  \end{block}

  \begin{alertblock}{A highlighted block}

    This block catches your eye, so \textbf{important stuff} should probably go
    here.

    Curabitur eu libero vehicula, cursus est fringilla, luctus est. Morbi
    consectetur mauris quam, at finibus elit auctor ac. Aliquam erat volutpat.
    Aenean at nisl ut ex ullamcorper eleifend et eu augue. Aenean quis velit
    tristique odio convallis ultrices a ac odio.

    \begin{itemize}
      \item \textbf{Fusce dapibus tellus} vel tellus semper finibus. In
        consequat, nibh sed mattis luctus, augue diam fermentum lectus.
      \item \textbf{In euismod erat metus} non ex. Vestibulum luctus augue in
        mi condimentum, at sollicitudin lorem viverra.
      \item \textbf{Suspendisse vulputate} mauris vel placerat consectetur.
        Mauris semper, purus ac hendrerit molestie, elit mi dignissim odio, in
        suscipit felis sapien vel ex.
    \end{itemize}

    Aenean tincidunt risus eros, at gravida lorem sagittis vel. Vestibulum ante
    ipsum primis in faucibus orci luctus et ultrices posuere cubilia Curae.

  \end{alertblock}

\end{column}

\separatorcolumn

\begin{column}{\colwidth}

  \begin{block}{A block containing an enumerated list}

    Vivamus congue volutpat elit non semper. Praesent molestie nec erat ac
    interdum. In quis suscipit erat. \textbf{Phasellus mauris felis, molestie
    ac pharetra quis}, tempus nec ante. Donec finibus ante vel purus mollis
    fermentum. Sed felis mi, pharetra eget nibh a, feugiat eleifend dolor. Nam
    mollis condimentum purus quis sodales. Nullam eu felis eu nulla eleifend
    bibendum nec eu lorem. Vivamus felis velit, volutpat ut facilisis ac,
    commodo in metus.

    \begin{enumerate}
      \item \textbf{Morbi mauris purus}, egestas at vehicula et, convallis
        accumsan orci. Orci varius natoque penatibus et magnis dis parturient
        montes, nascetur ridiculus mus.
      \item \textbf{Cras vehicula blandit urna ut maximus}. Aliquam blandit nec
        massa ac sollicitudin. Curabitur cursus, metus nec imperdiet bibendum,
        velit lectus faucibus dolor, quis gravida metus mauris gravida turpis.
      \item \textbf{Vestibulum et massa diam}. Phasellus fermentum augue non
        nulla accumsan, non rhoncus lectus condimentum.
    \end{enumerate}

  \end{block}

  \begin{block}{Fusce aliquam magna velit}

    Et rutrum ex euismod vel. Pellentesque ultricies, velit in fermentum
    vestibulum, lectus nisi pretium nibh, sit amet aliquam lectus augue vel
    velit. Suspendisse rhoncus massa porttitor augue feugiat molestie. Sed
    molestie ut orci nec malesuada. Sed ultricies feugiat est fringilla
    posuere.

    \begin{figure}
      \centering
      \begin{tikzpicture}
        \begin{axis}[
            scale only axis,
            no markers,
            domain=0:2*pi,
            samples=100,
            axis lines=center,
            axis line style={-},
            ticks=none]
          \addplot[red] {sin(deg(x))};
          \addplot[blue] {cos(deg(x))};
        \end{axis}
      \end{tikzpicture}
      \caption{Another figure caption.}
    \end{figure}

  \end{block}

  \begin{block}{Nam cursus consequat egestas}

    Nulla eget sem quam. Ut aliquam volutpat nisi vestibulum convallis. Nunc a
    lectus et eros facilisis hendrerit eu non urna. Interdum et malesuada fames
    ac ante \textit{ipsum primis} in faucibus. Etiam sit amet velit eget sem
    euismod tristique. Praesent enim erat, porta vel mattis sed, pharetra sed
    ipsum. Morbi commodo condimentum massa, \textit{tempus venenatis} massa
    hendrerit quis. Maecenas sed porta est. Praesent mollis interdum lectus,
    sit amet sollicitudin risus tincidunt non.

    Etiam sit amet tempus lorem, aliquet condimentum velit. Donec et nibh
    consequat, sagittis ex eget, dictum orci. Etiam quis semper ante. Ut eu
    mauris purus. Proin nec consectetur ligula. Mauris pretium molestie
    ullamcorper. Integer nisi neque, aliquet et odio non, sagittis porta justo.

    \begin{itemize}
      \item \textbf{Sed consequat} id ante vel efficitur. Praesent congue massa
        sed est scelerisque, elementum mollis augue iaculis.
        \begin{itemize}
          \item In sed est finibus, vulputate
            nunc gravida, pulvinar lorem. In maximus nunc dolor, sed auctor eros
            porttitor quis.
          \item Fusce ornare dignissim nisi. Nam sit amet risus vel lacus
            tempor tincidunt eu a arcu.
          \item Donec rhoncus vestibulum erat, quis aliquam leo
            gravida egestas.
        \end{itemize}
      \item \textbf{Sed luctus, elit sit amet} dictum maximus, diam dolor
        faucibus purus, sed lobortis justo erat id turpis.
      \item \textbf{Pellentesque facilisis dolor in leo} bibendum congue.
        Maecenas congue finibus justo, vitae eleifend urna facilisis at.
    \end{itemize}
    
    Nulla eget sem quam. Ut aliquam volutpat nisi vestibulum convallis. Nunc a
    lectus et eros facilisis hendrerit eu non urna. Interdum et malesuada fames
    ac ante \textit{ipsum primis} in faucibus. Etiam sit amet velit eget sem
    euismod tristique. Praesent enim erat, porta vel mattis sed, pharetra sed
    ipsum. Morbi commodo condimentum massa, \textit{tempus venenatis} massa
    hendrerit quis. Maecenas sed porta est. Praesent mollis interdum lectus,
    sit amet sollicitudin risus tincidunt non.
  \end{block}

\end{column}

\separatorcolumn

\begin{column}{\colwidth}

  \begin{block}{A block containing some math}

    Nullam non est elit. In eu ornare justo. Maecenas porttitor sodales lacus,
    ut cursus augue sodales ac.

    $$
    \int_{-\infty}^{\infty} e^{-x^2}\,dx = \sqrt{\pi}
    $$

    Interdum et malesuada fames $\{1, 4, 9, \ldots\}$ ac ante ipsum primis in
    faucibus. Cras eleifend dolor eu nulla suscipit suscipit. Sed lobortis non
    felis id vulputate.

    \heading{A heading inside a block}

    Praesent consectetur mi $x^2 + y^2$ metus, nec vestibulum justo viverra
    nec. Proin eget nulla pretium, egestas magna aliquam, mollis neque. Vivamus
    dictum $\mathbf{u}^\intercal\mathbf{v}$ sagittis odio, vel porta erat
    congue sed. Maecenas ut dolor quis arcu auctor porttitor.

    \heading{Another heading inside a block}

    Sed augue erat, scelerisque a purus ultricies, placerat porttitor neque.
    Donec $P(y \mid x)$ fermentum consectetur $\nabla_x P(y \mid x)$ sapien
    sagittis egestas. Duis eget leo euismod nunc viverra imperdiet nec id
    justo.

  \end{block}

  \begin{block}{Nullam vel erat at velit convallis laoreet}

    Class aptent taciti sociosqu ad litora torquent per conubia nostra, per
    inceptos himenaeos. Phasellus libero enim, gravida sed erat sit amet,
    scelerisque congue diam. Fusce dapibus dui ut augue pulvinar iaculis.

    \begin{table}
      \centering
      \begin{tabular}{l r r c}
        \toprule
        \textbf{First column} & \textbf{Second column} & \textbf{Third column} & \textbf{Fourth} \\
        \midrule
        Foo & 13.37 & 384,394 & $\alpha$ \\
        Bar & 2.17 & 1,392 & $\beta$ \\
        Baz & 3.14 & 83,742 & $\delta$ \\
        Qux & 7.59 & 974 & $\gamma$ \\
        \bottomrule
      \end{tabular}
      \caption{A table caption.}
    \end{table}

    Donec quis posuere ligula. Nunc feugiat elit a mi malesuada consequat. Sed
    imperdiet augue ac nibh aliquet tristique. Aenean eu tortor vulputate,
    eleifend lorem in, dictum urna. Proin auctor ante in augue tincidunt
    tempor. Proin pellentesque vulputate odio, ac gravida nulla posuere
    efficitur. Aenean at velit vel dolor blandit molestie. Mauris laoreet
    commodo quam, non luctus nibh ullamcorper in. Class aptent taciti sociosqu
    ad litora torquent per conubia nostra, per inceptos himenaeos.

    Nulla varius finibus volutpat. Mauris molestie lorem tincidunt, iaculis
    libero at, gravida ante. Phasellus at felis eu neque suscipit suscipit.
    Integer ullamcorper, dui nec pretium ornare, urna dolor consequat libero,
    in feugiat elit lorem euismod lacus. Pellentesque sit amet dolor mollis,
    auctor urna non, tempus sem.

  \end{block}

  \begin{block}{References}

    \nocite{*}
    \footnotesize{\bibliographystyle{plain}\bibliography{poster}}

  \end{block}

\end{column}

\separatorcolumn
\end{columns}
\end{frame}

\end{document}

