% Model

We now summarize the model drawing largely from the notation and terminology of \cite{usenix_minerva,arxiv_athena,simulations,bravo}. The model is related work and not claimed to be original to this work. 

An audit $\mathcal{A}$ is a function that takes as input the sample of ballots and outputs either (1) \emph{Correct: stop the audit} or (2) \emph{Undetermined: sample more ballots}. As described in the introduction, we ignore a third option of moving to a hand count, leaving that decision to election officials. This is standard in the literature, see, for example, \cite{bravo}. We provide more detail on the use of this option in section \ref{sec:related}, on related work. 

\BRAVO and \Minerva are modeled as binary hypothesis tests where the null hypothesis $H_0$ corresponds to a tied election and the alternative hypothesis $H_a$ to an election tally as announced. 
(When the number of ballots is odd, $H_0$ corresponds to the announced loser winning by one ballot.)
Thus the null hypothesis is the outcome distinct from the announced one which is most difficult to detect; the probability of failing to detect it, given that the null hypothesis is true, is the worst case such probability and should be below the risk limit \cite{Bayesian-RLA}.

\begin{definition}[Risk Limiting Audit ($\alpha$-RLA)]
An audit $\mathcal{A}$ is a Risk Limiting Audit with 
risk limit $\alpha$ iff for sample $X$
$$
Pr[\mathcal{A}(X) 
= \text{Correct} \,|\, H_0]\le \alpha
$$
\end{definition}

The stopping conditions of \BRAVO and \Minerva rely on the following ratios.

\begin{definition}[\BRAVO Ratio] \label{def:bravo-ratio} The \BRAVO audit uses the ratio $\sigma$. Consider a sample size of $n$ ballots with $k$ for the reported winner. The proportion of ballots for the reported winner under the alternative hypothesis and null hypothesis are $p_a$ and $p_0$ respectively.
\begin{equation}
    \sigma(k, p_a, p_0, n) \triangleq \frac{p_a^{k} (1-p_a)^{n-k}}{p_0^{k} (1-p_0)^{n-k}} 
    \label{eqn:bravoratio}
\end{equation}
\end{definition}

In \BRAVO, $p_0=\frac{1}{2}$. A \BRAVO audit outputs correct if and only if
$$\sigma(k,p_a,\frac{1}{2},n)\ge \frac{1}{\alpha}.$$

%If testing the \BRAVO stopping condition after each individual ballot
% is drawn (a \B \BRAVO audit), 
If $K$ is the random variable indicating the number of ballots in the sample that contain a vote for the reported winner, it is easy to see that the ratio $\sigma$ is the likelihood ratio:
$$
\frac{Pr[K=k|H_a,n]}{Pr[K=k|H_0,n]}= \frac{\binom{n}{k}p_a^{k} (1-p_a)^{n-k}}{\binom{n}{k}(\frac{1}{2})^n} =\sigma(k, p_a, \frac{1}{2}, n)
$$

\begin{comment}
\begin{definition}[$(\alpha,p)$-\BRAVO ]\label{def:bravo}  An audit $\mathcal{A}$ is the \B~$(\alpha, p)$-\BRAVO audit iff the following stopping condition is tested at each ballot draw. If the sample $X$ is of size $n$ and has $k$ ballots for the winner,  
\begin{equation}
    \mathcal{A}(X) =  \left\{ \begin{array}{ll} \text{Correct} & ~\sigma(k, p, \frac{1}{2}, n) 
         %\triangleq \frac{p^{k} (1-p)^{n-k}}{(\frac{1}{2})^n} 
        \geq \frac{1}{\alpha}\\
        Undetermined & ~else 
    \end{array}
    \right .
    \label{eqn:bravo}
\end{equation}
\end{definition}
\end{comment}

Where \BRAVO uses the ratio of the values of the probability distribution functions, \Minerva uses the ratio of their \emph{tails}. Now it becomes useful to have shorthand for a sequence of cumulative round sizes and the corresponding sequence
of cumulative winner ballot tallies.
We use:
$$\bm{n_j}\triangleq(n_1,n_2,\ldots,n_j) \quad\text{and}\quad \bm{k_j}\triangleq(k_1,k_2,\ldots,k_j)$$
Also, let $K_j$ be the random variable indicating the cumulative number of ballots in the sample after the $j$th round is drawn.

\begin{definition}[\Minerva Ratio] \label{def:minerva_ratio} The \R \Minerva audit uses the ratio $\tau_j$. We use cumulative round sizes $\bm{n_j}$, with corresponding $\bm{k_j}$ ballots for the reported winner in each round. The proportion of ballots for the reported winner under the alternative hypothesis and null hypothesis are $p_a$ and $p_0$ respectively.
         \begin{equation}
         \begin{aligned}
             \label{eqn:tau}
                 \tau_{j}(k_{j}, p_a,p_0, \bm{n_j}, \alpha )  \triangleq\\
                 \frac{Pr[K_{j} \geq k_{j} \wedge \forall_{i < j} ({\mathcal{A}}(X_i) ~\neq \text{Correct}) \mid H_a, \bm{n_j}]}{Pr[K_{j} \geq k_{j} \wedge \forall_{i < j} ({\mathcal{A}}(X_i) ~\neq \text{Correct}) \mid H_0, \bm{n_j}]}
         \end{aligned}
         \end{equation}
\end{definition}

Notice that the ratios for \BRAVO and \Minerva are for two candidates. They may both be extended to multi-candidate plurality elections through the computation of $\ell -1$ audits for $\ell$ candidates by pairing the announced winner with all other candidates. 

The {\em margin}, given a contest and two candidates, is the difference between the votes received by each as a fraction of the total number of votes. 

\begin{comment}
\begin{definition}[$ (\alpha, p, \bm{n_j} ) $-\Minerva]
     \label{def:minerva}
     Given \B $(\alpha, p)$-\BRAVO and cumulative round sizes\\ $\bm{n_j}$, the corresponding \R \Minerva stopping rule for the $j^{th}$ round is:
 \begin{equation}
     \mathcal{A}(X_{j})=  \left\{ \begin{array}{ll} \text{Correct} ~~~~ \tau_{j}(k_{j}, p_a, \frac{1}{2}, \bm{n_j}, \alpha ) \geq \frac{1}{\alpha}\\
             % & \\
             % incorrect& ~~~ \sigma_n < \frac{\beta}{1-\alpha} \\
             % & \\
             Undetermined ~~else \\
         \end{array}
         \right .
         \label{eqn:minerva-test}
 \end{equation}
\end{definition}
\end{comment}



