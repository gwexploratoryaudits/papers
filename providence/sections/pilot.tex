% pilot
The Rhode Island Board of Elections performed a pilot audit in Providence 
in February 2022. The contest audited was a single yes-or-no question 
%on School Construction and Renovation Projects
from the November 2021 special election. 
The question had announced margin $25.67\%$.
The audit used risk-limit $10\%$.
A first round size of $140$ ballots with large probability of stopping ($95\%$) was selected
in order to give the potential for more interesting analysis afterwards; the large margin 
made a first round with large stopping probability practical.
Selection order was tracked for the sake of analysis.
As expected, the audit concluded in the first round with a \Providence risk of $4.18\%$. Table~\ref{tab:pilot-risks} shows risk measures for the drawn sample using \Minerva and \BRAVO (both EoR and SO).

\begin{table}
\begin{center}
\begin{tabular}{ |c|c|c|c|c| } 
\hline
%\diagbox[dir=NW]{First \\Round \\Size}{RLA}
ballots& \rotatebox{45}{\Providence} & \rotatebox{45}{\Minerva} & \rotatebox{45}{SO \BRAVO} & \rotatebox{45}{EoR \BRAVO} \\
\hline
140 & \bf{4.18\%} & \bf{4.18\%} & \bf{5.41\%} & 36.6\% \\
\hline
\end{tabular}
\end{center}
\caption{Risk measures for the drawn first round of $140$ ballots in the Providence, RI pilot audit. Risks in bold meet the risk-limit ($10\%$) and thus correspond to audits that would stop.}
\label{tab:pilot-risks}
\end{table}

TODO: Add examples of how the audits perform for various hypothetical round schedules. I wait to do this until I'm done with the workload estimates since the examples here should be chosen to motivate that section.
